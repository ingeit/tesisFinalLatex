%%%%%%%%%%%%%%%%%%%%%%%%%%%%%%%%%%%%%%%%%%%%%%%%%%%%%%%%
%   |------------------------------------------|       %
%   | Web App embebida en dispositivos móviles |       %
%   |  para la gestión de registros sobre la   |       %
%   |   contaminación de afluentes y ríos.     |       %
%   |                                          |       %
%   |          Proyecto de graduación          |       %
%   |__________________________________________|       %
%                                                      %
%   Autores                                            %
%   -------                                            %
%                                                      %
% * Bruno, Ricardo Hugo (CX 1409686)                   %
%     rburnount@gmail.com                              %
% * Gómez Véliz, Kevin Shionen (CX 1411828)            %
%     ing.gomezvelizkevin@gmail.com                    %
%                                                      %
%   Tutor                                              %
%   -------                                            %
%                                                      %
% * Ing. Cohen, Daniel Eduardo                         %
%        dcohen.tuc@gmail.com                          %
%                                                      %
%   Cotutor                                            %
%   -------                                            %
%                                                      %
% * Ing. Nieto, Luis Eduardo                           %
%        lnieto@herrera.unt.edu.ar                     %
%                                                      %
%                                                      %
%%%%%%%%%%%%%%%%%%%%%%%%%%%%%%%%%%%%%%%%%%%%%%%%%%%%%%%%

% ********* Introducción ********** %

%TODO: *Agregar Secciones de Android

\chapter{Introducción}
\label{chap:introduccion}

El proyecto surge del convenio de la cátedra de Sistemas con Microprocesadores de la Facultad de Ciencias Exactas y Tecnología de la UNT, con la Facultad de Ciencias Naturales e Instituto Miguel Lillo, para le desarrollo de una solución tecnológica para el control del estado de contaminación del agua en ríos y afluentes en la provincia de Tucumán.

Este proyecto consiste en mejorar los procesos actuales que se llevan acabo para el control de contaminación del agua. El mismo tiene como objetivo principal ayudar al medio ambiente, utilizando un sistema de gestión que permite la creación y administración de registros, los cuales contienen datos de muestras del universo de estudio que, al ser procesadas, brinda el estado de contaminación de un río o afluente, mediante indicadores biológicos (diferentes especies de insectos). 
 
Esto se logra mediante salidas de campo de alumnos de escuelas rurales (usuarios de la aplicación) donde se busca obtener una muestra de agua del afluente en estudio; una vez obtenida la muestra y mediante el uso de la aplicación, se establece la biodiversidad que allí se encuentra y se calcula el índice de contaminación.

Por otro lado, la tecnología de los dispositivos móviles ha avanzado rápidamente en los últimos años, llegando a ser actualmente auténticas computadoras de bolsillo. La gran demanda por este tipo de dispositivos genera un gran interés por parte de empresas/instituciones que desean crear aplicaciones para un mercado en pleno auge, buscando aprovechar no solo la gran cantidad de usuarios de estas plataformas, sino también la posibilidad de ofrecer nuevas funcionalidades y capacidades, previamente imposibles, para sus procesos actuales.

Por ello, el objetivo del presente trabajo de graduación es aplicar dichas tecnologías, los conocimientos y competencias adquiridas a lo largo de la carrera, en la construcción de un producto de software que satisface las necesidades y genera valor agregado a un cliente determinado. Dicho producto es una aplicación web embebida en dispositivos móviles, que es el tipo de producto que experimenta un vertiginoso crecimiento en la actualidad.  \newpage


\section{Objetivos del sistema}

\textbf{Objetivo general}:
\newline

Generar registros digitales, de muestras del agua de ríos/afluentes, realizados por los alumnos en salidas de campos para el estudio de la contaminación.

Estos registros deben contener fotos de las muestras (insectos encontrados en el agua), coordenadas geográficas (obtenida mediante GPS integrado del dispositivo móvil), foto del paisaje para futuras referencias, y el indice de contaminación en cuestión (calculado en base a las diferentes especies de insectos encontrados).
\newline

\textbf{Objetivos principales} 

\begin{itemize}
    \item Permitir que los usuarios del sistema puedan realizar el estudio de campo de una manera rápida y eficiente valiéndose de la tecnología de un dispositivo móvil.
    \item El sistema cuenta de dos partes, una aplicación móvil para generar registros de las muestras del estudio de campo, y una aplicación web para gestionar y administrar dichos registros.
\end{itemize}

El sistema debe diseñarse para:

\begin{itemize}
    \item Asegurar la escalabilidad de los requisitos.
    \item Mantener de forma sencilla la plataforma.
    \item Promover la seguridad de la información en todas sus capas.
    \item Ser fácil de usar.
\end{itemize}

\section{Selección del modelo de ciclo de vida y de la metodología de desarrollo}

El desarrollo iterativo evolutivo, en contraste con el ciclo de vida de cascada o secuencial, consiste en la programación y pruebas tempranas de un sistema parcial en ciclos repetitivos. 

Normalmente supone que el desarrollo se inicia antes de que todos los requisitos están definidos en detalle; el feedback se utiliza para aclarar y mejorar las características cambiantes.

El prototipado como ciclo de vida se basa en la construcción de un prototipo que ayude a comprender los requisitos del sistema. Los prototipos se usan para verificar la viabilidad del diseño del software. Sirven como una herramienta iterativa del desarrollo del software donde el prototipo evoluciona hasta llegar al sistema final. 

La metodología Script o V-Script es una metodología de desarrollo de software que tiene un alto componente dinámico, orientado hacia la interfaz de usuario. Se adapta perfectamente al paradigma de orientación a objetos, aunque se han usado técnicas Script en metodologías estructuradas para el diseño de interfaz de usuario.
Mediante el proceso Script se capturan las necesidades del usuario con la construcción de maquetas o prototipos desechables, tratando de capturar la expectativa del usuario: qué es lo que el usuario espera que haga el producto. A su vez, define las interfaces de usuario y permite integrar los aspectos del modelo estático y funcional.
