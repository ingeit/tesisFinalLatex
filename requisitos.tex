%%%%%%%%%%%%%%%%%%%%%%%%%%%%%%%%%%%%%%%%%%%%%%%%%%%%%%%%
%   |------------------------------------------|       %
%   | Web App embebida en dispositivos móviles |       %
%   |  para la gestión de registros sobre la   |       %
%   |   contaminación de afluentes y ríos.     |       %
%   |                                          |       %
%   |          Proyecto de graduación          |       %
%   |__________________________________________|       %
%                                                      %
%   Autores                                            %
%   -------                                            %
%                                                      %
% * Bruno, Ricardo Hugo (CX 1409686)                   %
%     rburnount@gmail.com                              %
% * Gomez Veliz, Kevin Shionen (CX 1411828)            %
%     ing.gomezvelizkevin@gmail.com                    %
%                                                      %
%   Tutor                                              %
%   -------                                            %
%                                                      %
% * Ing. Cohen, Daniel Eduardo                         %
%        dcohen.tuc@gmail.com                          %
%                                                      %
%   Cotutor                                            %
%   -------                                            %
%                                                      %
% * Ing. Nieto, Luis Eduardo                           %
%        lnieto@herrera.unt.edu.ar                     %
%                                                      %
%                                                      %
%%%%%%%%%%%%%%%%%%%%%%%%%%%%%%%%%%%%%%%%%%%%%%%%%%%%%%%%

\chapter{Disciplina de Requisitos}
\label{chap:requisito}

\section{Introducción}

Esta especificación tiene como objetivo analizar y documentar las necesidades funcionales que deberán ser soportadas por el sistema a desarrollar. Para ello, se identificarán los requisitos que ha de satisfacer el nuevo sistema mediante entrevistas, el estudio de los problemas de las unidades afectadas y sus necesidades actuales. Además de identificar los requisitos se deberán establecer las prioridades, los cual proporciona un punto de referencia para validar el sistema final que compruebe que se ajusta a las necesidades del usuario.

\section{Identificación de usuarios participantes}

Los objetivos de esta tarea son identificar a los responsables de cada una de las unidades y a los principales usuarios implicados. Para ello se consideran los siguientes aspectos:

\begin{itemize}
    \item Incorporación de usuarios al equipo de proyecto.
    \item Conocimiento de los usuarios de las funciones a automatizar.
    \item Repercusión del nuevo sistema sobre las actividades actuales de los usuarios.
    \item Implicaciones legales del nuevo sistema.
 
\end{itemize}

Se identificaron los siguientes usuarios:

 \begin{itemize}
 
\item \emph{Grupo de Administradores:} Formado por los solicitantes del software en cuestión.
 
\item \emph{Grupo de Alumnos:} Formado principalmente por alumnos de escuelas/colegios que realizan muestras, las cuales generan registros en el sistema.

 \end{itemize}
 
 Es de destacar la necesidad de una participación activa de los usuarios del futuro sistema en las actividades de desarrollo del mismo, con objeto de conseguir la máxima adecuación del sistema a sus necesidades y facilitar el conocimiento paulatino de dicho sistema, permitiendo una rápida implantación.

  \section{Educción de requisitos}

  \subsection{Estudio de Documentación. Planificación y Realización de Entrevistas}
  
  Esta tarea tiene como finalidad capturar los requisitos de usuarios para el desarrollo del sistema.
  
  Para el análisis de requisitos se usaron distintas técnicas de educción de requisitos. Entre ellas el estudio de la documentación provista por parte del administrador; entrevistas abiertas y estructuradas, análisis del proceso actual.

\section{Especificación de Requisitos de Software}

\renewcommand{\thesubsection}{\arabic{subsection}}
 \subsection{Introducción}
 
 Este documento es una Especificación de Requisitos Software de la  Web App embebida en dispositivos móviles para la gestión de registros sobre la contaminación de afluentes y ríos. Esta documentación es fruto de las entrevistas, estudio de la documentación y del funcionamiento del proceso actual, así como del análisis llevado a cabo por el equipo de desarrollo.

El objetivo de la especificación es definir en forma clara, precisa, completa y verificable todas las funcionalidades y restricciones del sistema que se desea construir.

Esta documentación está sujeta a revisiones por el grupo de administradores que se recogerán por medio de sucesivas versiones del documento, hasta alcanzar la aprobación por parte de los mismos. Una vez aprobado, servirá de base al equipo de desarrollo para la construcción del sistema en cuestión.

Esta especificación se ha realizado de acuerdo al estándar “IEEE Recomended Practice for software Requirements Specifications(IEEE/ANSI 830-1993)”.


    \subsection{Objetivos y alcance del sistema}

    El presente proyecto tiene como objetivo principal ayudar al medio ambiente, utilizando un sistema de gestión que permite la creación y administración de registros, los cuales contienen datos de muestras del universo de estudio que, al ser procesadas, brinda el estado de contaminación de un río o afluente, mediante indicadores biológicos.
    Estos registros cuentan con contenido multimedia y coordenadas geograficas

\section{Definiciones, acrónimos y abreviaturas}

    \subsubsection{Definiciones:}
  
        \begin{itemize}
 
        \item \emph{Salida de campo:} La principal aportación de la salida de campo es que permite al alumnado adquirir un aprendizaje significativo en el que el principal elemento del proceso de enseñanza- aprendizaje es la construcción de significados. La persona aprende un concepto, un fenómeno, un procedimiento, un comportamiento, etc.

        \item \emph{Insectos:} Nuestro universo de estudio obliga solo a tener en cuenta 4 insectos, los cuales sirven para indicar el posible grado de contaminación del agua. Estos insectos son los siguientes:
            \begin{itemize}
                \item Elimidos.
                \item Patudos.
                \item Plecopteros.
                \item Tricopteros.
            \end{itemize}

        \item \emph{Muestra}: Una muestra esta compuesta por los insectos encontrados en una salida de campo. 

        \item \emph{Indice de Contaminación}: Para el sistema, el indice de contaminación es el valor calculado, mediante la cantidad de diferentes insectos encontrados en el universo de estudio, de la siguiente manera:
        \begin{table}[H]
            \centering
            \begin{tabular}{|p{3.8cm}|l|l|}
                \hline
                \centering
                Cantidad de insectos encontrados  & Indice & Interpretación \\ \hline 
                            0                     & 0 & Muy contaminado \\ \hline
                            1                     & 1 & Contaminado \\ \hline
                            2                     & 2 & Con contaminacion media \\ \hline
                            3                     & 3 & En buen estado \\ \hline
                            4                     & 4 & En excelente estado \\ 
                \hline
            \end{tabular}
        \end{table}

        

        \item \emph{Foto paisaje}: Foto obtenida del paisaje en donde se realizo la muestra de los insectos encontrados, con el fin de facilitar un punto de referencia visual para proximas salidas de campo.
        
        \item \emph{Foto insectos}: Foto obtenida de la muestra que sirve para que los administradores de la aplicacion validen, o no, el registro en cuestion.

        \item \emph{Coordenadas geograficas}: Se usan para referenciar, mediante latitud y longitud, el lugar en donde se realizo el registro.

        \item \emph{Mapa}: Mapa digital (Google Maps) en donde se muestran los registros realizados por los usuarios.

        \item \emph{HTML:} HyperText Markup Language. Lenguaje de marcado para la elaboracion de paginas web. En nuestro caso, genera la vista final del sistema.
        
        \item \emph{HTTPS:} Hypertext Transfer Protocol Secure.

        \item \emph{BD:} Base de datos.
 
        \item \emph{CRUD:} Es el acrónimo de "Crear, Leer, Actualizar y Borrar" (del original en ingles: Create, Read, Update and Delete),que se usa para referirse a las funciones básicas en bases de datos.

        \item \emph{MCVS:} Modelo de Ciclo de Vida del Sistema.
        
         \end{itemize}
    

 \section{Descripción general}

    Esta sección nos presenta una descripción general del sistema con el fin de conocer las funciones que debe soportar, los datos asociados, las restricciones impuestas y cualquier otro factor que pueda influir en la construcción del mismo.
    El sistema nace como necesidad de un grupo de docentes de la Facultad de Ciencias Naturales e Instituto Miguel Lillo, por llevar un control mas exhaustivo de su investigación.
    Es prioritario el seguimiento de la contaminación de los afluentes y ríos ubicados en la provincia de Tucumán.
    
    \subsection{Proceso actual}

    Este seguimiento se realizaba mediante salidas de campos con alumnos de las escuelas rurales donde realizaban las siguientes actividades.
    \begin{itemize}
        \item Los alumnos obtienen muestras del río o afluente intentando capturar algunos de los insectos del universo de estudio.
        \item Los docentes a cargo verifican dichas muestras, e identifican las coincidencias, obteniendo de esta forma, un indice de contaminación.  
    \end{itemize}
    Todo esto se realiza de manera manual, anotando en papel y luego es transcripto a una planilla Excel.
    Todo el procedimiento antes descripto, dificulta la trazabilidad y administración de la información (introduciendo errores y demoras por manejo manual de la información), por lo que todo esto sería más efectivo y sencillo con la ayuda de la tecnología.

    \subsection{Reingeniería de proceso}

    El objetivo de este proyecto se basa en proporcionar facilidades al proceso actual de la siguiente forma: 

    \begin{itemize}
            \item Dos imágenes capturadas con la cámara de fotos del Smartphone:
            \begin{itemize}
                \item La primera imagen será una foto de los insectos encontrados en un río o afluente. El objetivo es encontrar 4 insectos diferentes para analizar la biodiversidad.
                \item La segunda imagen sera una foto del paisaje que servirá como un futuro punto de referencia para próximas salidas de campo. 
            \end{itemize}
            \item Capturar coordenadas de manera automática con una precisión propia al GPS integrado del Smartphone, las cuales se guardaran como \emph{latitud} y \emph{longitud}
            \item El usuario deberá seleccionar, según su criterio personal, cuales de los 4 insectos fueron encontrados por él, mediante un formulario interactivo, el cual consta de imágenes reales de los mismos para una buena comparación y un campo de observaciones para realizar comentarios subjetivos sobre la muestra en cuestión.
            \item Al completar toda la información mencionada anteriormente, se mostrara una ruleta animada virtual, la cual mostrará un valor (indice de contaminación), indicando el posible grado contaminación del agua en donde se realizo la muestra.
        \item Lo anterior se realizará sin conexión a internet (2G, 3G, Wifi, etc), generando un registro de manera local, que luego, de manera automática, se subirá a los servidores al momento de adquirir alguna conexión a internet.
        \item Los administradores podrán gestionar, mediante un navegador web (Google Chrome, Internet Explorer, Mozilla, etc) o desde la aplicacion en el Smartphone, los registros previamente creados y guardados en el servidor. 
        \item Analizando todos los registros, se creará un renderizado de un mapa  (Google Maps) en donde los administradores y usuarios podrán visualizar, con trazos de diferentes colores, el grado de contaminación del agua en el curso de los riós o afluentes analizados en las salidas de campo.
    \end{itemize}
 

El sistema debe ser seguro, escalable, de fácil mantenimiento y muy simple de usar utilizando sólo interfaces táctiles para los usuarios finales. El futuro sistema llevará el nombre \emph{Agüita}.

Las funciones que debe realizar el sistema se pueden agrupar de la siguiente manera:

  \begin{itemize}
  
  \item \emph{Administración de usuarios:} Debe permitir gestionar los usuarios (CRUD). Los mismos pueden ser usuarios o administradores del sistema.
    
    Para que un usuario pueda generar un registro, deberá estar previamente registrado e iniciar sesión por una única vez en su Smartphone.
    
    Los usuarios deben poder configurar/modificar las opciones de su perfil como ser: nombre, apellido, lugar de residencia, institución a la que pertenece y grado correspondiente.

    Los usuarios no pueden consultar la información personal de otros usuarios
  
  \item \emph{Administración de registros:} Debe permitir gestionar los registros (CRUD). Los usuarios generar registros. Los administradores pueden interactuar con los mismo cambiando su estado (valido - invalido) según la información brindada por los mismos (correcta o no).
  
  \item \emph{Consultas de registros realizados:} Los usuarios podrán consultar un listado de sus registros creados con la información completa.
  
  Los usuarios no podrán consultar los registros de otros usuarios.

  Los administradores podrán ver los registros creados por los todos los usuarios de la aplicación con toda su información correspondiente.

  \item \emph{Consulta de mapa:} Los usuarios y administradores pueden ver el mapa final con toda la información recopilada de todos los registros creados por los mismos.

  Todos los registros que estén en un estado \emph{rechazado} no se eliminaran de la base de datos, no obstante, los mismos no se tomaran en cuenta para el renderizado del mapa final. 
  
  \end{itemize}

 \section{Requisitos específicos}
        \subsubsection{Requisitos Funcionales}
        
        \begin{enumerate}[A.]

        \item \textbf{Gestión de Usuarios}
        
        \paragraph{Introducción:} El sistema permite introducir información sobre usuarios (Nombre y apellido, email, residencia, institución), modificar los ya existentes o borrarlos.
        
        \paragraph{Entrada:} @NombreUsuario + Nombre + Apellido + Institución + Grado + Residencia + Email
        
        \paragraph{Proceso:} Comprobar si se trata de un usuario nuevo, dándolo de alta o actualizarlo si ya existe.
        Se muestra en pantalla la ficha de datos del usuario. No pueden existir dos usuarios con el mismo nombre de usuario y/o Email. 
        
        \paragraph{Salida:} Datos de usuarios actualizados y mensajes de lo que está ocurriendo.
        
        \item \textbf{Gestión de Registros}
        
        \paragraph{Introducción:} El sistema debe permitir la  creación de registros (Indice, fecha, coordenadas, fotos capturadas, etc). 
        
        \paragraph{Entrada:} @idRegistro + Indice + FechaCreacion + Latitud + Longitud + FotoPaisaje + FotoInsectos + FotoMapa + Observaciones + Usuario + Ubicación 
        
        \paragraph{Proceso:} Cuando se crea un registro, se codifican todas las fotos, se recopila toda la información y se lo guarda de manera local hasta obtener una conexión a internet para la sincronización con el servidor online.

        No pueden existir 2 o mas registros con el mismo idRegistro
        
        \paragraph{Salida:} Datos de los registros actualizados y mensajes de lo que está ocurriendo.
        
        \item \textbf{Búsqueda de usuarios}
        
        \paragraph{Introducción:}El sistema debe permitir la búsqueda de usuarios por su nombre de usuario o nombre y apellido o institución. 
        
        \paragraph{Entrada:} Tipo de Búsqueda (nombre de usuario o nombre y apellido o institución) + CadenaBúsqueda
        
        \paragraph{Proceso:} Buscar las correspondencias de usuarios usando el tipo y la cadena de búsqueda. Mostrar los resultados, aunque no se haya encontrado nada.
        
        \paragraph{Salida:} Usuario \{Nombre de usuario + Nombre + Apellido + Institución + Grado + Residencia + Email + Cantidad de Registros realizados\}
        
         \item \textbf{Busqueda de registros}
        
        \paragraph{Introducción:} El sistema debe permitir la búsqueda de registros por su fecha de creación, indice o institución.
        \paragraph{Entrada:} Tipo de Busqueda (fecha de creación o indice o institución) + CadenaBusqueda
        
        \paragraph{Proceso:} Buscar las correspondencias de registros usando el tipo y la cadena de búsqueda. Mostrar los resultados, aunque no se haya encontrado nada.
        
        \paragraph{Salida:} Registro \{idRegistro + Indice + FechaCreacion + Latitud + Longitud + FotoPaisaje + FotoInsectos + FotoMapa + Observaciones + Usuario + Ubicación \}
        
        \item \textbf{Listado de Usuarios}
        
        \paragraph{Introducción:} Permite listar todos los Usuarios registrados.
        
        \paragraph{Entrada:}
        
        \paragraph{Proceso:} Listar los usuarios registrados que estén dados de alta en el sistema
        
        \paragraph{Salida:} Usuarios \{Nombre de usuario + Nombre + Apellido + Institución + Grado + Residencia + Email + Registros realizados\}

        \item \textbf{Listado de Registros}
        
        \paragraph{Introducción:} Permite listar todos los registros credos por todos los usuarios.
        
        \paragraph{Entrada:} 
        
        \paragraph{Proceso:} Listar los registros realizados
        
        \paragraph{Salida:} Registros \{idRegistro + Indice + FechaCreacion + Latitud + Longitud + FotoPaisaje + FotoInsectos + FotoMapa + Observaciones + Usuario + Ubicación\}

        \item \textbf{Listar Mapa}
        
        \paragraph{Introducción:} Permite ver el mapa interactivo con la información de los registros creados por los usuarios.
        
        \paragraph{Entrada:} 
        
        \paragraph{Proceso:} Listar Mapa con puntos obtenidos mediante la latitud y longitud de los registros almacenados, indicando el indice de contaminación en dichos puntos
        
        \paragraph{Salida:} Mapa \{Latitud + Longitud + Indice\}

        \end{enumerate}
         
        \subsubsection{Suposiciones y Dependencias:}
        
            \alternativo{Suposiciones:} 
            
            Se asume que los requisitos en este documento son estables una vez que sean aprobados por los responsables de la aplicación. Cualquier petición de cambios en la especificación debe ser aprobada por todas las partes intervinientes y será gestionada por el equipo de desarrollo.
            
            \alternativo{Dependencias:} 
            
            No existen dependencias.
            
          \subsubsection{Requisitos de Usuario y Tecnológicos:}
        
            \alternativo{Requisitos de usuario:} 
            
            Los usuarios serán los que descarguen la aplicación desde Google Play, App Store o Tienda de aplicaciones de Windows mediante su Smartphone.
            
            \alternativo{Requisitos tecnológicos:} 
            
            En vista de que es necesario instalar la aplicacion en Smartphones y ademas en una PC convencional (PC de escritorio, notebook o similares), se opto por un entorno económico y fácil de instalar. La aplicacion se ejecutara sobre un esquema cliente/servidor, con los procesos e interfaz de usuario ejecutándose en los Smartphones y éstos solicitando requerimientos al servidor que cumple su proceso.
            
        \subsubsection{Requisitos de Interfaces Externas}
        
            \alternativo{Interfaces de usuario:} 
            
            Las interfaces de la aplicación deben ser intuitivas, fáciles de usar, amigables y de respuesta rápida. La interfaz de usuario debe ser orientada al uso táctil.
            
            \alternativo{Interfaces Hardware:} 
            
            \begin{itemize}
                \item Requisitos para los Smartphones:
                \begin{itemize}
                    \item Los Smartphones de los usuarios que ejecutaran la aplicacion deberán tener las siguientes características independientemente de su S.O:
                    \begin{itemize}
                        \item Cámara fotográfica de 1 Mega Pixeles.
                        \item GPS integrado.
                        \item Conexión a internet
                        \item Pantalla táctil de 3.5" o superior.
                        \item Espacio disponible de 5 MB para la instalación de la aplicacion + Cantidad de MB variable ocupado por cada registro creado. 
                    \end{itemize}
                \end{itemize}
                \item Requisitos mínimos para el servidor:
                \begin{itemize}
                    \item Procesador AMD Sempron 3000 o equivalente.  Procesador Intel Celeron o equivalente.
                    \item 1 GB de memoria RAM.
                    \item Conexión a internet.
                \end{itemize}
                \item Requisitos mínimos para las PC o notebook de los administradores
                \begin{itemize}
                    \item Procesador AMD Sempron 3000 o equivalente. Procesador Intel Celeron o equivalente.
                    \item 1 GB de memoria RAM.
                    \item Periféricos de entrada/salida.
                    \item Conexión a internet
                \end{itemize}
            \end{itemize}
            
            \alternativo{Interfaces Software:} 
            
            S.O soportados por la aplicación:
            \begin{itemize}
                \item Smartphones 
                \begin{itemize}
                    \item Android 5.0 o posterior.
                    \item iOS 9.0 o posterior.
                    \item Windows Mobile 5.0 o posterior. 
                \end{itemize}
                \item Servidor 
                \begin{itemize}
                    \item  El sistema operativo sera Ubuntu Server 14.04 LTS. 
                \end{itemize}
                \item PC o notebook de los administradores 
                \begin{itemize}
                    \item Sistema Operativo Ubuntu 14.04 LTS o Windows 7 en adelante. 
                \end{itemize} 
            \end{itemize}
            
        \subsubsection{Requisitos de Rendimiento:}
        
        El Tiempo de respuesta de la aplicación de cada función solicitada por el usuario no debe ser superior a los 3 segundos en una velocidad efectiva de conexión con el servidor a través de 3G.
        
        \subsubsection{Requisitos de Desarrollo:}
        
        El ciclo de vida será Prototipado Evolutivo, debiendo orientarse hacia el desarrollo de un sistema flexible que permita incorporar de manera sencilla cambios y nuevas funcionalidades.
        
       \subsubsection{Restricciones de Diseño:}
       
        \alternativo{Ajuste a estándares:} Interfaz de usuario basada Material Desing (Google)
        
        \alternativo{Seguridad:} La seguridad de los datos y de sesion será establecida por la utilización de JSON Web Tokens incorporado al uso de API RESTful ademas de la seguridad brindada por el Sistema Gestor de Base de Datos Relacional y No Relacional.

        \alternativo{Política de respaldo:} 
        
        El administrador llevará a cabo un respaldo de datos en discos externos o en la nube por el tiempo que el considere necesario. El motor de Base de datos estará configurado para realizar backups una vez al mes. Conservar los
        respectivos archivos de respaldo de los últimos 6 backups.
        
        \alternativo{Base de Datos:}
        
        El Sistema Gestor de Base de Datos debe ser relacional en el servidor y se accederán a los mismos usando la tecnología innoDB.
        En los Smartphone el Sistema Gestor de Base de Datos es No Relacional dado el requisito de creacion de registros de manera local y posterior sincronizacion automática al detectar una conexion de internet. 
        
        \alternativo{Política de Borrado:}
        
        No se ha definido

\section{Estimación del proyecto}