%%%%%%%%%%%%%%%%%%%%%%%%%%%%%%%%%%%%%%%%%%%%%%%%%%%%%%%%
%   |------------------------------------------|       %
%   | Web App embebida en dispositivos móviles |       %
%   |  para la gestión de registros sobre la   |       %
%   |   contaminación de afluentes y ríos.     |       %
%   |                                          |       %
%   |          Proyecto de graduación          |       %
%   |__________________________________________|       %
%                                                      %
%   Autores                                            %
%   -------                                            %
%                                                      %
% * Bruno, Ricardo Hugo (CX 1409686)                   %
%     rburnount@gmail.com                              %
% * Gomez Veliz, Kevin Shionen (CX 1411828)            %
%     ing.gomezvelizkevin@gmail.com                    %
%                                                      %
%   Tutor                                              %
%   -------                                            %
%                                                      %
% * Ing. Cohen, Daniel Eduardo                         %
%        dcohen.tuc@gmail.com                          %
%                                                      %
%   Cotutor                                            %
%   -------                                            %
%                                                      %
% * Ing. Nieto, Luis Eduardo                           %
%        lnieto@herrera.unt.edu.ar                     %
%                                                      %
%                                                      %
%%%%%%%%%%%%%%%%%%%%%%%%%%%%%%%%%%%%%%%%%%%%%%%%%%%%%%%%

\chapter{Disciplina de Requisitos}
\label{chap:requisito}

\section{Introducción}

Esta especificación tiene como objetivo analizar y documentar las necesidades funcionales que deberán ser soportadas por el sistema a desarrollar. Para ello, se identificarán los requisitos que ha de satisfacer el nuevo sistema mediante entrevistas, el estudio de los problemas de las unidades afectadas y sus necesidades actuales. Además de identificar los requisitos se deberán establecer las prioridades, los cual proporciona un punto de referencia para validar el sistema final que compruebe que se ajusta a las necesidades del usuario.

\section{Identificación de usuarios participantes}

Los objetivos de esta tarea son identificar a los responsables de cada una de las unidades y a los principales usuarios implicados. Para ello se consideran los siguientes aspectos:

\begin{itemize}
    \item Incorporación de usuarios al equipo de proyecto.
    \item Conocimiento de los usuarios de las funciones a automatizar.
    \item Repercusión del nuevo sistema sobre las actividades actuales de los usuarios.
    \item Implicaciones legales del nuevo sistema.
 
\end{itemize}

Se identificaron los siguientes usuarios:

 \begin{itemize}
 
\item \emph{Grupo de Administradores:} Formado por los solicitantes del software en cuestión.
 
\item \emph{Grupo de Usuarios:} Formado principalmente por alumnos de escuelas/colegios que realizan muestras, las cuales generan registros en el sistema.

 \end{itemize}
 
 Es de destacar la necesidad de una participación activa de los usuarios del futuro sistema en las actividades de desarrollo del mismo, con objeto de conseguir la máxima adecuación del sistema a sus necesidades y facilitar el conocimiento paulatino de dicho sistema, permitiendo una rápida implantación.

  \section{Educción y extracción de requisitos.}

  \subsection{Planificación y Realización de Entrevistas. Estudio de Documentación}
  
  Esta tarea tiene como finalidad capturar los requisitos de usuarios para el desarrollo del sistema.

Para el análisis de requisitos se usaron distintas técnicas de educción de requisitos. Entre ellas el estudio de la documentación provista por parte del cliente; entrevistas abiertas y estructuradas, análisis del proceso actual.
  
  \subsection{Entrevistas. Planificación y Realización}
  
  La entrevista consiste en una interacción sistemática con un usuario para educir los conocimientos de éste. Se hicieron entrevistas abiertas y estructuradas. 

El proceso consiste en la  planificación de las entrevistas, en donde se incluye fecha, hora, lugar, duración estimada y guión de la entrevista; luego se llevan a cabo las entrevistas y se las documenta identificando los requisitos con sus prioridades.  

A partir de las entrevistas realizadas con el cliente, se identifican los requisitos que debe cumplir el sistema y se establecerá una prioridad para los mismos, de acuerdo a las necesidades de los usuarios y a los objetivos a cubrir por el sistema completo.

  \subsection{Estudio de la Documentación}
  
El estudio de la documentación consiste en la extracción de requisitos de los documentos e impresiones que forman parte del proceso actual.

Una vez que se obtiene toda la documentación relevante al sistema que se va a desarrollar, se lee cuidadosamente la información contenida en esos documentos y luego con la técnica de análisis estructural de textos, se extraen conceptos fundamentales del dominio buscando estructuras preestablecidas.

\section{Estimación del proyecto}

 \subsection{Puntos de Función y COCOMO}
 
      \subsubsection{Introducción}
      
      Como alternativa se propuso en este trabajo la utilización combinada de dos métodos
(Puntos de Función y COCOMO) tendientes a proporcionar una estimación más precisa.
Primero se utilizará el  método de Puntos de Función para determinar las líneas de código del proyecto software, la cual mantiene una distorsión.

En segundo lugar se aplicará el método COCOMO partiendo de la información producida por el anterior (líneas de código) para llegar a una estimación precisa de las horas hombre a aplicar y fundamentalmente a la estimación de la duración total del proyecto, dado por sus características intrínsecas independientemente de los recursos a emplear.

      \subsubsection{Aplicación del Método}
      
      Este método calcula los puntos de función de un sistema descomponiendo al mismo en cinco funciones principales (entradas, salidas, consultas, ficheros internos y externos), asignándoles valores de acuerdo a su complejidad y en función de la cantidad de cada uno de ellos se llega a determinar, mediante su sumatoria, los puntos de función.
A continuación se detallan los parámetros a considerar en el cálculo del método:

 \begin{itemize}
 
\item \emph{Datos lógicos internos (ILF):} Corresponde al almacén de datos identificados por un nombre en un diagrama de flujo de datos. Un ejemplo de este parámetro serían los archivos maestros.
 
\item \emph{Datos de Interfase externos (EIF):} Similar a los ILF, pero estos almacenes de datos son mantenidos por otra aplicación. Por ejemplo un sitio Web  que muestra datos fuera de la empresa mantenido por otro sistema interno a la empresa.
 
\item \emph{Entrada externas (EI):} Son datos o información de control que se introducen en la aplicación desde fuera de sus límites. Estos datos mantienen los datos lógicos internos. Un ejemplo de este parámetro son las pantallas de entradas en las que aparecen botones como agregar, modificar y quitar.
 
\item \emph{Salidas externas (EO):} Son datos o información de control que salen de los límites de la aplicación. Esta salida debe ser considerada única si tiene un formato único o si el diseño requiere un proceso lógico distinto.  Un ejemplo de este parámetro son los informes, cada informe producido se cuenta como una salida.
 
\item \emph{Consultas externas (EQ):} Representan los requisitos de información a la aplicación, no actualiza ILF y no contienen datos derivados. Por ejemplo se tienen las búsquedas inmediatas de datos.
 \end{itemize}

      \subsubsection{Descripción de la Técnica}
      
      Para el cálculo de la complejidad, se utilizaron los siguientes parámetros:
     
\begin{itemize}
 
    \item \emph{Data Element Type (DET):} Campo único identificable por el usuario. Hace referencia a las claves únicas de los almacenes de datos.

\item \emph{Record Element Type (RET):} Subgrupo de datos identificables por el usuario.

\item \emph{File Type Referenced (FTR):} ILF leído o mantenido.

\end{itemize}
      \subsubsection{Definición de parámetros básicos externos del sistema}

Para el cálculo de la complejidad en un ILF se accede a una tabla de doble entrada de acuerdo a la cantidad de DET y RET que presenta.
      
      \subsubsection{Ejecución del método en el proyecto actual}

\begin{itemize}
    
    \item Se identificaron 4 ILF.
    
    \begin{table}[H]
        \centering
        \begin{tabular}{|l|l|l|l|l|}
            \hline
            ILF         & RET & DET & Complejidad & Valor \\ \hline
            Insectos    & 1   & 1   & Baja        & 7     \\ \hline
            Registros   & 1   & 12  & Baja        & 7     \\ \hline
            Ubicaciones & 1   & 3   & Baja        & 7     \\ \hline
            Usuarios    & 1   & 12  & Baja        & 7     \\ 
            \hline
        \end{tabular}
    \end{table}
    
    Baja: $4 * 7 = 28$ -
    Media: 0 -
    Alta: 0

    Total = 28
    
    \item Se identificaron 3 ELF.
    
    \begin{table}[H]
        \centering
        \begin{tabular}{|l|l|l|l|l|}
            \hline
            ELF                             & RET & DET & Complejidad & Valor \\ \hline
            Google Maps                     & 1   & 2   & Baja        & 5     \\ \hline
            Google Geocodificación inversa  & 1   & 2   & Baja        & 5     \\ \hline
            Google Statics Maps             & 1   & 2   & Baja        & 5     \\
            \hline
        \end{tabular}
    \end{table}

    Baja: $3 * 5 = 15$
    Media: 0 -
    Alta: 0
    
    Total = 15

    \item Se identificaron  EI.
    
    \begin{table}[H]
        \centering
        \begin{tabular}{|l|l|l|l|l|}
            \hline
            EI                 & FTR & DET & Complejidad & Valor \\ \hline
            Nuevo Registro     & 2   & 16  & Alta        & 6     \\ \hline
            Eliminar Registro  & 1   & 1   & Baja        & 3     \\ \hline
            Nuevo Usuario      & 1   & 9   & Baja        & 3     \\ \hline
            Modificar Usuario  & 1   & 6   & Baja        & 3     \\ \hline
            Eliminar Usuario   & 1   & 1   & Baja        & 3     \\ \hline
        \end{tabular}
    \end{table}
    
    Baja: $4 * 3 = 12$ -
    Media: 0 -
    Alta: $1 * 6 = 6$

    Total = 18

    \item Se identificaron x EO. // PROCESAMIENTO ADICIONAL
    
    \begin{table}[H]
        \centering
        \begin{tabular}{|l|l|l|l|l|}
            \hline
            EO                      & FTR & DET  & Complejidad & Valor \\ \hline
            Inicio de sesión        & 1   & 4    & Baja        & 4     \\ \hline
            Filtro de Usuarios      & 1   & 14   & Baja        & 4     \\ \hline
            Filtro de Registros     & 1   & 16   & Baja        & 4     \\ \hline
            Renderizado de Mapa     & 1   & 3    & Baja        & 4     \\
            \hline
        \end{tabular}
    \end{table}
    
    Baja: $4 * 4 = 16$ -
    Media: 0 -
    Alta: 0 

    Total = 16
    
    \item Se identificaron x EQ. SIN PROCESAMIENTO

    \begin{table}[H]
        \centering
        \begin{tabular}{|l|l|l|l|l|}
            \hline
            EQ                      & FTR & DET & Complejidad & Valor \\ \hline
            Listado de Usuarios     & 1   & 13  & Baja        & 3     \\ \hline
            Listado de Registros    & 1   & 15  & Baja        & 3     \\
            \hline
        \end{tabular}
    \end{table}
    
    Baja: $2 * 3 = 6$ -
    Media: 0 -
    Alta: 0 

    Total = 6
    
\end{itemize}

    \subsubsection{Resumen}
    
    \begin{table}[H]
        \centering
        \begin{tabular}{|l|l|l|l|l|}
            \hline
                    Parámetro & Complejidad & Cantidad & Peso & Total   \\ \hline
                    ILF       & Baja        & 4        & 7    & 28      \\ \hline
                    ELF       & Baja        & 3        & 5    & 15      \\ \hline
    \multirow{3}{*}{EI}       & Baja        & 4        & 3    & 12      \\ \cline{2-5}
                              & Alta        & 1        & 6    & 6       \\ \cline{2-5}
                              &             &          & TOTAL    & 18  \\ \hline
                    EO        & Baja        & 4        & 4    & 16      \\ \hline
                    EQ        & Baja        & 2        & 3    & 6       \\ \hline
    \end{tabular}
\end{table}
    
    $PF = ILF + ELF + EI + EO + EQ$
    
    $PF = 28 + 15 + 18 + 16 + 6 = 83$
    
    $PF = 83$
    
    \subsubsection{Ajuste de los Puntos de Función}
        
   \begin{table}[H]
   \centering
    \begin{tabular}{|p{5cm}|p{10cm}|l|}
        \hline
        Característica & Descripción del grado de influencia & Valor \\ \hline
        Comunicación de datos & Mas de una pc cliente y aplicación para dispositivos móviles, pero la aplicación soporta un solo tipo de protocolo de comunicaciones & 4 \\ \hline
        Funciones distribuidas & El proceso es distribuido y la transferencia de datos son en ambas direcciones & 4 \\ \hline
        Rendimiento & Herramientas y desarrollo especifico para los requisitos impuestos por el cliente & 5 \\ \hline
        Configuración de explotación usada por otros sistemas & No se indican restricciones & 0 \\ \hline
        Tasa de transacciones & No se prevén picos & 0 \\ \hline
        Entrada de datos EN-LÍNEA & Más del 30\% de las transacciones son entradas de datos interactivas & 5 \\ \hline
        Diseño para la eficiencia del usuario final & Más de 6 funciones que incrementan la eficiencia del usuario final, y se establecieron requisitos de eficiencia del usuario que obligan a diseñar tareas que tienen en cuenta factores humanos & 4 \\ \hline
        Actualización on-line & Actualización de 4 o más ficheros. El volumen de actualización es bajo y la recuperación fácil & 2 \\ \hline
        Lógica de proceso interno compleja & Se requiere cumplir con 3 de las 5 características & 3 \\ \hline
        Reutilización del código & Se reutiliza código dentro de la misma aplicación & 1 \\ \hline
        Facilidad de instalación & No se realizaron consideraciones ni se requirieron desarrollos especiales para la instalación por parte del usuario & 0 \\ \hline
    \end{tabular}
\end{table}
\begin{table}[H]
    \centering
     \begin{tabular}{|p{5cm}|p{10cm}|l|}
         \hline
        Facilidad de operación & No se definieron por parte del usuario necesidades especiales de operación o respaldo distintas de las normales & 0 \\ \hline
        Instalación en distintos lugares & Se necesita diseñar la aplicación para ser usada en múltiples lugares y funcionará  en múltiples entornos de  Hardware y Software & 3 \\ \hline
        Facilidad de cambio & No existe ninguna especificación por parte de los usuarios en este sentido & 0 \\ \hline
    \end{tabular}
\end{table}

Grado de Influencia: $TDI = 31$

Factor de ajuste: VAF = $ (TDI*0.01) + 0.65 = 0.96$


\paragraph{Puntos de función ajustados:} 

PFA = $PF * VAF = 83 * 0.96 = 79.68$

\subsubsection{Estimación de las líneas de código necesarias}

QSM SLOC/FP Data JavaScript: 54 LOC por PF


\paragraph{LOC=} $54*79.68 = 4302.72$
        
\subsubsection{Aplicación del método COCOMO}

    La estimación del tiempo de desarrollo y la cantidad necesaria de personas partipantes se realizó utilizando el método intermedio del modelo orgánico.

\begin{table}[H]
\centering
    \begin{tabular}{|l|l|}
        \hline
        RELY & 1    \\ \hline
        DATA & 1.08 \\ \hline
        CPLX & 0.7  \\ \hline
        TIME & 1    \\ \hline
        STOR & 1    \\ \hline
        VIRT & 0.87 \\ \hline
        TURN & 0.87 \\ \hline
        ACAP & 1.19 \\ \hline
        AEXP & 1.13 \\ \hline
        PCAP & 1    \\ \hline
        VEXP & 1.1  \\ \hline
        LEXP & 1.07 \\ \hline
        MODP & 1    \\ \hline
        TOOL & 0.83 \\ \hline
        SCED & 1.04 \\
        \hline
    \end{tabular}
\end{table}
    
    
Factor de Ajuste: A = $0.782$
\\

\begin{framed}
\centering
% KLOC son las lineas de codigo en MILES, osea es el LOC calculado dividido 1000
MM = $A*3.2*(KLOC)^{1.05} = 11.58$ meses hombre

TDEV = $2,5 (MM)^{0.38} = 6.34$ meses

Número de personas = ${11.58}/{6.34} = 1.82 \approx 2$ personas
\end{framed}

\section{Especificación de Requisitos de Software}
\setcounter{secnumdepth}{2}
\renewcommand{\thesubsection}{\arabic{subsection}}
 \subsection{Introducción}
 
 Este documento es una Especificación de Requisitos Software de la  Web App embebida en dispositivos móviles para la gestión de registros sobre la contaminación de afluentes y ríos. Esta documentación es fruto de las entrevistas, estudio de la documentación y del funcionamiento de la aplicación web actual, así como del análisis llevado a cabo por el equipo de desarrollo.

El objetivo de la especificación es definir en forma clara, precisa, completa y verificable todas las funcionalidades y restricciones del sistema que se desea construir.

Esta documentación está sujeta a revisiones por el grupo de administradores que se recogerán por medio de sucesivas versiones del documento, hasta alcanzar la aprobación por parte de los mismos. Una vez aprobado, servirá de base al equipo de desarrollo para la construcción del sistema en cuestión.

Esta especificación se ha realizado de acuerdo al estándar “IEEE Recomended Practice for software Requirements Specifications(IEEE/ANSI 830-1993)”.

 
    \subsubsection{Objetivos y alcance del sistema}
% Esta subsección debería
% a) Identificar el producto de software a ser producido por su nombre (ej: Report Generator, Host DBMS)
% b) Explicar lo que el producto software hará y, si es necesario, lo que no hará.
% c) Describir la aplicación del software que está siendo especificado, incluyendo beneficios relevantes, objetivos, y metas
% En esta etapa se detallan los objetivos del sistema, describiendo brevemente QUÉ es lo que el sistema debe hacer. En el alcance del sistema %  se describe en lenguaje natural el ámbito del sistema, su dominio y sus límites.

Los principales objetivos del sistema a desarrollar son permitir que los usuarios puedan hacer uso del mismo mediante un SmartPhone con cualquier Sistema Operativo (iOS, Android, Windows Phone), pudiendo asi, realizar los siguientes items:
\begin{itemize}
    \item En las salidas de campo, los alumnos podrán generar registros con la siguiente información:
    \begin{itemize}
        \item Dos imágenes capturadas con la cámara de fotos del SmartPhone:
        \begin{itemize}
            \item La primera imagen sera una foto de los insectos encontrados en un rio o afluente. El objetivo es encontrar 4 insectos diferentes para analizar la biodiversidad.
            \item La segunda imagen sera una foto del paisaje que servirá como un futuro punto de referencia para próximas salidas de campo. 
        \end{itemize}
        \item Capturar coordenadas de manera automática con una precisión propia al GPS integrado del SmartPhone, las cuales se guardaran como \emph{latitud} y \emph{longitud}
        \item El usuario deberá seleccionar, según su criterio personal, cuales de los 4 insectos fueron encontrados por él, mediante un formulario interactivo, el cual consta de imágenes reales de los mismos para una buena comparación y un campo de observaciones para realizar comentarios subjetivos sobre la muestra en cuestión.
        \item Al completar toda la información mencionada anteriormente, se mostrara una ruleta animada virtual, la cual mostrará un valor (indice de contaminación), indicando el posible grado contaminación del agua en donde se realizo la muestra.
    \end{itemize}
    \item Lo anterior se realizará sin conexión a internet (2G, 3G, Wifi, etc), generando un registro de manera local, que luego, de manera automática, se subirá a los servidores al momento de adquirir alguna conexión a internet.
    \item Los administradores podrán gestionar, mediante un navegador web (Google Chrome, Internet Explorer, Mozilla, etc) o desde la aplicacion en el SmartPhone, los registros previamente creados y guardados en el servidor. 
    \item Analizando todos los registros, se creará un renderizado de un mapa  (Google Maps) en donde los administradores y usuarios podrán visualizar, con trazos de diferentes colores, el grado de contaminación del agua en el curso de los riós o afluentes analizados en las salidas de campo.
\end{itemize}
 

El sistema debe ser seguro, escalable, de fácil mantenimiento y muy simple de usar utilizando sólo interfaces táctiles para los usuarios finales. El futuro sistema llevará el nombre \emph{Aguita}.

El desarrollo lo llevarán a cabo los alumnos Ricardo Hugo Bruno y Kevin Shionen Gomez Veliz, con la opción a ser responsables del posterior mantenimiento del mismo.

    \subsubsection{Definiciones, Acrónimos y Abreviaturas}
%Esta subsección debería proveer las definiciones de todos los términos, acrónimos, y abreviaturas requeridas para interpretar apropiadamente la ERS. Esta información podría ser provista por referencia a uno o más apéndices en la ERS o por referencia a otros documentos
    \subsubsection{Definiciones:}
  
        \begin{itemize}
 
        \item \emph{Aplicación web:} Una aplicación web es un sitio web en donde el contenido de todas o algunas de sus páginas se determina en el momento mismo de la solicitud, gracias a la ejecución de un intérprete en el servidor que traduce el código fuente en información visible en un navegador Web, de manera dinámica y con acceso en línea a los datos.
 
        \item \emph{Salida de campo:} La principal aportación de la salida de campo es que permite al alumnado adquirir un aprendizaje significativo en el que el principal elemento del proceso de enseñanza- aprendizaje es la construcción de significados. La persona aprende un concepto, un fenómeno, un procedimiento, un comportamiento, etc.

        \item \emph{Insectos:} Nuestro universo de estudio obliga solo a tener en cuenta 4 insectos, los cuales sirven para indicar el posible grado de contaminación del agua. Estos insectos son los siguientes:
            \begin{itemize}
                \item Elimidos.
                \item Patudos.
                \item Plecopteros.
                \item Tricopteros.
            \end{itemize}

        \item \emph{Indice de Contaminación}: Para el sistema, el indice de contaminación es el valor calculado, mediante la cantidad de insectos encontrados en el universo de estudio, de la siguiente manera:
        \begin{table}[H]
            \centering
            \begin{tabular}{|p{3.8cm}|l|}
                \hline
                \centering
                Cantidad de insectos encontrados  & Indice \\ \hline
                            0                     & 0 \\ \hline
                            1                     & 1 \\ \hline
                            2                     & 2 \\ \hline
                            3                     & 3 \\ \hline
                            4                     & 4 \\ 
                            \hline
            \end{tabular}
        \end{table}

        \end{itemize}

        \subsubsection{Acrónimos:}

        \begin{itemize}

        \item \emph{HTML:} HyperText Markup Language.
        
        \item \emph{HTTPS:} Hypertext Transfer Protocol Secure

        \end{itemize}
 
        \subsubsection{Abreviaturas:}

        \begin{itemize}

        \item \emph{IEEE:} Institute of Electrical and Electronics Engineers.
 
        \item \emph{UP:} Unified Process.

        \item \emph{MCVS:} Modelo de Ciclo de Vida del Sistema.

         \end{itemize}
    

 \subsection{Descripción general}

    Esta sección nos presenta una descripción general del sistema con el fin de conocer las funciones que debe soportar, los datos asociados, las restricciones impuestas y cualquier otro factor que pueda influir en la construcción del mismo.

Las funciones que debe realizar el sistema se pueden agrupar de la siguiente manera:

  \begin{itemize}
  
  \item \emph{Administración de usuarios:} Debe permitir gestionar los usuarios teniendo en cuenta de si tratan de un usuario propiamente dicho o administradores de la aplicación.
    
    Para que un usuario pueda generar un registro, deberá estar previamente registrado e iniciar sesión por una única vez en su SmartPhone.
    
    Los usuarios deben poder configurar/modificar las opciones de su perfil como ser: nombre, apellido, lugar de residencia, institución a la que pertenece y grado correspondiente.

    Los usuarios no pueden consultar la información personal de otros usuarios
  
  \item \emph{Administración de registros:} Los administradores pueden interactuar con los registros cambiando su estado (aceptado - rechazado) según la información brindada por los mismos (correcta o no).
  
  \item \emph{Consultas de registros realizados:} Los usuarios podrán consultar un listado de sus registros creados con la información completa.
  
  Los usuarios no podrán consultar los registros de otros usuarios.

  Los administradores podrán ver los registros creados por los todos los usuarios de la aplicación con toda su información correspondiente.

  \item \emph{Consulta de mapa:} Los usuarios y administradores pueden ver el mapa final con toda la información recopilada de todos los registros creados por los mismos.

  Todos los registros que estén en un estado \emph{rechazado} no se eliminaran de la base de datos, no obstante, los mismos no se tomaran en cuenta para el renderizado del mapa final. 
  
  \end{itemize}

 \subsection{Requisitos específicos}
        \subsubsection{Requisitos Funcionales}
        
        \begin{enumerate}[A.]

        \item \textbf{Gestión de Usuarios}
        
        \paragraph{Introducción:} El sistema permite introducir información sobre usuarios (Nombre y apellido, email, residencia, institución), modificar los ya existentes o borrarlos.
        
        \paragraph{Entrada:} @NombreUsuario + Nombre + Apellido + Institución + Grado + Residencia + Email
        
        \paragraph{Proceso:} Comprobar si se trata de un usuario nuevo, dándolo de alta o actualizarlo si ya existe.
        Se muestra en pantalla la ficha de datos de la persona. No pueden existir dos usuarios con el mismo nombre de usuario y y/o Email. 
        
        \paragraph{Salida:}  Datos de usuarios actualizados y mensajes de lo que está ocurriendo.
        
        \item \textbf{Gestión de Registros}
        
        \paragraph{Introducción:}El sistema debe permitir la  creación de registros (Indice, fecha, coordenadas, fotos capturadas, etc). 
        
        \paragraph{Entrada:} @idRegistro + Indice + FechaCreacion + Latitud + Longitud + FotoPaisaje + FotoInsectos + FotoMapa + Observaciones + Usuario + Ubicación 
        
        \paragraph{Proceso:} Cuando se crea un registro, se codifican todas las fotos, se recopila toda la información y se lo guarda de manera local hasta obtener una conexión a internet para la sincronización con el servidor online.

        No pueden existir 2 o mas registros con el mismo idRegistro
        
        \paragraph{Salida:} Datos de lso registros actualizados y mensajes de lo que está ocurriendo.
        
          \item \textbf{Búsqueda de usuarios}
        
        \paragraph{Introducción:}El sistema debe permitir la búsqueda de usuarios por su nombre de usuario o nombre y apellido o institución. 
        
        \paragraph{Entrada:} Tipo de Búsqueda (nombre de usuario o nombre y apellido o institución) + CadenaBúsqueda
        
        \paragraph{Proceso:} Buscar las correspondencias de usuarios usando el tipo y la cadena de búsqueda. Mostrar los resultados, aunque no se haya encontrado nada.
        
        \paragraph{Salida:} Usuario {Nombre de usuario + Nombre + Apellido + Institución + Grado + Residencia + Email + Registros realizados}
        
         \item \textbf{Busqueda de registros}
        
        \paragraph{Introducción:} El sistema debe permitir la búsqueda de registros por su fecha de creación, indice o institución.
        \paragraph{Entrada:} Tipo de Busqueda (fecha de creación o indice o institución) + CadenaBusqueda
        
        \paragraph{Proceso:} Buscar las correspondencias de registros usando el tipo y la cadena de búsqueda. Mostrar los resultados, aunque no se haya encontrado nada.
        
        \paragraph{Salida:} Registro {idRegistro + Indice + FechaCreacion + Latitud + Longitud + FotoPaisaje + FotoInsectos + FotoMapa + Observaciones + Usuario + Ubicación }
        
        \item \textbf{Listado de compras}
        
        \paragraph{Introducción:} Permite listar las compras realizadas de un determinado usuario comprador.
        
        
        \paragraph{Entrada:} 
        
        \paragraph{Proceso:} Listar las compras realizadas
        
        \paragraph{Salida:} {NombreProducto + PrecioProducto + PrecioEnvio + PrecioTotal + Cantidad + FechaCompra + CódigoCompra + UsuarioVendedor}
        
         \item \textbf{Listado de ventas}
        
        \paragraph{Introducción:} Permite listar las ventas realizadas de un determinado usuario vendedor.
        
        \paragraph{Entrada:}
        
        \paragraph{Proceso:} Listar las ventas realizadas
        
        \paragraph{Salida:} {NombreProducto + PrecioProducto + PrecioEnvio + PrecioTotal + Cantidad + FechaVenta + CódigoCompra + UsuarioComprador}
 
         \end{enumerate}
         
        \subsubsection{Suposiciones y Dependencias:}
        
            \alternativo{Suposiciones:} 
            
            Se asume que los requisitos en este documento son estables una vez que sean aprobados por los responsables de la aplicación web. Cualquier petición de cambios en la especificación debe ser aprobada por todas las partes intervinientes y será gestionada por el equipo de desarrollo.
            
            \alternativo{Dependencias:} 
            
            El sistema tiene dependencia con el actual sitio web con el que compartirá los datos.
            
          \subsubsection{Requisitos de Usuario y Tecnológicos:}
        
            \alternativo{Requisitos de usuario:} 
            
            Los usuarios serán los que descarguen la aplicación de la tienda de aplicaciones de Android.
            
            \alternativo{Requisitos tecnológicos:} 
            
            La aplicación actuará como cliente de la aplicación web que ofrecerá Web services para el acceso a los datos.

            El cliente requiere un dispositivo móvil, con pantalla táctil superior a las 3 pulgadas y Sistema Operativo Android, versión mayor o igual a 2.1.
            
        \subsubsection{Requisitos de Interfaces Externas:}
        
            \alternativo{Interfaces de usuario:} 
            
            Las interfaces de la aplicación deben ser intuitivas, fáciles de usar, amigables y de respuesta rápida. La interfaz de usuario debe ser orientada al uso táctil.
            
            \alternativo{Interfaces Hardware:} 
            
            Dispositivo móvil con pantalla táctil de tamaño superior a las 3 pulgadas.
            
            \alternativo{Interfaces Software:} 
            
            Sistema Operativo Android, versión mayor o igual a 2.1.
            
        \subsubsection{Requisitos de Rendimiento:}
        
        El Tiempo de respuesta de la aplicación de cada función solicitada por el usuario no debe ser superior a los  10 segundos en una velocidad efectiva de conexión con el servidor a través de 3G.
        
        \subsubsection{Requisitos de Desarrollo:}
        
        El ciclo de vida será Prototipado Evolutivo, debiendo orientarse hacia el desarrollo de un sistema flexible que permita incorporar de manera sencilla cambios y nuevas funcionalidades.
        
       \subsubsection{Restricciones de Diseño:}
       
        \alternativo{Ajuste a estándares:} No se ha definido
        
        \alternativo{Seguridad:}
        
        La seguridad de los datos será establecida por la utilización de cifrado mediante el uso de HTTPS en la transmisión de datos.
        
Se implementará también ACL para cada objeto y de esa manera evitar accesos o modificaciones que no correspondan.

        \alternativo{Política de respaldo:} 
        
        No se ha definido 
        
        \alternativo{Base de Datos:}
        
        La base de datos que se utilizará estará en el servidor web que funciona actualmente con la aplicación web y accederá a la misma utilizando la tecnología de web services REST.
        
        \alternativo{Política de Borrado:}
        
        No se ha definido
        

\setcounter{secnumdepth}{-1}



