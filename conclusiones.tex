%%%%%%%%%%%%%%%%%%%%%%%%%%%%%%%%%%%%%%%%%%%%%%%%%%%%%%%%
%   |------------------------------------------|       %
%   | Web App embebida en dispositivos móviles |       %
%   |  para la gestión de registros sobre la   |       %
%   |   contaminación de afluentes y ríos.     |       %
%   |                                          |       %
%   |          Proyecto de graduación          |       %
%   |__________________________________________|       %
%                                                      %
%   Autores                                            %
%   -------                                            %
%                                                      %
% * Bruno, Ricardo Hugo (CX 1409686)                   %
%     rburnount@gmail.com                              %
% * Gómez Veliz, Kevin Shionen (CX 1411828)            %
%     ing.gomezvelizkevin@gmail.com                    %
%                                                      %
%   Tutor                                              %
%   -------                                            %
%                                                      %
% * Ing. Cohen, Daniel Eduardo                         %
%        dcohen.tuc@gmail.com                          %
%                                                      %
%   Cotutor                                            %
%   -------                                            %
%                                                      %
% * Ing. Nieto, Luis Eduardo                           %
%        lnieto@herrera.unt.edu.ar                     %
%                                                      %
%                                                      %
%%%%%%%%%%%%%%%%%%%%%%%%%%%%%%%%%%%%%%%%%%%%%%%%%%%%%%%%

\chapter{Conclusiones}

El desarrollo de nuestro proyecto final nos permitió poner en práctica temas que aprendimos en distintas asignaturas durante el transcurso de nuestra carrera, el aprendizaje y la experiencia de implementar nuevas tecnologías, enfrentándonos a problemas reales de diseño e integración que nos forzaron a tomar decisiones a fin de encontrar soluciones eficientes a los mismos. Además, nos dio la posibilidad de aprender a trabajar con herramientas con los cuales no estábamos familiarizados.

Con la puesta en funcionamiento de este proyecto pudimos generar un producto que
da valor agregado a la metodología de trabajo existente de los docentes de la Facultad de Ciencias Naturales e Instituto Miguel Lillo en conjunto con alumnos de escuelas rurales, que no poseen automatismo en dicha metodología, cumpliendo así, con los objetivos planteados inicialmente, obteniendo un producto seguro, escalable, de fácil mantenimiento y simple de usar.

Se logró implementar a través de Web services una comunicación eficiente con la base de datos, con mecanismos de seguridad que impiden la visualización o alteración indebida de los datos sensibles de los usuarios.

Comprendimos la importancia del uso de un desarrollo modular y orientado a objetos para la realización de proyectos flexibles, escalables, más legibles y manejables en aplicaciones actuales.

También podemos decir que este sistema queda abierto a la incorporación de futuras actualizaciones para satisfacer las necesidades de los usuarios cuando los procesos así lo requieran.

\label{chap:conclusiones}
