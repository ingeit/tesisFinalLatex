%%%%%%%%%%%%%%%%%%%%%%%%%%%%%%%%%%%%%%%%%%%%%%%%%%%%%%%%
%   |------------------------------------------|       %
%   |Aplicación de comercio electrónico para   |       %
%   |teléfonos móviles con S.O. Android        |       %
%   |                                          |       %
%   | Proyecto de graduación                   |       %
%   |__________________________________________|       %
%                                                      %
%   Autores                                            %
%   -------                                            %
%                                                      %
% * Soto, Paula Fabiana (CX05-0967-4)                  %
%     paulette255@gmail.com                            %
% * Vallejo, Sergio Daniel (CX05-0392-4)               %
%     vallejosergio@gmail.com                          %
%                                                      %
%   Tutor                                              %
%   -------                                            %
%                                                      %
% * Ing. Augusto Maximiliano Odstrcil                  %
%        modstrcil@gmail.com                           %
%                                                      %
%                                                      %
%%%%%%%%%%%%%%%%%%%%%%%%%%%%%%%%%%%%%%%%%%%%%%%%%%%%%%%%

\chapter{Conclusiones}

El desarrollo de nuestro proyecto final nos permitió poner en práctica temas que aprendimos en distintas asignaturas durante el transcurso de nuestra carrera, el aprendizaje y la experiencia de implementar nuevas tecnologías, enfrentándonos a problemas reales de diseño e integración que nos forzaron a tomar decisiones a fin de encontrar soluciones eficientes a los mismos. Además, nos dio la posibilidad de aprender a trabajar con herramientas con los cuales no estábamos familiarizados.

Con la puesta en funcionamiento de este proyecto damos valor agregado a la aplicación web existente cumpliendo con los objetivos planteados inicialmente, obteniendo un producto seguro, escalable, de fácil mantenimiento y simple de usar.

Se logró implementar a través de Web services una comunicación eficiente con la base de datos, con mecanismos de seguridad que impiden la visualización o alteración indebida de los datos sensibles de los usuarios.

Vimos la importancia del uso de un desarrollo modular para la realización de proyectos flexibles, escalables, más legibles y manejables en aplicaciones actuales.

También podemos decir que este sistema queda abierto a la incorporación de futuras actualizaciones para satisfacer las necesidades de los usuarios cuando los procesos así lo requieran.

Con respecto a la estimación del tiempo necesario para el desarrollo realizada al inicio del proyecto, cuyo resultado fue de alrededor de 1500 horas hombre, al finalizar el desarrollo del mismo estimamos haber dedicado alrededor de 1300 horas hombre quedando dentro del margen de error del método utilizado.

\label{chap:conclusiones}
