%%%%%%%%%%%%%%%%%%%%%%%%%%%%%%%%%%%%%%%%%%%%%%%%%%%%%%%%
%   |------------------------------------------|       %
%   | Web App embebida en dispositivos móviles |       %
%   |  para la gestión de registros sobre la   |       %
%   |   contaminación de afluentes y ríos.     |       %
%   |                                          |       %
%   |          Proyecto de graduación          |       %
%   |__________________________________________|       %
%                                                      %
%   Autores                                            %
%   -------                                            %
%                                                      %
% * Bruno, Ricardo Hugo (CX 1409686)                   %
%     rburnount@gmail.com                              %
% * Gomez Veliz, Kevin Shionen (CX 1411828)            %
%     ing.gomezvelizkevin@gmail.com                    %
%                                                      %
%   Tutor                                              %
%   -------                                            %
%                                                      %
% * Ing. Cohen, Daniel Eduardo                         %
%        dcohen.tuc@gmail.com                          %
%                                                      %
%   Cotutor                                            %
%   -------                                            %
%                                                      %
% * Ing. Nieto, Luis Eduardo                           %
%        lnieto@herrera.unt.edu.ar                     %
%                                                      %
%                                                      %
%%%%%%%%%%%%%%%%%%%%%%%%%%%%%%%%%%%%%%%%%%%%%%%%%%%%%%%%

% ********* Agradecimientos ********** %
%El * se agrega para que LaTeX no le asigne un número de capítulo
\chapter*{Agradecimientos}
%Se debe agregar esta línea para que este capítulo aparezca en el índice
%\addcontentsline{toc}{chapter}{Agradecimientos}


Agradecemos a nuestras familias que nos apoyaron y ayudaron durante el transcurso de la carrera.

Agradecemos a la Universidad en su conjunto, pública y gratuita, que nos formó académicamente.

Gracias a nuestro tutor Cohen, Daniel Eduardo por brindarnos la posibilidad de desarrollar este proyecto. Ademas agradecemos a nuestro cotutor Nieto, Luis Eduardo por brindarnos ayuda y acompañamiento de igual manera que nuestro tutor.

Por último, queremos agradecer a nuestros compañeros y amigos por los momentos de estudio, logros y festejos compartidos durante la carrera, que nos motivaron para seguir adelante.

\label{chap:agradecimientos}




