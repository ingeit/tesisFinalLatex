%%%%%%%%%%%%%%%%%%%%%%%%%%%%%%%%%%%%%%%%%%%%%%%%%%%%%%%%
%   |------------------------------------------|       %
%   |Aplicación de comercio electrónico para   |       %
%   |teléfonos móviles con S.O. Android        |       %
%   |                                          |       %
%   | Proyecto de graduación                   |       %
%   |__________________________________________|       %
%                                                      %
%   Autores                                            %
%   -------                                            %
%                                                      %
% * Soto, Paula Fabiana (CX05-0967-4)                  %
%     paulette255@gmail.com                            %
% * Vallejo, Sergio Daniel (CX05-0392-4)               %
%     vallejosergio@gmail.com                          %
%                                                      %
%   Tutor                                              %
%   -------                                            %
%                                                      %
% * Ing. Augusto Maximiliano Odstrcil                  %
%        modstrcil@gmail.com                           %
%                                                      %
%                                                      %
%%%%%%%%%%%%%%%%%%%%%%%%%%%%%%%%%%%%%%%%%%%%%%%%%%%%%%%%

% ********* Glosario ********** %

%% Formato general
%\newglossaryentry{<label>}{<key-val list>} 

%% Ejemplo
%\newglossaryentry{set}% the label
%{name=set,            % the term
% description={a collection of objects} % a brief description
%}

\newglossaryentry{SO}
{name=Sistema Operativo,
 description={Un sistemaasdasdad operativo (SO, frecuentemente OS, del inglés Operating System) es un programa o conjunto de programas que en un sistema informático gestiona los recursos de hardware y provee servicios a los programas de aplicación, ejecutándose en modo privilegiado respecto de los restantes.}
 }

\newglossaryentry{Open Source}
{name=Open Source,
 description={Código abierto (o fuente abierta) es el término con el que se conoce al software distribuido y desarrollado libremente. El código abierto tiene un punto de vista más orientado a los beneficios prácticos de poder acceder al código, que a las cuestiones éticas y morales las cuales se destacan en el software libre.}
 }

\newglossaryentry{Linux}
{name=Linux,
 description={Linux es un núcleo libre de sistema operativo basado en Unix. Es uno de los principales ejemplos de software libre. Linux está licenciado bajo la GPL v2 y está desarrollado por colaboradores de todo el mundo.}
 }

\newglossaryentry{Java}
{name=Java,
 description={Java es un lenguaje de programación publicado en el 1995 como un componente fundamental de la plataforma Java de Sun Microsystems. El lenguaje deriva mucho de su sintaxis de C y C++, pero tiene menos facilidades de bajo nivel que cualquiera de ellos. Las aplicaciones de Java son generalmente compiladas a bytecode (clase Java) que puede correr en cualquier máquina virtual Java (JVM) sin importar la arquitectura de la computadora. Java es un lenguaje de programación de propósito general, concurrente, basado en clases, y orientado a objetos, que fue diseñado específicamente para tener tan pocas dependencias de implementación como fuera posible. Su intención es permitir que los desarrolladores de aplicaciones escriban el programa una vez y lo ejecuten en cualquier dispositivo (conocido en inglés como WORA, o "write once, run anywhere"), lo que quiere decir que el código que es ejecutado en una plataforma no tiene que ser recompilado para correr en otra. Java es, a partir del 2012, uno de los lenguajes de programación más populares en uso, particularmente para aplicaciones de cliente-servidor de web, con unos 10 millones de usuarios reportados.}
 }
 
 \newglossaryentry{Python}
{name=Python,
 description={Python es un lenguaje de programación interpretado cuya filosofía hace hincapié en una sintaxis muy limpia y que favorezca un código legible. Se trata de un lenguaje de programación multiparadigma, ya que soporta orientación a objetos, programación imperativa y, en menor medida, programación funcional. Es un lenguaje interpretado, usa tipado dinámico y es multiplataforma.}
 }
 
 \newglossaryentry{HTML}
{name=HTML,
 description={HTML, siglas de HyperText Markup Language («lenguaje de marcado de hipertexto»), hace referencia al lenguaje de marcado predominante para la elaboración de páginas web que se utiliza para describir y traducir la estructura y la información en forma de texto, así como para complementar el texto con objetos tales como imágenes. El HTML se escribe en forma de «etiquetas», rodeadas por corchetes angulares (<,>). HTML también puede describir, hasta un cierto punto, la apariencia de un documento, y puede incluir un script (por ejemplo, JavaScript), el cual puede afectar el comportamiento de navegadores web y otros procesadores de HTML.}
 }
 
 \newglossaryentry{CSS}
{name=CSS,
 description={La idea que se encuentra detrás del desarrollo de CSS es separar la estructura de un documento de su presentación. La información de estilo puede ser adjuntada como un documento separado o en el mismo documento HTML.}
 }
 
 \newglossaryentry{JavaScript}
{name=Javascript,
 description={JavaScript es un lenguaje de programación interpretado,orientado a objetos, basado en prototipos, imperativo, débilmente tipado y dinámico. Se utiliza principalmente en su forma del lado del cliente (client-side), implementado como parte de un navegador web aunque existe una forma de JavaScript del lado del servidor. Su uso en aplicaciones externas a la web, por ejemplo en documentos PDF, aplicaciones de escritorio (mayoritariamente widgets) es también significativo.}
 }
 
  \newglossaryentry{HTTPS}
{name=HTTPS,
 description={Hypertext Transfer Protocol Secure (o Protocolo seguro de transferencia de hipertexto), más conocido por sus siglas HTTPS, es un protocolo de aplicación basado en el protocolo HTTP, destinado a la transferencia segura de datos de Hiper Texto, es decir, es la versión segura de HTTP.}
 }
 
 
 
\newglossaryentry{Subversion}
{name=Subversion,
 description={Sistema de control de versiones para el desarrollo colectivo de software o documentación.}
 }
 
\newglossaryentry{TCP/IP}
{name=TCP/IP (Internet Protocol Suite),
 description={Es un conjunto de protocolos de comunicación que se utiliza en Internet y en otras redes similares. Sus componentes más importantes son TCP e IP, y se encuentran en representados en un modelo de capas, que van desde la capa de enlace hasta la capa de aplicación, pasando por la capa de Internet y la de transporte.}
 }
 
 \newglossaryentry{TCP}
{name=TCP/IP (Internet Protocol Suite),
 description={Es un conjunto de protocolos de comunicación que se utiliza en Internet y en otras redes similares. Sus componentes más importantes son TCP e IP, y se encuentran en representados en un modelo de capas, que van desde la capa de enlace hasta la capa de aplicación, pasando por la capa de Internet y la de transporte.}
 }
 
 \newglossaryentry{framework}
{name=TCP/IP (Internet Protocol Suite),
 description={Es un conjunto de protocolos de comunicación que se utiliza en Internet y en otras redes similares. Sus componentes más importantes son TCP e IP, y se encuentran en representados en un modelo de capas, que van desde la capa de enlace hasta la capa de aplicación, pasando por la capa de Internet y la de transporte.}
 }
 
 \newglossaryentry{XML}
{name=TCP/IP (Internet Protocol Suite),
 description={Es un conjunto de protocolos de comunicación que se utiliza en Internet y en otras redes similares. Sus componentes más importantes son TCP e IP, y se encuentran en representados en un modelo de capas, que van desde la capa de enlace hasta la capa de aplicación, pasando por la capa de Internet y la de transporte.}
 }
 
 \newglossaryentry{JSON}
{name=TCP/IP (Internet Protocol Suite),
 description={Es un conjunto de protocolos de comunicación que se utiliza en Internet y en otras redes similares. Sus componentes más importantes son TCP e IP, y se encuentran en representados en un modelo de capas, que van desde la capa de enlace hasta la capa de aplicación, pasando por la capa de Internet y la de transporte.}
 }
 
 \newglossaryentry{SOAP}
{name=TCP/IP (Internet Protocol Suite),
 description={Es un conjunto de protocolos de comunicación que se utiliza en Internet y en otras redes similares. Sus componentes más importantes son TCP e IP, y se encuentran en representados en un modelo de capas, que van desde la capa de enlace hasta la capa de aplicación, pasando por la capa de Internet y la de transporte.}
 }
 
 \newglossaryentry{URI}
{name=TCP/IP (Internet Protocol Suite),
 description={Es un conjunto de protocolos de comunicación que se utiliza en Internet y en otras redes similares. Sus componentes más importantes son TCP e IP, y se encuentran en representados en un modelo de capas, que van desde la capa de enlace hasta la capa de aplicación, pasando por la capa de Internet y la de transporte.}
 }
 
  \newglossaryentry{Man-in-the-middle}
{name=TCP/IP (Internet Protocol Suite),
 description={Es un conjunto de protocolos de comunicación que se utiliza en Internet y en otras redes similares. Sus componentes más importantes son TCP e IP, y se encuentran en representados en un modelo de capas, que van desde la capa de enlace hasta la capa de aplicación, pasando por la capa de Internet y la de transporte.}
 }
 
 \newglossaryentry{SSL/TLS}
{name=TCP/IP (Internet Protocol Suite),
 description={Es un conjunto de protocolos de comunicación que se utiliza en Internet y en otras redes similares. Sus componentes más importantes son TCP e IP, y se encuentran en representados en un modelo de capas, que van desde la capa de enlace hasta la capa de aplicación, pasando por la capa de Internet y la de transporte.}
 }
 
 \newglossaryentry{ACL}
{name=TCP/IP (Internet Protocol Suite),
 description={Es un conjunto de protocolos de comunicación que se utiliza en Internet y en otras redes similares. Sus componentes más importantes son TCP e IP, y se encuentran en representados en un modelo de capas, que van desde la capa de enlace hasta la capa de aplicación, pasando por la capa de Internet y la de transporte.}
 }
 
 \newglossaryentry{DOM}
{name=TCP/IP (Internet Protocol Suite),
 description={Es un conjunto de protocolos de comunicación que se utiliza en Internet y en otras redes similares. Sus componentes más importantes son TCP e IP, y se encuentran en representados en un modelo de capas, que van desde la capa de enlace hasta la capa de aplicación, pasando por la capa de Internet y la de transporte.}
 }
 
 \newglossaryentry{AJAX}
{name=TCP/IP (Internet Protocol Suite),
 description={Es un conjunto de protocolos de comunicación que se utiliza en Internet y en otras redes similares. Sus componentes más importantes son TCP e IP, y se encuentran en representados en un modelo de capas, que van desde la capa de enlace hasta la capa de aplicación, pasando por la capa de Internet y la de transporte.}
 }
 
 
 
