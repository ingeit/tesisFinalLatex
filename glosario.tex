%%%%%%%%%%%%%%%%%%%%%%%%%%%%%%%%%%%%%%%%%%%%%%%%%%%%%%%%
%   |------------------------------------------|       %
%   | Web App embebida en dispositivos móviles |       %
%   |  para la gestión de registros sobre la   |       %
%   |   contaminación de afluentes y ríos.     |       %
%   |                                          |       %
%   |          Proyecto de graduación          |       %
%   |__________________________________________|       %
%                                                      %
%   Autores                                            %
%   -------                                            %
%                                                      %
% * Bruno, Ricardo Hugo (CX 1409686)                   %
%     rburnount@gmail.com                              %
% * Gómez Véliz, Kevin Shionen (CX 1411828)            %
%     ing.gomezvelizkevin@gmail.com                    %
%                                                      %
%   Tutor                                              %
%   -------                                            %
%                                                      %
% * Ing. Cohen, Daniel Eduardo                         %
%        dcohen.tuc@gmail.com                          %
%                                                      %
%   Cotutor                                            %
%   -------                                            %
%                                                      %
% * Ing. Nieto, Luis Eduardo                           %
%        lnieto@herrera.unt.edu.ar                     %
%                                                      %
%                                                      %
%%%%%%%%%%%%%%%%%%%%%%%%%%%%%%%%%%%%%%%%%%%%%%%%%%%%%%%%

% ********* Glosario ********** %

%% Formato general
%\newglossaryentry{<label>}{<key-val list>} 

%% Ejemplo
%\newglossaryentry{set}% the label
%{name=set,            % the term
% description={a collection of objects} % a brief description
%}

\newglossaryentry{HTML}
{name=HTML,
description={HTML, siglas de HyperText Markup Language («lenguaje de marcado de hipertexto»), hace referencia al lenguaje de marcado predominante para la elaboración de páginas web que se utiliza para describir y traducir la estructura y la información en forma de texto, así como para complementar el texto con objetos tales como imágenes. El HTML se escribe en forma de «etiquetas», rodeadas por corchetes angulares (<,>). HTML también puede describir, hasta un cierto punto, la apariencia de un documento, y puede incluir un script (por ejemplo, JavaScript), el cual puede afectar el comportamiento de navegadores web y otros procesadores de HTML.}
}

\newglossaryentry{CSS}
{name=CSS,
description={La idea que se encuentra detrás del desarrollo de CSS es separar la estructura de un documento de su presentación. La información de estilo puede ser adjuntada como un documento separado o en el mismo documento HTML.}
}

\newglossaryentry{JavaScript}
{name=Javascript,
description={JavaScript es un lenguaje de programación interpretado,orientado a objetos, basado en prototipos, imperativo, débilmente tipado y dinámico. Se utiliza principalmente en su forma del lado del cliente (client-side), implementado como parte de un navegador web aunque existe una forma de JavaScript del lado del servidor. Su uso en aplicaciones externas a la web, por ejemplo en documentos PDF, aplicaciones de escritorio (mayoritariamente widgets) es también significativo.}
}

\newglossaryentry{framework}
{name=Framework (entorno de trabajo),
 description={Es un conjunto estandarizado de conceptos, prácticas y criterios para enfocar un tipo de problemática particular que sirve como referencia, para enfrentar y resolver nuevos problemas de índole similar.
 En el desarrollo de software, un entorno de trabajo es una estructura conceptual y tecnológica de asistencia definida, normalmente, con artefactos o módulos concretos de software, que puede servir de base para la organización y desarrollo de software. Típicamente, puede incluir soporte de programas, bibliotecas, y un lenguaje interpretado, entre otras herramientas, para así ayudar a desarrollar y unir los diferentes componentes de un proyecto.}
 }

 \newglossaryentry{URI}
{name=URI (Identificador de recursos uniforme),
 description={ Es una cadena de caracteres que identifica los recursos de una red de forma unívoca. La diferencia respecto a un localizador de recursos uniforme (URL) es que estos últimos hacen referencia a recursos que, de forma general, pueden variar en el tiempo. Normalmente estos recursos son accesibles en una red o sistema. Los URI pueden ser localizador de recursos uniforme (URL), uniform resource name (URN), o ambos.}
 }

 \newglossaryentry{UML}
{name=UML,
 description={El lenguaje unificado de modelado (UML, por sus siglas en inglés, Unified Modeling Language) es el lenguaje de modelado de sistemas de software más conocido y utilizado en la actualidad; está respaldado por el Object Management Group (OMG).
 Es un lenguaje gráfico para visualizar, especificar, construir y documentar un sistema. UML ofrece un estándar para describir un "plano" del sistema (modelo), incluyendo aspectos conceptuales tales como procesos, funciones del sistema, y aspectos concretos como expresiones de lenguajes de programación, esquemas de bases de datos y compuestos reciclados.}

\newglossaryentry{SO}
{name=Sistema Operativo,
 description={Un sistemaasdasdad operativo (SO, frecuentemente OS, del inglés Operating System) es un programa o conjunto de programas que en un sistema informático gestiona los recursos de hardware y provee servicios a los programas de aplicación, ejecutándose en modo privilegiado respecto de los restantes.}
 }

\newglossaryentry{LINUX}
{name=Linux,
 description={Linux es un núcleo libre de sistema operativo basado en Unix. Es uno de los principales ejemplos de software libre. Linux está licenciado bajo la GPL v2 y está desarrollado por colaboradores de todo el mundo.}
 }

\newglossaryentry{MACOS}
{name=Mac OS,
 description={Mac OS (del inglés Macintosh Operating System, en español Sistema Operativo de Macintosh) es el nombre del sistema operativo creado por Apple para su línea de computadoras Macintosh, también aplicado retroactivamente a las versiones anteriores a System 7.6, y que apareció por primera vez en System 7.5.1. Es conocido por haber sido uno de los primeros sistemas dirigidos al gran público en contar con una interfaz gráfica compuesta por la interacción del mouse con ventanas, iconos y menús.}
 }

 
  \newglossaryentry{HTTPS}
{name=HTTPS,
 description={Hypertext Transfer Protocol Secure (o Protocolo seguro de transferencia de hipertexto), más conocido por sus siglas HTTPS, es un protocolo de aplicación basado en el protocolo HTTP, destinado a la transferencia segura de datos de Hiper Texto, es decir, es la versión segura de HTTP.}
 }
 
\newglossaryentry{TCP}
{name=TCP/IP (Internet Protocol Suite),
 description={Es un conjunto de protocolos de comunicación que se utiliza en Internet y en otras redes similares. Sus componentes más importantes son TCP e IP, y se encuentran en representados en un modelo de capas, que van desde la capa de enlace hasta la capa de aplicación, pasando por la capa de Internet y la de transporte.}
 }