%%%%%%%%%%%%%%%%%%%%%%%%%%%%%%%%%%%%%%%%%%%%%%%%%%%%%%%%
%   |------------------------------------------|       %
%   | Web App embebida en dispositivos móviles |       %
%   |  para la gestión de registros sobre la   |       %
%   |   contaminación de afluentes y ríos.     |       %
%   |                                          |       %
%   |          Proyecto de graduación          |       %
%   |__________________________________________|       %
%                                                      %
%   Autores                                            %
%   -------                                            %
%                                                      %
% * Bruno, Ricardo Hugo (CX 1409686)                   %
%     rburnount@gmail.com                              %
% * Gomez Veliz, Kevin Shionen (CX 1411828)            %
%     ing.gomezvelizkevin@gmail.com                    %
%                                                      %
%   Tutor                                              %
%   -------                                            %
%                                                      %
% * Ing. Cohen, Daniel Eduardo                         %
%        dcohen.tuc@gmail.com                          %
%                                                      %
%   Cotutor                                            %
%   -------                                            %
%                                                      %
% * Ing. Nieto, Luis Eduardo                           %
%        lnieto@herrera.unt.edu.ar                     %
%                                                      %
%                                                      %
%%%%%%%%%%%%%%%%%%%%%%%%%%%%%%%%%%%%%%%%%%%%%%%%%%%%%%%%

% ********* Glosario ********** %

%% Formato general
%\newglossaryentry{<label>}{<key-val list>} 

%% Ejemplo
%\newglossaryentry{set}% the label
%{name=set,            % the term
% description={a collection of objects} % a brief description
%}

\newglossaryentry{SO}
{name=Sistema Operativo,
 description={Un sistemaasdasdad operativo (SO, frecuentemente OS, del inglés Operating System) es un programa o conjunto de programas que en un sistema informático gestiona los recursos de hardware y provee servicios a los programas de aplicación, ejecutándose en modo privilegiado respecto de los restantes.}
 }

\newglossaryentry{Open Source}
{name=Open Source,
 description={Código abierto (o fuente abierta) es el término con el que se conoce al software distribuido y desarrollado libremente. El código abierto tiene un punto de vista más orientado a los beneficios prácticos de poder acceder al código, que a las cuestiones éticas y morales las cuales se destacan en el software libre.}
 }

\newglossaryentry{Linux}
{name=Linux,
 description={Linux es un núcleo libre de sistema operativo basado en Unix. Es uno de los principales ejemplos de software libre. Linux está licenciado bajo la GPL v2 y está desarrollado por colaboradores de todo el mundo.}
 }

 \newglossaryentry{HTML}
{name=HTML,
 description={HTML, siglas de HyperText Markup Language («lenguaje de marcado de hipertexto»), hace referencia al lenguaje de marcado predominante para la elaboración de páginas web que se utiliza para describir y traducir la estructura y la información en forma de texto, así como para complementar el texto con objetos tales como imágenes. El HTML se escribe en forma de «etiquetas», rodeadas por corchetes angulares (<,>). HTML también puede describir, hasta un cierto punto, la apariencia de un documento, y puede incluir un script (por ejemplo, JavaScript), el cual puede afectar el comportamiento de navegadores web y otros procesadores de HTML.}
 }
 
 \newglossaryentry{CSS}
{name=CSS,
 description={La idea que se encuentra detrás del desarrollo de CSS es separar la estructura de un documento de su presentación. La información de estilo puede ser adjuntada como un documento separado o en el mismo documento HTML.}
 }
 
 \newglossaryentry{JavaScript}
{name=Javascript,
 description={JavaScript es un lenguaje de programación interpretado,orientado a objetos, basado en prototipos, imperativo, débilmente tipado y dinámico. Se utiliza principalmente en su forma del lado del cliente (client-side), implementado como parte de un navegador web aunque existe una forma de JavaScript del lado del servidor. Su uso en aplicaciones externas a la web, por ejemplo en documentos PDF, aplicaciones de escritorio (mayoritariamente widgets) es también significativo.}
 }
 
  \newglossaryentry{HTTPS}
{name=HTTPS,
 description={Hypertext Transfer Protocol Secure (o Protocolo seguro de transferencia de hipertexto), más conocido por sus siglas HTTPS, es un protocolo de aplicación basado en el protocolo HTTP, destinado a la transferencia segura de datos de Hiper Texto, es decir, es la versión segura de HTTP.}
 }
 
\newglossaryentry{Subversion}
{name=Subversion,
 description={Sistema de control de versiones para el desarrollo colectivo de software o documentación.}
 }
 
\newglossaryentry{TCP/IP}
{name=TCP/IP (Internet Protocol Suite),
 description={Es un conjunto de protocolos de comunicación que se utiliza en Internet y en otras redes similares. Sus componentes más importantes son TCP e IP, y se encuentran en representados en un modelo de capas, que van desde la capa de enlace hasta la capa de aplicación, pasando por la capa de Internet y la de transporte.}
 }
 
 \newglossaryentry{TCP}
{name=TCP (Internet Protocol Suite),
 description={Es un conjunto de protocolos de comunicación que se utiliza en Internet y en otras redes similares. Sus componentes más importantes son TCP e IP, y se encuentran en representados en un modelo de capas, que van desde la capa de enlace hasta la capa de aplicación, pasando por la capa de Internet y la de transporte.}
 }
 
 \newglossaryentry{framework}
{name=TCP/IP (Internet Protocol Suite),
 description={Es un conjunto de protocolos de comunicación que se utiliza en Internet y en otras redes similares. Sus componentes más importantes son TCP e IP, y se encuentran en representados en un modelo de capas, que van desde la capa de enlace hasta la capa de aplicación, pasando por la capa de Internet y la de transporte.}
 }
 
 \newglossaryentry{XML}
{name=TCP/IP (Internet Protocol Suite),
 description={Es un conjunto de protocolos de comunicación que se utiliza en Internet y en otras redes similares. Sus componentes más importantes son TCP e IP, y se encuentran en representados en un modelo de capas, que van desde la capa de enlace hasta la capa de aplicación, pasando por la capa de Internet y la de transporte.}
 }
 
 \newglossaryentry{JSON}
{name=TCP/IP (Internet Protocol Suite),
 description={Es un conjunto de protocolos de comunicación que se utiliza en Internet y en otras redes similares. Sus componentes más importantes son TCP e IP, y se encuentran en representados en un modelo de capas, que van desde la capa de enlace hasta la capa de aplicación, pasando por la capa de Internet y la de transporte.}
 }
 
 \newglossaryentry{SOAP}
{name=TCP/IP (Internet Protocol Suite),
 description={Es un conjunto de protocolos de comunicación que se utiliza en Internet y en otras redes similares. Sus componentes más importantes son TCP e IP, y se encuentran en representados en un modelo de capas, que van desde la capa de enlace hasta la capa de aplicación, pasando por la capa de Internet y la de transporte.}
 }
 
 \newglossaryentry{URI}
{name=TCP/IP (Internet Protocol Suite),
 description={Es un conjunto de protocolos de comunicación que se utiliza en Internet y en otras redes similares. Sus componentes más importantes son TCP e IP, y se encuentran en representados en un modelo de capas, que van desde la capa de enlace hasta la capa de aplicación, pasando por la capa de Internet y la de transporte.}
 }
 
  \newglossaryentry{Man-in-the-middle}
{name=TCP/IP (Internet Protocol Suite),
 description={Es un conjunto de protocolos de comunicación que se utiliza en Internet y en otras redes similares. Sus componentes más importantes son TCP e IP, y se encuentran en representados en un modelo de capas, que van desde la capa de enlace hasta la capa de aplicación, pasando por la capa de Internet y la de transporte.}
 }
 
 \newglossaryentry{SSL/TLS}
{name=TCP/IP (Internet Protocol Suite),
 description={Es un conjunto de protocolos de comunicación que se utiliza en Internet y en otras redes similares. Sus componentes más importantes son TCP e IP, y se encuentran en representados en un modelo de capas, que van desde la capa de enlace hasta la capa de aplicación, pasando por la capa de Internet y la de transporte.}
 }
 
 \newglossaryentry{ACL}
{name=TCP/IP (Internet Protocol Suite),
 description={Es un conjunto de protocolos de comunicación que se utiliza en Internet y en otras redes similares. Sus componentes más importantes son TCP e IP, y se encuentran en representados en un modelo de capas, que van desde la capa de enlace hasta la capa de aplicación, pasando por la capa de Internet y la de transporte.}
 }
 
 \newglossaryentry{DOM}
{name=TCP/IP (Internet Protocol Suite),
 description={Es un conjunto de protocolos de comunicación que se utiliza en Internet y en otras redes similares. Sus componentes más importantes son TCP e IP, y se encuentran en representados en un modelo de capas, que van desde la capa de enlace hasta la capa de aplicación, pasando por la capa de Internet y la de transporte.}
 }
