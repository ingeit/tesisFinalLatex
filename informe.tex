%%%%%%%%%%%%%%%%%%%%%%%%%%%%%%%%%%%%%%%%%%%%%%%%%%%%%%%%
%   |------------------------------------------|       %
%   | Web App embebida en dispositivos móviles |       %
%   |  para la gestión de registros sobre la   |       %
%   |   contaminación de afluentes y ríos.     |       %
%   |                                          |       %
%   |          Proyecto de graduación          |       %
%   |__________________________________________|       %
%                                                      %
%   Autores                                            %
%   -------                                            %
%                                                      %
% * Bruno, Ricardo Hugo (CX 1409686)                   %
%     rburnount@gmail.com                              %
% * Gomez Veliz, Kevin Shionen (CX 1411828)            %
%     ing.gomezvelizkevin@gmail.com                    %
%                                                      %
%   Tutor                                              %
%   -------                                            %
%                                                      %
% * Ing. Cohen, Daniel Eduardo                         %
%        dcohen.tuc@gmail.com                          %
%                                                      %
%   Cotutor                                            %
%   -------                                            %
%                                                      %
% * Ing. Nieto, Luis Eduardo                           %
%        lnieto@herrera.unt.edu.ar                     %
%                                                      %
%                                                      %
%%%%%%%%%%%%%%%%%%%%%%%%%%%%%%%%%%%%%%%%%%%%%%%%%%%%%%%%

% ********* Informe principal ********** %

%% Clase del documento tipo reporte
\documentclass[12pt,a4paper]{report}

%% Paquetes adicionales
\usepackage{hyperref}
\usepackage[spanish,activeacute]{babel}    		% Soporte spanish
\usepackage[spanish]{translator}
\usepackage[utf8]{inputenc}		% Entradas con acentos, eñes, etc
\usepackage{float}					% Para agregar imagenes fijas
\usepackage{latexsym} 				% Símbolos
\usepackage{graphicx} 				% Inclusión de gráficos
\usepackage[pdftex=true,colorlinks=true,plainpages=true]{hyperref} % Soporte hipertexto
%\usepackage[pdftex=true,plainpages=true]{hyperref} % Soporte hipertexto sin pintar los enlaces
\usepackage{anysize} 				% Soporte para el comando \marginsize
\usepackage{makeidx} 				% Soporte de indices alfabeticos
\usepackage{subfigure} 				% Soporte de indices alfabeticos
\usepackage[none]{tocbibind} % Agregar bibliografía, indices, etc. al indice general
\usepackage{color}					% Uso de colores
\usepackage{listings}				% Para gregar listas personalizadas (codigos)
\usepackage{array}					% para que funcionen los parámetros m de \begin{tabular}
\usepackage{alltt}					% Extension de verbatim para usar comandos latex.
\usepackage{fancyhdr}				% Incluye encabezados y pie de páginas más complejos.
\usepackage{glossaries}     % Incluye glosario de términos técnicos.
\usepackage{verbatim} 
\usepackage{longtable}
\usepackage{multirow}
\usepackage{enumerate}      %Soporte de listas enumeradas con letras
\usepackage[table]{xcolor}
\usepackage{framed}     %Texto recuadrado



%Uso de fuente Libertine
%\usepackage{libertine}
%\usepackage[T1]{fontenc}

%Defino comando alternativo que consiste en un paragraph con salto de línea y se usa en los flujos alternativos de los casos de uso
\newcommand{\alternativo}[1]{\paragraph{#1}\mbox{}\\}

\newcommand{\HRule}{\rule{\linewidth}{0.5mm}}

\renewcommand{\labelitemi}{$\bullet$}
\renewcommand{\labelitemii}{$-$}

%Remover numeraciones de capítulos y secciones
\setcounter{secnumdepth}{-1}

%% Para la ruptura de palabras
%\include{divPalabras}

%Separación entre párrafos
%%%%\setlength{\parskip}{\baselineskip} 

%interlineado
\renewcommand{\baselinestretch}{1.5}

%% Definición de colores grises
\definecolor{gray99}{gray}{.99}
\definecolor{gray97}{gray}{.97}
\definecolor{gray85}{gray}{.85}
\definecolor{gray75}{gray}{.75} 
\definecolor{gray45}{gray}{.45}
\definecolor{grisOscuro}{HTML}{BFBFBF}
\definecolor{grisClaro}{HTML}{DFDFDF}

%% Lista personalizada para códigos y configuraciones.
\lstset{ frame=Ltb,
     framerule=0pt,
     aboveskip=0.5cm,
     framextopmargin=3pt,
     framexbottommargin=3pt,
     framexleftmargin=0cm,
     framesep=0pt,
     rulesep=.4pt,
     backgroundcolor=\color{gray97},
     rulesepcolor=\color{black},
     stringstyle=\ttfamily,
     showstringspaces = false,
     basicstyle=\small\ttfamily,
     commentstyle=\color{gray45},
     keywordstyle=\bfseries,
     numbers=left,
     numbersep=5pt,
     numberstyle=\tiny,
     numberfirstline = false,
     breaklines=true,
   }
 
\lstnewenvironment{listing}[1][]
   {\lstset{#1}\pagebreak[0]}{\pagebreak[0]}
 
\lstdefinestyle{consola}
   {basicstyle=\scriptsize\bf\ttfamily,
    backgroundcolor=\color{gray85},
   }
   
\lstdefinestyle{configuracion}
   {basicstyle=\small\bf\ttfamily,
    backgroundcolor=\color{gray97},
   }
\lstdefinestyle{configuracion_small}	% Lo mismo que arriba pero más chiquito
   {basicstyle=\footnotesize\bf\ttfamily,
    backgroundcolor=\color{gray97},
   }

\lstdefinestyle{TeX}
   {language=TeX,
   }
   
\lstdefinestyle{bash}
   {language=bash,
   }

%% Definición de los códigos, configuraciones y logs flotantes.
%% Codigos Latex
\floatstyle{plain}
\newfloat{codigo}{thb}{lop}
\floatname{codigo}{Código}
%% Codigos de configuración
\floatstyle{ruled}
\newfloat{configuracion}{thb}{lop}
\floatname{configuracion}{Configuración}
%
\newfloat{configuracion_small}{htb}{lop}
\floatname{configuracion_small}{Detalle}
%% Cuadros de registros (logs)
\floatstyle{boxed}
\newfloat{logs}{thb}{lop}
\floatname{logs}{Registro}
\hypersetup{
  colorlinks = true,
  linkcolor  = black
}

%% Espaciado entre lineas
%\baselinestretch{2.0}

%% Encabezados y pie de página
\pagestyle{headings}
% \pagestyle{fancy}
  
% \lhead{\includegraphics[width=0.07\textwidth]{imagenes/logo-unt.png}}

%% Título, autor y fecha
\title{Web App embebida en dispositivos móviles para la gestión de registros sobre la contaminación de afluentes y ríos.}
\author{\href{mailto:rbrunount@gmail.com}{Ricardo Hugo Bruno}\\
\href{mailto:masterk63@gmail.com}{Kevin Shionen Gomez Veliz}}
\date{2018}

% Diccionario de términos técnicos (glosario)
% Generar el glosario por cada entrada nueva, con el comando (vieja no saques esa línea por favor):
%\makeindex -s informe.ist -t informe.glg -o informe.gls informe.glo

\makenoidxglossaries
%%%%%%%%%%%%%%%%%%%%%%%%%%%%%%%%%%%%%%%%%%%%%%%%%%%%%%%%
%   |------------------------------------------|       %
%   |Aplicación de comercio electrónico para   |       %
%   |teléfonos móviles con S.O. Android        |       %
%   |                                          |       %
%   | Proyecto de graduación                   |       %
%   |__________________________________________|       %
%                                                      %
%   Autores                                            %
%   -------                                            %
%                                                      %
% * Soto, Paula Fabiana (CX05-0967-4)                  %
%     paulette255@gmail.com                            %
% * Vallejo, Sergio Daniel (CX05-0392-4)               %
%     vallejosergio@gmail.com                          %
%                                                      %
%   Tutor                                              %
%   -------                                            %
%                                                      %
% * Ing. Augusto Maximiliano Odstrcil                  %
%        modstrcil@gmail.com                           %
%                                                      %
%                                                      %
%%%%%%%%%%%%%%%%%%%%%%%%%%%%%%%%%%%%%%%%%%%%%%%%%%%%%%%%

% ********* Glosario ********** %

%% Formato general
%\newglossaryentry{<label>}{<key-val list>} 

%% Ejemplo
%\newglossaryentry{set}% the label
%{name=set,            % the term
% description={a collection of objects} % a brief description
%}

\newglossaryentry{SO}
{name=Sistema Operativo,
 description={Un sistemaasdasdad operativo (SO, frecuentemente OS, del inglés Operating System) es un programa o conjunto de programas que en un sistema informático gestiona los recursos de hardware y provee servicios a los programas de aplicación, ejecutándose en modo privilegiado respecto de los restantes.}
 }

\newglossaryentry{Open Source}
{name=Open Source,
 description={Código abierto (o fuente abierta) es el término con el que se conoce al software distribuido y desarrollado libremente. El código abierto tiene un punto de vista más orientado a los beneficios prácticos de poder acceder al código, que a las cuestiones éticas y morales las cuales se destacan en el software libre.}
 }

\newglossaryentry{Linux}
{name=Linux,
 description={Linux es un núcleo libre de sistema operativo basado en Unix. Es uno de los principales ejemplos de software libre. Linux está licenciado bajo la GPL v2 y está desarrollado por colaboradores de todo el mundo.}
 }

\newglossaryentry{Java}
{name=Java,
 description={Java es un lenguaje de programación publicado en el 1995 como un componente fundamental de la plataforma Java de Sun Microsystems. El lenguaje deriva mucho de su sintaxis de C y C++, pero tiene menos facilidades de bajo nivel que cualquiera de ellos. Las aplicaciones de Java son generalmente compiladas a bytecode (clase Java) que puede correr en cualquier máquina virtual Java (JVM) sin importar la arquitectura de la computadora. Java es un lenguaje de programación de propósito general, concurrente, basado en clases, y orientado a objetos, que fue diseñado específicamente para tener tan pocas dependencias de implementación como fuera posible. Su intención es permitir que los desarrolladores de aplicaciones escriban el programa una vez y lo ejecuten en cualquier dispositivo (conocido en inglés como WORA, o "write once, run anywhere"), lo que quiere decir que el código que es ejecutado en una plataforma no tiene que ser recompilado para correr en otra. Java es, a partir del 2012, uno de los lenguajes de programación más populares en uso, particularmente para aplicaciones de cliente-servidor de web, con unos 10 millones de usuarios reportados.}
 }
 
 \newglossaryentry{Python}
{name=Python,
 description={Python es un lenguaje de programación interpretado cuya filosofía hace hincapié en una sintaxis muy limpia y que favorezca un código legible. Se trata de un lenguaje de programación multiparadigma, ya que soporta orientación a objetos, programación imperativa y, en menor medida, programación funcional. Es un lenguaje interpretado, usa tipado dinámico y es multiplataforma.}
 }
 
 \newglossaryentry{HTML}
{name=HTML,
 description={HTML, siglas de HyperText Markup Language («lenguaje de marcado de hipertexto»), hace referencia al lenguaje de marcado predominante para la elaboración de páginas web que se utiliza para describir y traducir la estructura y la información en forma de texto, así como para complementar el texto con objetos tales como imágenes. El HTML se escribe en forma de «etiquetas», rodeadas por corchetes angulares (<,>). HTML también puede describir, hasta un cierto punto, la apariencia de un documento, y puede incluir un script (por ejemplo, JavaScript), el cual puede afectar el comportamiento de navegadores web y otros procesadores de HTML.}
 }
 
 \newglossaryentry{CSS}
{name=CSS,
 description={La idea que se encuentra detrás del desarrollo de CSS es separar la estructura de un documento de su presentación. La información de estilo puede ser adjuntada como un documento separado o en el mismo documento HTML.}
 }
 
 \newglossaryentry{JavaScript}
{name=Javascript,
 description={JavaScript es un lenguaje de programación interpretado,orientado a objetos, basado en prototipos, imperativo, débilmente tipado y dinámico. Se utiliza principalmente en su forma del lado del cliente (client-side), implementado como parte de un navegador web aunque existe una forma de JavaScript del lado del servidor. Su uso en aplicaciones externas a la web, por ejemplo en documentos PDF, aplicaciones de escritorio (mayoritariamente widgets) es también significativo.}
 }
 
  \newglossaryentry{HTTPS}
{name=HTTPS,
 description={Hypertext Transfer Protocol Secure (o Protocolo seguro de transferencia de hipertexto), más conocido por sus siglas HTTPS, es un protocolo de aplicación basado en el protocolo HTTP, destinado a la transferencia segura de datos de Hiper Texto, es decir, es la versión segura de HTTP.}
 }
 
 
 
\newglossaryentry{Subversion}
{name=Subversion,
 description={Sistema de control de versiones para el desarrollo colectivo de software o documentación.}
 }
 
\newglossaryentry{TCP/IP}
{name=TCP/IP (Internet Protocol Suite),
 description={Es un conjunto de protocolos de comunicación que se utiliza en Internet y en otras redes similares. Sus componentes más importantes son TCP e IP, y se encuentran en representados en un modelo de capas, que van desde la capa de enlace hasta la capa de aplicación, pasando por la capa de Internet y la de transporte.}
 }
 
 \newglossaryentry{TCP}
{name=TCP/IP (Internet Protocol Suite),
 description={Es un conjunto de protocolos de comunicación que se utiliza en Internet y en otras redes similares. Sus componentes más importantes son TCP e IP, y se encuentran en representados en un modelo de capas, que van desde la capa de enlace hasta la capa de aplicación, pasando por la capa de Internet y la de transporte.}
 }
 
 \newglossaryentry{framework}
{name=TCP/IP (Internet Protocol Suite),
 description={Es un conjunto de protocolos de comunicación que se utiliza en Internet y en otras redes similares. Sus componentes más importantes son TCP e IP, y se encuentran en representados en un modelo de capas, que van desde la capa de enlace hasta la capa de aplicación, pasando por la capa de Internet y la de transporte.}
 }
 
 \newglossaryentry{XML}
{name=TCP/IP (Internet Protocol Suite),
 description={Es un conjunto de protocolos de comunicación que se utiliza en Internet y en otras redes similares. Sus componentes más importantes son TCP e IP, y se encuentran en representados en un modelo de capas, que van desde la capa de enlace hasta la capa de aplicación, pasando por la capa de Internet y la de transporte.}
 }
 
 \newglossaryentry{JSON}
{name=TCP/IP (Internet Protocol Suite),
 description={Es un conjunto de protocolos de comunicación que se utiliza en Internet y en otras redes similares. Sus componentes más importantes son TCP e IP, y se encuentran en representados en un modelo de capas, que van desde la capa de enlace hasta la capa de aplicación, pasando por la capa de Internet y la de transporte.}
 }
 
 \newglossaryentry{SOAP}
{name=TCP/IP (Internet Protocol Suite),
 description={Es un conjunto de protocolos de comunicación que se utiliza en Internet y en otras redes similares. Sus componentes más importantes son TCP e IP, y se encuentran en representados en un modelo de capas, que van desde la capa de enlace hasta la capa de aplicación, pasando por la capa de Internet y la de transporte.}
 }
 
 \newglossaryentry{URI}
{name=TCP/IP (Internet Protocol Suite),
 description={Es un conjunto de protocolos de comunicación que se utiliza en Internet y en otras redes similares. Sus componentes más importantes son TCP e IP, y se encuentran en representados en un modelo de capas, que van desde la capa de enlace hasta la capa de aplicación, pasando por la capa de Internet y la de transporte.}
 }
 
  \newglossaryentry{Man-in-the-middle}
{name=TCP/IP (Internet Protocol Suite),
 description={Es un conjunto de protocolos de comunicación que se utiliza en Internet y en otras redes similares. Sus componentes más importantes son TCP e IP, y se encuentran en representados en un modelo de capas, que van desde la capa de enlace hasta la capa de aplicación, pasando por la capa de Internet y la de transporte.}
 }
 
 \newglossaryentry{SSL/TLS}
{name=TCP/IP (Internet Protocol Suite),
 description={Es un conjunto de protocolos de comunicación que se utiliza en Internet y en otras redes similares. Sus componentes más importantes son TCP e IP, y se encuentran en representados en un modelo de capas, que van desde la capa de enlace hasta la capa de aplicación, pasando por la capa de Internet y la de transporte.}
 }
 
 \newglossaryentry{ACL}
{name=TCP/IP (Internet Protocol Suite),
 description={Es un conjunto de protocolos de comunicación que se utiliza en Internet y en otras redes similares. Sus componentes más importantes son TCP e IP, y se encuentran en representados en un modelo de capas, que van desde la capa de enlace hasta la capa de aplicación, pasando por la capa de Internet y la de transporte.}
 }
 
 \newglossaryentry{DOM}
{name=TCP/IP (Internet Protocol Suite),
 description={Es un conjunto de protocolos de comunicación que se utiliza en Internet y en otras redes similares. Sus componentes más importantes son TCP e IP, y se encuentran en representados en un modelo de capas, que van desde la capa de enlace hasta la capa de aplicación, pasando por la capa de Internet y la de transporte.}
 }
 
 \newglossaryentry{AJAX}
{name=TCP/IP (Internet Protocol Suite),
 description={Es un conjunto de protocolos de comunicación que se utiliza en Internet y en otras redes similares. Sus componentes más importantes son TCP e IP, y se encuentran en representados en un modelo de capas, que van desde la capa de enlace hasta la capa de aplicación, pasando por la capa de Internet y la de transporte.}
 }
 
 
 

% Además de incluir la entrada en glosario.tex, hay que referenciarlo con \gls{texto}, sino no aparece. Y ejecutar el comando de arriba para generar.


%%%% Inicio del informe %%%%
\begin{document}
%\maketitle % Título


\begin{titlepage}
 
\begin{center}
 
 
% Upper part of the page
\includegraphics[width=0.15\textwidth]{imagenes/logo-unt.png}\\[0.5cm]
 
\textsc{\LARGE Universidad Nacional de Tucumán}\\[0.5cm]

\textsc{\normalsize Facultad de Ciencias Exactas y Tecnología}\\[0.5cm]

\textsc{\small Departamento de Electricidad, Electrónica y Computación}\\[0.7cm]
 
\textsc{\LARGE Trabajo de Graduación}\\[0.1cm]

\textsc{\normalsize Informe Final}\\[0.1cm]
 
 
% Title
\HRule \\[0.4cm]
{ \LARGE \bfseries Web App embebida en dispositivos móviles para la gestión de registros sobre la contaminación de afluentes y ríos.
}\\[0.4cm]
\HRule \\[0.4cm]
 
\end{center} 

\section*{Autores}
\textsc{Bruno}, Ricardo Hugo (CX 1409686) - Ing. en Computación

\textsc{Gomez Veliz}, Kevin Shionen (CX 1411828) - Ing. en Computación

\section*{Tutores}
\textsc{Tutor - } Ing. \textsc{Cohen} Daniel Eduardo

\textsc{Cotutor - } Ing. \textsc{Nieto} Luis Eduardo

\begin{center}
\subsection*{1 de Marzo de 2018}
\end{center}

\end{titlepage}

%% Agradecimientos
%%%%%%%%%%%%%%%%%%%%%%%%%%%%%%%%%%%%%%%%%%%%%%%%%%%%%%%%
%   |------------------------------------------|       %
%   | Web App embebida en dispositivos móviles |       %
%   |  para la gestión de registros sobre la   |       %
%   |   contaminación de afluentes y ríos.     |       %
%   |                                          |       %
%   |          Proyecto de graduación          |       %
%   |__________________________________________|       %
%                                                      %
%   Autores                                            %
%   -------                                            %
%                                                      %
% * Bruno, Ricardo Hugo (CX 1409686)                   %
%     rburnount@gmail.com                              %
% * Gomez Veliz, Kevin Shionen (CX 1411828)            %
%     ing.gomezvelizkevin@gmail.com                    %
%                                                      %
%   Tutor                                              %
%   -------                                            %
%                                                      %
% * Ing. Cohen, Daniel Eduardo                         %
%        dcohen.tuc@gmail.com                          %
%                                                      %
%   Cotutor                                            %
%   -------                                            %
%                                                      %
% * Ing. Nieto, Luis Eduardo                           %
%        lnieto@herrera.unt.edu.ar                     %
%                                                      %
%                                                      %
%%%%%%%%%%%%%%%%%%%%%%%%%%%%%%%%%%%%%%%%%%%%%%%%%%%%%%%%

% ********* Agradecimientos ********** %
%El * se agrega para que LaTeX no le asigne un número de capítulo
\chapter*{Agradecimientos}
%Se debe agregar esta línea para que este capítulo aparezca en el índice
%\addcontentsline{toc}{chapter}{Agradecimientos}


Agradecemos a nuestras familias que nos apoyaron y ayudaron durante el transcurso de la carrera.

Agradecemos a la Universidad en su conjunto, pública y gratuita, que nos formó académicamente.

Gracias a nuestro tutor Cohen, Daniel Eduardo por brindarnos la posibilidad de desarrollar este proyecto. Ademas agradecemos a nuestro cotutor Nieto, Luis Eduardo por brindarnos ayuda y acompañamiento de igual manera que nuestro tutor.

Por último, queremos agradecer a nuestros compañeros y amigos por los momentos de estudio, logros y festejos compartidos durante la carrera, que nos motivaron para seguir adelante.

\label{chap:agradecimientos}






\tableofcontents % Tabla de contenido

%% Introducción.
%%%%%%%%%%%%%%%%%%%%%%%%%%%%%%%%%%%%%%%%%%%%%%%%%%%%%%%%
%   |------------------------------------------|       %
%   | Web App embebida en dispositivos móviles |       %
%   |  para la gestión de registros sobre la   |       %
%   |   contaminación de afluentes y ríos.     |       %
%   |                                          |       %
%   |          Proyecto de graduación          |       %
%   |__________________________________________|       %
%                                                      %
%   Autores                                            %
%   -------                                            %
%                                                      %
% * Bruno, Ricardo Hugo (CX 1409686)                   %
%     rburnount@gmail.com                              %
% * Gomez Veliz, Kevin Shionen (CX 1411828)            %
%     ing.gomezvelizkevin@gmail.com                    %
%                                                      %
%   Tutor                                              %
%   -------                                            %
%                                                      %
% * Ing. Cohen, Daniel Eduardo                         %
%        dcohen.tuc@gmail.com                          %
%                                                      %
%   Cotutor                                            %
%   -------                                            %
%                                                      %
% * Ing. Nieto, Luis Eduardo                           %
%        lnieto@herrera.unt.edu.ar                     %
%                                                      %
%                                                      %
%%%%%%%%%%%%%%%%%%%%%%%%%%%%%%%%%%%%%%%%%%%%%%%%%%%%%%%%

% ********* Introducción ********** %

%TODO: *Agregar Secciones de Android

\chapter{Introducción}
\label{chap:introduccion}

\begin{comment}
Requisitos del proyecto y los problemas a encarar
Razón de ser del proyecto: aplicación, investigación, etc.
Especificaciones generales del proyecto.


introducción es una sección inicial que establece el propósito y los objetivos de todo el contenido posterior del escrito. En general va seguido del cuerpo o desarrollo del tema, y de las conclusiones.

En la introducción normalmente se describe el alcance del documento, y se da una breve explicación o resumen de éste. También puede explicar algunos antecedentes que son importantes para el posterior desarrollo del tema central. Un lector al leer una introducción debería poder hacerse una idea sobre el contenido del texto, antes de comenzar su lectura propiamente dicha.

En artículos técnicos, la introducción generalmente incluye una o más sub-secciones estándar, como lo son el resumen o síntesis, el prefacio y los agradecimientos. Alternativamente, la sección de introducción puede ser un capítulo más del trabajo en sí, dividido en las sub-secciones anteriormente mencionadas. Cuando el libro se divide en capítulos numerados, por convención la introducción y cualquier otro asunto delante de las secciones de cuerpo o desarrollo no se enumeran (o se enumeran de manera distinta) y preceden al capítulo 1.

El concepto de introducción es independiente del contenido del documento al cual introduce. Siempre debe presentar el objeto o problema a desarrollar, ya este se trate de una especificación formal, un producto, un personaje o un ente cualquiera.
\end{comment}

La tecnología de los dispositivos móviles ha avanzado rápidamente en los últimos años, llegando a ser actualmente auténticas computadoras de bolsillo. La gran demanda por este tipo de dispositivos genera un gran interés por parte de empresas/instituciones que desean crear aplicaciones para un mercado en pleno auge, buscando aprovechar no sólo la gran cantidad de usuarios de estas plataformas, sino también la posibilidad de ofrecer funcionalidades y capacidades imposibles para sus procesos actuales.

% Muchas empresas comenzaron a desarrollar aplicaciones para teléfonos celulares y tablets como un complemento de sus sistemas existentes, haciendo que sea necesario que se desarrollen tecnologías para la comunicación entre la aplicación principal y los clientes que sean seguras y eficientes.

La plataforma que actualmente posee una mayor cantidad de usuarios y mayor crecimiento es Android, debido a que se trata de un Sistema Operativo abierto que cualquier fabricante puede adaptar e instalar en sus dispositivos, que está en constante evolución, y que aporta gran cantidad de servicios y aplicaciones. 

\section{Objetivos}

Los principales objetivos del sistema a desarrollar son permitir que usuarios puedan realizar el estudio de campo de una manera rápida y eficiente valiéndose de la tecnología de un smartphone, con cualquier SO instalado, que hoy en dia se realiza de forma manual con lápiz y papel.

El sistema cuenta de dos partes, una aplicación movil para generar registros de las muestras del estudio de campo, y una aplicación web para gestionar y administrar dichos registros.

El sistema debe ser seguro, escalable, de fácil mantenimiento y muy simple de usar. Se deben implementar mecanismos de seguridad que impidan la visualización o alteración indebida de los datos sensibles de cada usuario.

Nuestros objetivos personales son aplicar los conocimientos adquiridos a lo largo de la carrera, el aprendizaje del trabajo en equipo, la coordinación en el desarrollo y sumar experiencia y conocimientos.


\section{Android}
Android es un sistema operativo Open Source, para dispositivos móviles basado en Linux. La mayor parte de su código fue publicado por Google bajo la licencia Apache. Actualmente se lo puede encontrar instalado en teléfonos celulares, tablet PCs y televisores.

%La estructura del sistema operativo Android se compone de aplicaciones que se ejecutan en un framework Java de aplicaciones orientadas a objetos sobre el núcleo de las bibliotecas de Java en una máquina virtual Dalvik con compilación en tiempo de ejecución.

Las aplicaciones de Android normalmente están escritas en el lenguaje de programación orientado a objetos Java, aunque es posible utilizar frameworks para desarrollar aplicaciones en otros lenguajes, como HTML5+CSS3+Javascript o Python. Estas aplicaciones se ejecutan sobre una máquina virtual Dalvik, que está optimizada para requerir poca memoria y está diseñada para permitir ejecutar varias instancias de la máquina virtual simultáneamente, delegando en el sistema operativo subyacente el soporte de aislamiento de procesos, gestión de memoria e hilos.

Incluye una tienda de aplicaciones llamada Google Play Store, desarrollada por Google. Esta aplicación se encuentra instalada en la mayoría de los dispositivos Android y permite a los usuarios navegar y descargar aplicaciones publicadas por los desarrolladores.

Por otra parte, los usuarios pueden instalar aplicaciones desde otras tiendas virtuales (tales como Amazon AppStore o GetJar) o directamente en el dispositivo si se dispone del archivo APK de la aplicación.

% La compañía de investigación Canalys estimó en el segundo trimestre del año 2009 que Android tenía una cuota de mercado del 2,8\% de smartphones en todo el mundo. En el cuarto trimestre de 2010 esta cifra había aumentado al 33\% del mercado, convirtiéndose en la plataforma para smartphones más vendida. En el tercer trimestre de 2011 se estima que más de la mitad (52,5\%) del mercado de smartphones pertenece a Android. En el tercer trimestre de 2012 Android tenía una cuota del 75\% del mercado mundial de smartphones según la firma de investigación IDC. En la figura \ref{fig:ventas-android} puede verse la evolución de las ventas de teléfonos celulares según su sistema operativo.

% \begin{figure}[htbp]
%   \centering
%     %lo que agregué entre corchetes hace que el ancho de la imagen ocupe el 80% del área de texto. Si sacás eso la imagen no se redimensiona y se va de la hoja. Se puede usar algo parecido para limitar el alto si es necesario.
%     \includegraphics[width=0.8\textwidth]{imagenes/ventasAndroid.png}
%      \caption{Ventas de smartphones según su sistema operativo, a nivel mundial}
%     \label{fig:ventas-android}
% \end{figure}


% La cuota de Android en el mercado varía según la ubicación. En julio de 2012, la cuota de mercado de Android en los Estados Unidos fue de 52\%, y se eleva al 90\% en China.

\section{Desarrollo de Software}

Para resolver un problema, se sigue un Proceso de Resolución:

\begin{enumerate}
    \item Identificar el problema.
    \item Definir y Representar el problema.
    \item Explorar las posibles estrategias.
    \item Aplicar y mejorar las estrategias.
    \item Mirar atrás y Evaluar los efectos de la actividad realizada.
\end{enumerate}

El proceso de resolución de problemas software abarca:

\begin{enumerate}
    \item Análisis y especificación de requisitos (qué)
    \item Diseño del sistema Software (cómo)
    \item Codificación (realización del cómo)
    \item Pruebas
    \item Instalación (la solución debe ser usada)
\end{enumerate}

\section{Análisis y Diseño Orientado a Objetos}

El análisis se enfoca en la investigación de los problemas y los requisitos, más que en la solución. Es un término utilizado, por ejemplo, en el análisis de requisitos (una investigación de los requisitos) o análisis orientado a objetos (una investigación de los objetos de dominio).

Durante el análisis orientado a objetos, se trata de encontrar y describir los objetos o conceptos en el dominio del problema.

El diseño tiende a una solución conceptual (de software y hardware) que cumpla los requisitos, en lugar de su implementación. Los diseños pueden ser implementados, y la implementación (como ser el código) expresa el diseño realizado, verdadero y completo.
El término es muy utilizado, como en el diseño orientado a objetos o el diseño de bases de datos.

Durante el diseño orientado a objetos, se procura definir objetos de software, y cómo ellos colaboran para satisfacer los requisitos.

\section{Lenguaje de Modelado Unificado (UML)}

UML es un lenguaje estándar de diagramación. Puede ser usado para visualizar, especificar, construir y documentar los artefactos de un sistema de software.

Hay tres maneras de aplicar UML:

\begin{itemize}
    \item \emph{Como boceto:} Diagramas  informales e incompletos (frecuentemente dibujados en pizarras blancas) creados para explorar partes dificultosas del espacio del problema o de la solución, aprovechando el poder de los lenguajes visuales.
    \item \emph{Como plano:} Diagramas de diseño relativamente detallados, usados para:
        \begin{itemize}
            \item Ingeniería inversa para visualizar y entender mejor el código existente en diagramas UML.
            \item Generación de código. (ingeniería hacia adelante). Antes de programar, algunos diagramas detallados pueden proveer una guía para la generación de código (ej. Java), manualmente o automáticamente con una herramienta. 
        \end{itemize}
    \item \emph{Como lenguaje de programación:} Especificación ejecutable completa, de un sistema software en UML. El código ejecutable es automáticamente generado.
\end{itemize}

\textbf{En el presente proyecto, en general se procuró usar UML para la diagramación de bocetos, con un enfoque ágil, para la comunicación entre los miembros del equipo, y la resolución de distintos problemas, a lo largo de las disciplinas del proceso.
Sin embargo, algunos diagramas se realizaron con bastante detalle, utilizando herramientas CASE de UML  para la documentación de casos relevantes y para la posterior generación de código.}

Una misma notación puede ser usada para tres perspectivas y tipos de modelos:

\begin{itemize}
    \item \emph{Perspectiva conceptual:} Los diagramas son interpretados como cosas descriptas en una situación del mundo real o dominio de interés.
    \item \emph{Perspectiva de especificación (software):} Los diagramas describen abstracciones o componentes de software con especificaciones e interfaces, pero no obligadas a una implementación particular.
    \item \emph{Perspectiva de implementación (software)} Los diagramas describen implementaciones de software en una tecnología particular.
\end{itemize}

\textbf{Las tres perspectivas se aplicaron a lo largo del proceso de desarrollo del sistema.}

\section{Algunas definiciones}
\begin{itemize}
    \item \emph{Análisis Orientado a Objetos (AOO):} El AOO es un método de análisis que examina los requisitos desde la perspectiva de las clases y objetos que se encuentran en el vocabulario del dominio del problema.
 
    \item \emph{Diseño Orientado a Objetos (DOO):} El DOO es un método de diseño que abarca el proceso de descomposición orientada a objetos y una notación para describir los modelos lógico y físico, así como los modelos estático y dinámico del sistema que se diseña.

    \item \emph{Programación Orientada a Objetos (POO):} La POO es un método de implementación en el que los programas se organizan como colecciones cooperativas de objetos, cada uno de los cuales representa una instancia de alguna clase, y cuyas clases son, todas ellas, miembros de una jerarquía de clases unidas mediante relaciones de herencia.

    \item \emph{Proceso Software:} colección de actividades que comienza con la identificación de una necesidad y concluye con el retiro del software que satisface dicha necesidad.

    \item \emph{Ciclo de vida:} la transformación que el producto software sufre a lo largo de su vida, desde que nace (o se detecta una necesidad) hasta que muere (o se retira el sistema).

    \item \emph{Estados:} resultado de las transformaciones que sufre el producto software a lo largo de su ciclo de vida y representan en esencia el producto mismo.

    \item \emph{Modelo de ciclo de vida:} es la representación descriptiva y prescriptiva del ciclo de vida que indica el orden cronológico en el que deben llevar a cabo las actividades del proceso software. Un ciclo de vida debe determinar el orden de las fases del proceso software y establecer los criterios de transición para pasar de una fase a la siguiente.
 
    \item \emph{Disciplina:} Una colección de actividades relacionadas con un área de interés importante dentro del proyecto global.
    
\end{itemize}

\section{Ciclo de vida Iterativo Evolutivo}

\begin{quote}
“Deberías usar un desarrollo iterativo sólo en los proyectos que quieras que tengan éxito” - Martin Fowler
\end{quote}

El desarrollo iterativo evolutivo, en contraste con el ciclo de vida de cascada o secuencial, consiste en la programación y pruebas tempranas de un sistema parcial, en ciclos repetitivos. También normalmente supone que el desarrollo se inicia antes de que todos los requisitos se definan en detalle; el feedback se utiliza para aclarar y mejorar las características cambiantes.

Cuenta con pasos de desarrollo cortos y rápidos, feedback, y adaptación para aclarar los requisitos y el diseño, sin tanta especulación antes de la programación, como en método de cascada. La investigación demuestra que los métodos iterativos se asocian con mayores tasas de éxito y de productividad, y bajos niveles de defectos.

En este ciclo de vida, el desarrollo está organizado en una serie de mini-proyectos cortos y de longitud fija (timeboxing), llamados iteraciones; el resultado de cada uno es un sistema parcial probado, testeado e integrado. Cada iteración tiene sus propias actividades de análisis de requisitos, diseño, implementación y pruebas.

El ciclo de vida iterativo se basa en la ampliación y refinamiento sucesivos de un sistema a través de múltiples iteraciones, con retroalimentación cíclica y adaptación, como conductores principales para converger al sistema  adecuado. Como el sistema crece gradualmente con el tiempo, iteración por iteración, se lo llama desarrollo iterativo e incremental. Y debido a que la retroalimentación y la adaptación producen una evolución en las especificaciones y el diseño, también es conocido como desarrollo iterativo y evolutivo.

La salida de una iteración no es un prototipo experimental o desechable, y el desarrollo iterativo no es prototipado. Por el contrario, la salida es un subconjunto, en calidad de producción, del sistema final.

\subsection{Beneficios}
\begin{itemize}
    \item Menos fracasos de proyectos, mejor productividad y tasas de defectos reducidas; demostrado por investigación de métodos iterativos evolutivos.
    \item Mitigación temprana, más que tardía, de altos riesgos (técnicos, requisitos, objetivos, usabilidad, etc.)
    \item Progreso visible y temprano.
    \item Feedback, compromiso de usuario, y adaptación tempranos, llevando a un sistema refinado que satisface más estrechamente las necesidades reales de los clientes.
\end{itemize}

Según nuestra propia experiencia, una ventaja adicional es que nos permitió refinar además nuestras propias actividades, mejorando la productividad, estimación, diagramación y calidad del desarrollo en cada iteración.

\textbf{Siguiendo recomendaciones de metodologías ágiles, trabajamos con ciclo de vida iterativo evolutivo.}

\section{Proceso Unificado (UP)}

Un proceso de desarrollo de software describe un enfoque para la construcción, despliegue, y posiblemente mantenimiento del software.

El Proceso Unificado nació como un popular proceso de desarrollo iterativo para la construcción de sistemas orientados a objetos. Es un marco de desarrollo de software que se caracteriza por estar dirigido por casos de uso, centrado en la arquitectura y por ser iterativo e incremental.\\[5cm]

\begin{figure}[htbp]
  \centering
    %lo que agregué entre corchetes hace que el ancho de la imagen ocupe el 80% del área de texto. Si sacás eso la imagen no se redimensiona y se va de la hoja. Se puede usar algo parecido para limitar el alto si es necesario.
    \includegraphics[width=0.8\textwidth]{imagenes/rup}
        \hfill
	\caption{Disciplinas y fases del Proceso Unificado}
	\label{fig:rup}
\end{figure}

\subsection{Características}
    
\begin{itemize}
    \item \emph{Iterativo e Incremental:} Es un marco de desarrollo iterativo e incremental compuesto de cuatro fases denominadas Inicio, Elaboración, Construcción y Transición. Cada una de estas fases es a su vez dividida en una serie de iteraciones. Estas iteraciones ofrecen como resultado un incremento del producto desarrollado que añade o mejora las funcionalidades del sistema en desarrollo. Cada una de estas iteraciones se divide a su vez en una serie de disciplinas. Aunque todas las iteraciones suelen incluir trabajo en casi todas las disciplinas, el grado de esfuerzo dentro de cada una de ellas varía a lo largo del proyecto.
    \item \emph{Dirigido por los casos de uso:} Los casos de uso se utilizan para capturar los requisitos funcionales y para definir los contenidos de las iteraciones. La idea es que cada iteración tome un conjunto de casos de uso o escenarios y desarrolle todo el camino a través de las distintas disciplinas: diseño, implementación, prueba, etc.
    \item \emph{Centrado en la arquitectura:} Asume que no existe un modelo único que cubra todos los aspectos del sistema. Por dicho motivo existen múltiples modelos y vistas que definen la arquitectura de software de un sistema.
    \item \emph{Enfocado en los riesgos:} Requiere que el equipo del proyecto se centre en identificar los riesgos críticos en una etapa temprana del ciclo de vida. Los resultados de cada iteración, en especial los de la fase de Elaboración, deben ser seleccionados en un orden que asegure que los riesgos principales son considerados primero.
    \item \emph{UML:} Hace uso extensivo de diagramas UML para sus artefactos.
\end{itemize}

\textbf{Se trabajó sumando al proyecto, para cada iteración, un porcentaje (30\% aprox.) del total de casos de uso previstos.  A medida que se iba iterando se desarrollaban las disciplinas para estos nuevos casos de uso y se refinaban los casos de uso agregados en iteraciones anteriores, gracias a la experiencia adquirida y con la ayuda del feedback del usuario final.}





%% Disciplina de Requisitos
%%%%%%%%%%%%%%%%%%%%%%%%%%%%%%%%%%%%%%%%%%%%%%%%%%%%%%%%
%   |------------------------------------------|       %
%   | Web App embebida en dispositivos móviles |       %
%   |  para la gestión de registros sobre la   |       %
%   |   contaminación de afluentes y ríos.     |       %
%   |                                          |       %
%   |          Proyecto de graduación          |       %
%   |__________________________________________|       %
%                                                      %
%   Autores                                            %
%   -------                                            %
%                                                      %
% * Bruno, Ricardo Hugo (CX 1409686)                   %
%     rburnount@gmail.com                              %
% * Gómez Véliz, Kevin Shionen (CX 1411828)            % 
%     ing.gomezvelizkevin@gmail.com                    %
%                                                      %
%   Tutor                                              %
%   -------                                            %
%                                                      %
% * Ing. Cohen, Daniel Eduardo                         %
%        dcohen.tuc@gmail.com                          %
%                                                      %
%   Cotutor                                            %
%   -------                                            %
%                                                      %
% * Ing. Nieto, Luis Eduardo                           %
%        lnieto@herrera.unt.edu.ar                     %
%                                                      %
%                                                      %
%%%%%%%%%%%%%%%%%%%%%%%%%%%%%%%%%%%%%%%%%%%%%%%%%%%%%%%%

\chapter{Disciplina de Requisitos}
\label{chap:requisito}

  \section{Introducción}

    Esta especificación tiene como objetivo analizar y documentar las necesidades funcionales que deberán ser soportadas por el sistema a desarrollar. Para ello, se identificarán los requisitos que ha de satisfacer el nuevo sistema mediante entrevistas, el estudio de los problemas de las unidades afectadas y sus necesidades actuales. Además de identificar los requisitos se deberán establecer las prioridades, las cuales proporcionan un punto de referencia para validar el sistema final, que comprueben que se ajusten a las necesidades del usuario.

  \section{Identificación de usuarios participantes}

    Los objetivos de esta tarea son identificar a los responsables de cada una de las unidades y a los principales usuarios implicados. Para ello se consideran los siguientes aspectos:

    \begin{itemize}
      \item Incorporación de usuarios al equipo de proyecto.
      \item Conocimiento de los usuarios de las funciones a automatizar.
      \item Repercusión del nuevo sistema sobre las actividades actuales de los usuarios.
      \item Implicaciones legales del nuevo sistema.
    \end{itemize}

      Se identificaron los siguientes usuarios:

    \begin{itemize}
      \item \emph{Grupo de Administradores:} Formado por los solicitantes del software en cuestión.
      
      \item \emph{Grupo de Alumnos:} Formado principalmente por alumnos de escuelas/colegios que realizan muestras, las cuales generan registros en el sistema.
    \end{itemize}

    Es de destacar la necesidad de una participación activa de los usuarios del futuro sistema en las actividades de desarrollo del mismo, con objeto de conseguir la máxima adecuación del sistema a sus necesidades y facilitar el conocimiento paulatino de dicho sistema, permitiendo una rápida implantación.

  \section{Educción de requisitos}

    \subsection{Estudio de Documentación. Planificación y Realización de Entrevistas}

      Esta tarea tiene como finalidad capturar los requisitos de usuarios para el desarrollo del sistema.

      Para el análisis de requisitos se usaron distintas técnicas de educción de requisitos. Entre ellas el estudio de la documentación provista por parte del administrador; entrevistas abiertas y estructuradas, análisis del proceso actual.

  \section{Especificación de Requisitos de Software}

    \renewcommand{\thesubsection}{\arabic{subsection}}
    \subsection{Introducción}

      Este documento es una Especificación de Requisitos Software de la  Web App embebida en dispositivos móviles para la gestión de registros sobre la contaminación de afluentes y ríos. Esta documentación es fruto de las entrevistas, estudio de la documentación y del funcionamiento del proceso actual, así como del análisis llevado a cabo por el equipo de desarrollo.

      El objetivo de la especificación es definir en forma clara, precisa, completa y verificable todas las funcionalidades y restricciones del sistema que se desea construir.

      Esta documentación está sujeta a revisiones por el grupo de administradores que se recogerán por medio de sucesivas versiones del documento, hasta alcanzar la aprobación por parte de los mismos. Una vez aprobado, servirá de base al equipo de desarrollo para la construcción del sistema en cuestión.

      Esta especificación se ha realizado de acuerdo al estándar “IEEE Recomended Practice for software Requirements Specifications(IEEE/ANSI 830-1993)”.


    \subsection{Objetivos y alcance del sistema}

      El presente proyecto tiene como objetivo principal ayudar al medio ambiente, utilizando un sistema de gestión que permite la creación y administración de registros, los cuales contienen datos de muestras del universo de estudio que, al ser procesadas, brinda el estado de contaminación de un río o afluente, mediante indicadores biológicos.
      Estos registros cuentan con contenido multimedia y coordenadas geográficas

  \section{Definiciones, acrónimos y abreviaturas}

    \subsection{Definiciones}
      \begin{itemize}
        
        \item \emph{Salida de campo:} La principal aportación de la salida de campo es que permite al alumnado adquirir un aprendizaje significativo en el que el principal elemento del proceso de enseñanza-aprendizaje es la construcción de significados. La persona aprende un concepto, un fenómeno, un procedimiento, un comportamiento, etc.
        
        \item \emph{Insectos:} Nuestro universo de estudio obliga solo a tener en cuenta 4 insectos, los cuales sirven para indicar el posible grado de contaminación del agua. Estos insectos son los siguientes:
        
        \begin{itemize}
          \item Elimidos.
          \item Patudos.
          \item Plecopteros.
          \item Tricopteros.
        \end{itemize}
        
        \item \emph{Muestra}: Una muestra esta compuesta por los insectos encontrados en una salida de campo. 
        
        \item \emph{Indice de Contaminación}: Para el sistema, el indice de contaminación es el valor calculado, mediante la cantidad de diferentes insectos encontrados en el universo de estudio, de la siguiente manera:
        
        \begin{table}[H]
          \centering
          \begin{tabular}{|p{3.8cm}|l|l|}
            \hline
            \centering
            Cantidad de insectos encontrados  & Indice & Interpretación \\ \hline 
            0                     & 0 & Muy contaminado \\ \hline
            1                     & 1 & Contaminado \\ \hline
            2                     & 2 & Con contaminación media \\ \hline
            3                     & 3 & En buen estado \\ \hline
            4                     & 4 & En excelente estado \\ 
            \hline
          \end{tabular}
        \end{table}
        
        \item \emph{Foto paisaje}: Foto obtenida del paisaje en donde se realizo la muestra de los insectos encontrados, con el fin de facilitar un punto de referencia visual para próximas salidas de campo.
        
        \item \emph{Foto insectos}: Foto obtenida de la muestra que sirve para que los administradores de la aplicación validen, o no, el registro en cuestión.
        
        \item \emph{Coordenadas geográficas}: Se usan para referenciar, mediante latitud y longitud, el lugar en donde se realizo el registro.
        
        \item \emph{Mapa}: Mapa digital (Google Maps) en donde se muestran los registros realizados por los usuarios.
        
        \item \emph{HTML:} HyperText Markup Language. Lenguaje de marcado para la elaboración de paginas web. En nuestro caso, genera la vista final del sistema.
        
        \item \emph{BD:} Base de datos.
        
        \item \emph{CRUD:} Es el acrónimo de ``Crear, Leer, Actualizar y Borrar'' (del original en ingles: Create, Read, Update and Delete), que se usa para referirse a las funciones básicas en bases de datos.
        
        \item \emph{MCVS:} Modelo de Ciclo de Vida del Sistema.
      \end{itemize}

  \section{Descripción general}

    Esta sección nos presenta una descripción general del sistema con el fin de conocer las funciones que debe soportar, los datos asociados, las restricciones impuestas y cualquier otro factor que pueda influir en la construcción del mismo.
    El sistema nace como necesidad de un grupo de docentes de la Facultad de Ciencias Naturales e Instituto Miguel Lillo, por llevar un control mas exhaustivo de su investigación.
    Es prioritario el seguimiento de la contaminación de los afluentes y ríos ubicados en la provincia de Tucumán.

    \subsection{Proceso actual}

      Este seguimiento se realizaba mediante salidas de campos con alumnos de las escuelas rurales donde realizaban las siguientes actividades:
      
      \begin{itemize}
        \item Los alumnos obtienen muestras del río o afluente intentando capturar algunos de los insectos del universo de estudio.
        
        \item Los docentes a cargo verifican dichas muestras, identificando las coincidencias, obteniendo de esta forma, un indice de contaminación.  
      \end{itemize}
      
      Todo esto se realiza de manera manual, anotando en papel y luego es transcripto a una planilla Excel.
      Todo el procedimiento antes descripto, dificulta la trazabilidad y administración de la información (introduciendo errores y demoras por el manejo manual de la información), por lo que todo esto sería más efectivo y sencillo con la ayuda de la tecnología. \newpage

    \subsection{Reingeniería de proceso}

      El objetivo de este proyecto se basa en proporcionar facilidades al proceso actual de la siguiente forma: 

      \begin{itemize}
        \item Dos imágenes capturadas con la cámara de fotos del Smartphone:
        
        \begin{itemize}
          \item La primera imagen será una foto de los insectos encontrados en un río o afluente. El objetivo es encontrar 4 insectos diferentes para analizar la biodiversidad.
         
          \item La segunda imagen será una foto del paisaje que servirá como un futuro punto de referencia para próximas salidas de campo. 
        \end{itemize}
        
        \item Capturar coordenadas de manera automática con una precisión propia al GPS integrado del Smartphone, las cuales se guardarán como \emph{latitud} y \emph{longitud}
        
        \item El usuario deberá seleccionar, según su criterio personal, cuales de los 4 insectos fueron encontrados por él, mediante un formulario interactivo, el cual consta de imágenes reales de los mismos para una buena comparación y un campo de observaciones para realizar comentarios subjetivos sobre la muestra en cuestión.
        
        \item Al completar toda la información mencionada anteriormente, se visualizará una ruleta virtual animada, la cual mostrará un valor (indice de contaminación), indicando el posible grado contaminación del agua en donde se realizo la muestra.

        \item Lo anterior se realizará sin conexión a internet (2G, 3G, Wifi, etc), generando un registro de manera local, que luego, de manera automática, se subirá a los servidores al momento de adquirir alguna conexión a internet.

        \item Los administradores podrán gestionar, mediante un navegador web (Google Chrome, Internet Explorer, Mozilla, etc) o desde la aplicación en el Smartphone, los registros previamente creados y guardados en el servidor.

        \item Analizando todos los registros, se creará un renderizado de un mapa  (Google Maps) en donde los administradores y usuarios podrán visualizar, con trazos de diferentes colores, el grado de contaminación del agua en el curso de los ríos o afluentes analizados en las salidas de campo.
      \end{itemize}

      El sistema debe ser seguro, escalable, de fácil mantenimiento y muy simple de usar utilizando solo interfaces táctiles para los usuarios finales. El futuro sistema llevará el nombre \emph{Agüita}.

      Las funciones que debe realizar el sistema se pueden agrupar de la siguiente manera:

      \begin{itemize}

        \item \emph{Gestión de usuarios:} Debe permitir gestionar los usuarios (CRUD). Los mismos pueden ser usuarios o administradores del sistema.
        Para que un usuario pueda generar un registro, deberá estar previamente registrado e iniciar sesión por una única vez en su Smartphone.
        Los usuarios deben poder configurar/modificar las opciones de su perfil como ser: nombre, apellido, lugar de residencia, institución a la que pertenece y grado correspondiente.
        Los usuarios no pueden consultar la información personal de otros usuarios.

        \item \emph{Gestión de registros:} Debe permitir gestionar los registros (CRUD). Los usuarios generar registros. Los administradores pueden interactuar con los mismos, cambiando su estado (pendiente - valido - invalido) según la información brindada por los mismos (no verificada por los administradores - correcta - incorrecta) respectivamente.

        \item \emph{Consultas de registros realizados:} Los usuarios podrán consultar un listado de sus registros creados con la información completa.
        Los usuarios no podrán consultar los registros de otros usuarios.
        Los administradores podrán ver los registros creados por los todos los usuarios de la aplicación con toda su información correspondiente.

        \item \emph{Consulta de mapa:} Los usuarios y administradores pueden ver el mapa final con toda la información recopilada de todos los registros creados por los mismos.
        Todos los registros que estén en un estado \emph{rechazado} no se eliminaran de la base de datos, no obstante, los mismos no se tomaran en cuenta para el renderizado del mapa final. 

      \end{itemize}
      \newpage 
  \section{Requisitos específicos}

    \begin{enumerate}[A.]
      \item \textbf{Gestión de Usuarios}
        \begin{itemize}  
          \item \textbf{Alta de usuarios:}
            \\ \textbf{Introducción:} El sistema permite introducir información sobre usuarios en la aplicación.
            \\ \textbf{Entrada:} IdUsuario + Email + Usuario + Contraseña + Nombre + Apellido + Institución + Grado + Residencia + Rol + Foto Perfil + Estado 
            \\ \textbf{Proceso:} El sistema comprueba la inexistencia previa de un usuario, buscando coincidencias en el nombre de usuario y mail. En caso de no encontrar un usuario con los nombre de usuario y mail especificados, se creará y guardará el nuevo usuario, al cual se le asignará el número único IdUsuario (este número se obtiene a partir del máximo existente hasta el momento, siendo 1 para el primer usuario). En este caso, se devolverá un mensaje de éxito y el IdUsuario. En caso que ya existiera un usuario con los nombre de usuario y mail especificados, se devolverá un mensaje de error informando tal situación
            \\ \textbf{Salida:} @IdUsuario + mensaje
            \\
          \item \textbf{Modificación de usuarios:}
            \\ \textbf{Introducción:} El sistema permite modificar información sobre usuarios existentes en la aplicación.
            \\ \textbf{Entrada:} @IdUsuario + Nombre + Apellido + Institución + Grado + Residencia + Foto Perfil
            \\ \textbf{Proceso:} El sistema comprueba la existencia previa de un usuario en base a @IdUsuario y actualiza la información del mismo. En caso de éxito, se devolverá un mensaje de éxito y el IdUsuario. En caso de error se devolverá un mensaje con el motivo del mismo.
            \\ \textbf{Salida:} @IdUsuario + mensaje
            \\
          \item \textbf{Cambiar estado de usuarios:}
            \\ \textbf{Introducción:} El sistema permite habilitar/inhabilitar usuarios existentes en la aplicación.
            \\ \textbf{Entrada:} @IdUsuario + Estado Actual
            \\ \textbf{Proceso:} El sistema comprueba la existencia previa de un usuario en base a @IdUsuario para poder modificar su estado. Hay dos tipos de estado: ``Habilitado" - ``Inhabilitado". Este proceso permuta el estado actual del usuario. Si un usuario tiene estado ``Habilitado", este se cambia a ``Inhabilitado" y viceversa. Los registros asociados al usuario no deben eliminarse ni modificarse. En caso de error se devolverá un mensaje con el motivo del mismo.
            \\ \textbf{Salida:} @IdUsuario + Mensaje
            \\
          \item \textbf{Ver detalles de usuario:}
            \\ \textbf{Introducción:} El sistema permite ver detalles relacionados a los usuarios existentes en él. Se debe visualizar usuario, nombre, apellido, institución, grado, residencia, foto de perfil y cantidad de registros generados
            \\ \textbf{Entrada:} @IdUsuario
            \\ \textbf{Proceso:} El sistema comprueba la existencia previa del usuario en base a @IdUsuario. En caso de éxito, se presenta la información del mismo. En caso de error se devolverá un mensaje con el motivo del mismo.
            \\ \textbf{Salida:} Usuario + Nombre + Apellido + Institución + Grado + Residencia + Email + Foto Perfil + Cantidad Registros
            \\
          \item \textbf{Cambiar contraseña de usuario:}
            \\ \textbf{Introducción:} El sistema permite cambiar la contraseña a los usuarios existentes en él.
            \\ \textbf{Entrada:} @IdUsuario + Contraseña Anterior + Contraseña Nueva
            \\ \textbf{Proceso:} El sistema comprueba la existencia previa del usuario en base a @IdUsuario, luego se impactara la nueva contraseña, dejando en desuso la anterior. En caso de éxito, se presenta la información del mismo. En caso de error se devolverá un mensaje con el motivo del mismo.
            \\ \textbf{Salida:} @IdUsuario + Mensaje
            \\
          \item \textbf{Cambiar rol de usuario:}
            \\ \textbf{Introducción:} El sistema permite cambiar el Rol a los usuarios existentes en él. Esta acción solo la pueden realizar los usuarios del grupo ``Administradores"
            \\ \textbf{Entrada:} @IdUsuario + Rol
            \\ \textbf{Proceso:} El sistema comprueba la existencia previa del usuario en base a @IdUsuario. Hay dos tipos de roles: ``Administrador" - ``Usuario". Este proceso modificará el Rol del mismo según valor de entrada. En caso de éxito, se presenta la información del mismo. En caso de error se devolverá un mensaje con el motivo del mismo.
            \\ \textbf{Salida:} @IdUsuario + Mensaje
            \\
          \item \textbf{Búsqueda de usuarios:}
            \\ \textbf{Introducción:} El sistema permite introducir parámetros con los que se buscará usuarios que coincidan con los mismos.
            \\ \textbf{Entrada:} Usuario o Nombre o Apellido o Email
            \\ \textbf{Proceso:} El sistema lista al usuario que cumpla con los parámetros de búsqueda en caso de coincidencia. En caso de no encontrar algún usuario, se mostrará un mensaje vacío, indicando que la búsqueda no arrojo resultados.
            \\ \textbf{Salida:} Usuario + Nombre + Apellido + Institución + Grado + Residencia + Email + Foto Perfil + Cantidad Registros
            \\
        \end{itemize}

      \item \textbf{Gestión de Registros}
        \begin{itemize}
          \item \textbf{Alta de registros:}
            \\ \textbf{Introducción:} El sistema permite dar de alta un nuevo registro ingresando indice, insectos encontrados, fecha, latitud, longitud, foto del paisaje, foto de la muestra, foto del mapa (vista aérea con un PIN indicando la ubicación terrestre), observaciones del usuario, IdUsuario (creador del registro), IdUbicacion (país, localidad, provincia).
            \\ \textbf{Entrada:} IdRegistro + Indice + Insectos Encontrados + Fecha Creación + Latitud + Longitud + Foto Paisaje + Foto Muestra + Foto Mapa + Observaciones Usuario + IdUsuario + IdUbicación
            \\ \textbf{Proceso:} El sistema crea un registro al cual se le asignará el número único IdRegistro (este número se obtiene a partir del máximo existente hasta el momento, siendo 1 para el primer registro). Los estados de validación posibles son ``Valido'', ``Invalido'' y ``Pendiente''. Un registro nuevo se creará con un valor de estado de validación igual a ``Pendiente''.  y con observaciones del administrador sin contenido, además, las fotos se almacenarán en Base64. En caso de éxito, se devolverá un mensaje de éxito y el IdRegistro.
            \\ \textbf{Salida:} IdRegistro + Mensaje
            \\
          \item \textbf{Cambiar estado de registros:}
            \\ \textbf{Introducción:} El sistema permite cambiar el estado de un registro. Solo los administradores del sistema tendrán los permisos para realizar esta acción. Se podrá cambiar el estado del registro ``Valido'' / ``Invalido'' y viceversa. Inicialmente, el registro se crea con un estado ``Pendiente", el cual, una vez modificado, no se podrá volver a asignar.
            \\ \textbf{Entrada:} @IdRegistro + Estado Validación
            \\ \textbf{Proceso:} El sistema modifica el registro con el valor de estado validación correspondiente. En caso de éxito, se devolverá un mensaje de éxito y el IdRegistro.
            \\ \textbf{Salida:} IdRegistro + Mensaje
            \\
          \item \textbf{Asignar observaciones de administrador a registros:}
            \\ \textbf{Introducción:} El sistema permite agregar observaciones de administrador al registro. Solo los administradores del sistema tendrán los permisos para realizar esta acción.
            \\ \textbf{Entrada:} @IdRegistro + Observaciones Administrador
            \\ \textbf{Proceso:} El sistema modifica el registro agregando una observación de administrador. En caso de éxito, se devolverá un mensaje de éxito y el IdRegistro.
            \\ \textbf{Salida:} IdRegistro + Mensaje
            \\
          \item \textbf{Ver detalles de registro:}
            \\ \textbf{Introducción:} El sistema permite ver detalles relacionados a los registros existentes en él. Se debe visualizar indice, insectos encontrados, fecha, latitud, longitud, foto del paisaje, foto de la muestra, foto del mapa, observaciones del usuario, estado de validación, usuario que lo creó, país, provincia, localidad.
            \\ \textbf{Entrada:} @IdRegistro
            \\ \textbf{Proceso:} El sistema comprueba la existencia previa del registro en base a @IdRegistro. En caso de éxito, se presenta la información del mismo. En caso de error se devolverá un mensaje con el motivo del mismo.
            \\ \textbf{Salida:} Indice + Insectos Encontrados + Fecha Creación + Latitud + Longitud + Foto Paisaje + Foto Muestra + Foto Mapa + Observaciones Usuario + Observaciones Administrador + Estado Validación + IdUsuario + IdUbicación
            \\
          \item \textbf{Búsqueda de registros}
            \\ \textbf{Introducción:} El sistema permite buscar registros filtrando por los registros creados desde un intervalo de fechas, por estado de validación, por indice.
            \\ \textbf{Entrada:} Fecha Inicio Fecha Fin + Estado Validación + Indice
            \\ \textbf{Proceso:} El sistema lista los registros que cumplan con los parámetros de búsqueda en caso de coincidencia. En caso de no encontrar algún registro, se mostrará un mensaje vacío, indicando que la búsqueda no arrojo resultados. Si la búsqueda no contiene parámetros, se listan todos los registros existentes en el sistema
            \\ \textbf{Salida:} Indice + Insectos Encontrados + Fecha Creación + Latitud + Longitud + Observaciones Usuario + Observaciones Administrador + Estado Validación + IdUsuario + IdUbicación
        \end{itemize}

      \item \textbf{Gestión de Mapas}
        \begin{itemize}
          \item \textbf{Ver mapa}
          \\ \textbf{Introducción:} El sistema permite ver el mapa interactivo con la información de los registros almacenados que cumplan con la condición de estado de validación igual a "Valido".
          \\ \textbf{Entrada:} Arreglo de Registros 
          \\ \textbf{Proceso:} Mostrar un mapa con puntos obtenidos mediante la latitud y longitud de cada registros del arreglo.
          Los puntos indican, ademas de la posición geográfica del registro, el usuario que lo creó y el indice de contaminación numéricamente y ademas con un color de entre 4 diferentes para una rápida identificación visual.
          \\ \textbf{Salida:} Mapa + Puntos Geográficos
        \end{itemize}

    \end{enumerate}

    \subsection{Suposiciones y Dependencias}
      \begin{itemize}
        \item \textbf{Suposiciones:} Se asume que los requisitos en este documento son estables una vez que sean aprobados por los responsables de la aplicación. Cualquier petición de cambios en la especificación debe ser aprobada por todas las partes intervinientes y será gestionada por el equipo de desarrollo.
        \item \textbf{Dependencias:} El sistema trabaja en conjunto con Google Maps y el sistema de posicionamiento global mediante satélites, algún cambio que se realicen en estos, el sistema podría presentar inconsistencias, errores, y hasta dejar de funcionar.
      \end{itemize}

    \subsection{Requisitos de Usuario y Tecnológicos}
      \begin{itemize}
        \item \textbf{Requisitos de usuario:} Como se mencionó anteriormente, se identifican dos tipos de usuarios: Administradores y Alumnos. Los usuarios tendrán sus cuentas asociadas con Nombre de Usuario y Contraseña. Los mismos podrán iniciar sesión desde computadoras de escritorio (Sistema de administración y gestión de registros) o mediante la aplicación para dispositivos móviles Smartphones, siempre y cuando, dispongan de una cuenta valida. En caso contrario, deberán registrarse en el sistema mediante la aplicación correspondiente.

        \item \textbf{Requisitos tecnológicos:} Los administradores podrán iniciar sesión en el sistema mediante computadoras de escritorio, facilitando la gestión y administración del mismo. Ésta versión WEB restringe el uso a aquellos usuarios del grupo ``Alumnos", por otra parte, los alumnos podrán hacer uso del sistema solo para generar registros, y ver el mapa interactivo mediante la aplicación para Smartphones que deberán descargar la tienda.
        Se utilizará una plataforma de servicios en la nube o un servidor físico local provisto por el cliente de este sistema, para administrar una máquina virtual que hará de servidor web y servidor de base de datos. El sistema operativo que correrá la máquina virtual será Ubuntu Server, éste es un sistema operativo gratuito y de código abierto, por otra parte, el sistema gestor de base de datos será del tipo MySQL. Ambos sistemas (operativo y gestor de base de datos), serán instalados en su ultima versión estable al momento de la entrega del software.
        El sistema se ejecutara sobre un esquema de peticiones Cliente/Servidor (API Rest). La elección esta infraestructura se debe principalmente a 3 motivos: 
        \begin{itemize}
          \item Debido a que el sistema se puede usar mediante Smartphones y computadoras de escritorio de forma remota, la solución fue dividir y separar, como ya se menciono, la aplicación para los usuarios de el servidor de consultas y base de datos.
          \item Experiencia del equipo de desarrollo 
          \item Sistemas seleccionados de licencia gratuita.
        \end{itemize}
      \end{itemize}

    \subsection{Requisitos de Interfaces Externas}
      \begin{itemize}
        \item \textbf{Interfaz de usuario:} Las interfaces de la aplicación deben ser intuitivas, fáciles de usar, amigables y de respuesta rápida. La interfaz de usuario debe ser orientada al uso táctil de los Smartphones.
        \item \textbf{Interfaz Hardware:} 
          \begin{itemize}
            \item Requisitos para los Smartphones:
              \begin{itemize}
                \item Los Smartphones de los usuarios que ejecutaran la aplicación deberán tener las siguientes características independientemente de su S.O:
                  \begin{itemize}
                    \item Cámara fotográfica de 1 Mega Pixeles o más.
                    \item GPS integrado.
                    \item Conexión a internet vía WiFi o Red GSM
                    \item Pantalla táctil de 3.5'' o superior.
                    \item Espacio disponible de 10 MB para la instalación de la aplicación + Cantidad de MB variable ocupado por cada registro creado. 
                  \end{itemize}
              \end{itemize}
            \item Requisitos mínimos para el servidor:
            \begin{itemize}
              \item Procesador AMD Sempron 3000 o equivalente o procesador Intel Celeron o equivalente. Capacidad de virtualización
              \item 1 GB de memoria RAM.
              \item Conexión a internet.
            \end{itemize}
            \item Requisitos mínimos para las PC o notebook de los administradores
            \begin{itemize}
              \item Procesador AMD Sempron 3000 o equivalente. Procesador Intel Celeron o equivalente.
              \item 1 GB de memoria RAM.
              \item Periféricos de entrada/salida.
              \item Conexión a internet
            \end{itemize}
          \end{itemize}
      
        \item \textbf{Interfaz Software:} 
          Sistemas operativos soportados por la aplicación:
          \begin{itemize}
            \item Smartphones 
            \begin{itemize}
              \item Android 4.0 o posterior.
              \item iOS 9.0 o posterior.
            \end{itemize}
            \item Servidor 
            \begin{itemize}
              \item  El sistema operativo será Ubuntu Server en su ultima versión estable LTS. 
            \end{itemize}
            \item PC o notebook de los administradores 
            \begin{itemize}
              \item Cualquier sistema operativo con navegador web 
            \end{itemize} 
          \end{itemize}
      \end{itemize}

    \subsection{Requisitos de Rendimiento}

      El Tiempo de respuesta de la aplicación de cada función solicitada por el usuario no debe ser superior a los 3 segundos en una velocidad efectiva de conexión con el servidor a través de 3G.
      No obstante, los registros al generarse de manera offline (sin conexión a internet), se guardarán de manera local, por lo que si se generaron varios registros, al momento de que el Smartphone detecte conexión a internet, el tiempo de respuesta se ve afectado de manera directamente proporcional a la cantidad de registros que se estén subiendo al servidor en la nube en ese momento.

    \subsection{Requisitos de Desarrollo y Restricciones de Diseño}

      El ciclo de vida será Prototipado Evolutivo, debiendo orientarse hacia el desarrollo de un sistema flexible que permita incorporar de manera sencilla cambios y nuevas funcionalidades.

    \subsection{Ajuste a estándares} Interfaz de usuario basada Material Desing (Google)

    \subsection{Seguridad} 
      \begin{itemize}
        
        \item \textbf{En desarrollo:} Los desarrolladores acceden a la gestión del sistema operativo y/o sus aplicaciones a través de Secure Shell o SSH, el estándar de facto para la administración remota de servidores de manera segura. Tanto para la administración propia del servidor, como en las aplicaciones que se utilizan para administrar la base de datos (MySQL Workbench) y la gestión de archivos (SFTP), se realizan con clientes que establecen conexiones seguras con el servidor. La técnica empleada para dichas conexiones es el intercambio de claves públicas y validación con la clave privada de las aplicaciones cliente.
        \item \textbf{En producción:} Los usuarios del sistema acceden al mismo mediante la aplicación móvil o cualquier explorador web. Para operar con la aplicación, deben proveer un usuario y una contraseña. Esta información es procesada en el servidor en un proceso de validación de credenciales y devuelve al cliente el mensaje de inicio de sesión correcto. La comunicación entre cliente servidor viajara encriptada mediante el protocolo \gls{HTTPS}.
        \item \textbf{Roles y permisos:} Para reforzar la seguridad de la aplicación, cada usuario posee un rol (``Administrador'' o ``Alumno''), el cual tiene asociados diferentes permisos. Esto permite restringir el acceso a usuarios del grupo ``Alumno'' a la aplicación WEB de gestión y administración de registros. El rol ``Administrador'' tiene todos los permisos y privilegios, pudiendo así, gestionar el sistema e incluso generar registros como lo harían los Alumnos.
        \item \textbf{Red:} En la capa de red se crearon reglas de acceso al servidor mediante el uso del firewall provisto por el Sistema Operativo Ubuntu. A través de estas reglas se establece que servicios podrá ofrecer el servidor a los clientes de la aplicación.
        \item \textbf{Sistema Operativo:} En el sistema operativo se definen los usuarios y los permisos que poseen para realizar lecturas, escrituras y/o ejecuciones de archivos alojados en la memoria del servidor. De esta forma se previene que usuarios no autorizados puedan modificar, eliminar y ejecutar archivos en el servidor, incluso en la base de datos.
        \item \textbf{Servidor web:} A nivel servidor web, se prevé la implementación del protocolo de aplicación HTTPS que permite encriptar el trafico de información desde el servidor web hacia los navegadores o aplicaciones que realicen solicitudes, estableciendo un eslabón mas en la seguridad del sistema.
        \item \textbf{Política de respaldo:} El administrador llevará a cabo un respaldo de datos en discos externos o en la nube por el tiempo que el considere necesario. Ademas, se exportaran los registros en un archivo excel manteniendo la información necesaria para la recuperación de los registros a futuro.
        Por otro lado, el motor de Base de datos estará configurado para realizar backups cada cierto intervalo de tiempo definido por el administrador. Conservar los respectivos archivos de respaldo de los últimos 6 backups.
        \item \textbf{Política de Borrado:} No se ha definido
      \end{itemize}
      
  \section{Estimación del proyecto}

    \subsection{Planificación de etapas} 
      \begin{table}[H]
        \centering
        \begin{tabular}{|l|l|l|l|}
          \hline
          \centering
          Código  & Descripción  & Fecha Inicio & Fecha Fin \\ \hline
          A       & Estudio de factibilidad y acciones preliminares & 15/01/2017 & 25/01/2017 \\ \hline
          B       & Educción de requisitos & 01/02/2017 & 01/03/2017 \\ \hline
          C       & Diseño del prototipo & 02/03/2017 & 01/04/2017 \\ \hline
          D       & Corroborar diseño con requisitos & 01/04/2017 & 05/04/2017 \\ \hline
          E       & Desarrollo del prototipo & 10/04/2017 & 15/09/2017\\ \hline
          F       & Pruebas del prototipo & 15/09/2017 & 15/10/2017 \\ \hline
          G       & Refinamiento del prototipo & 15/11/2017 & 01/02/2018 \\ \hline
          H       & Análisis y evaluación el prototipo por parte del cliente & 15/02/2018 & 16/02/2018 \\ \hline
          I       & Refinamiento del prototipo & 01/03/2018 & 01/06/2018 \\ \hline
          J       & Entrega para producción & 15/09/2018 & 20/09/2018 \\ \hline
          K       & Seguimiento del sistema & 20/09/2018 & - \\ \hline
        \end{tabular}
        \caption {Fechas tentativas por etapa.}
      \end{table}

    \subsection{Duración estimada de tareas} 
      \begin{table}[H]
        \centering
        \begin{tabular}{|l|l|l|}
          \hline
          \centering
          Etapas  & Semanas  & Valor \% \\ \hline
          Estudio de factibilidad y acciones preliminares & 6 & 9\% \\ \hline
          Educcion de requisitos & 4 & 6\% \\ \hline
          Diseño del prototipo & 5 & 8\% \\ \hline
          Corroborar diseño con requisitos & 1 & 1\% \\ \hline
          Desarrollo del prototipo & 21 & 33\% \\ \hline
          Pruebas del prototipo & 4 & 6\% \\ \hline
          Refinamiento del prototipo & 11 & 17\% \\ \hline
          Análisis y evaluación el prototipo por parte del cliente & 1 & 1\% \\ \hline
          Refinamiento del prototipo & 12 & 18\% \\ \hline
          Entrega para producción & 1 & 1\% \\ \hline
          Seguimiento del sistema & - & - \\ \hline
          \textbf{Total} & 66 & 100\% \\ \hline
        \end{tabular}
        \caption {Estimación en semanas - Valor porcentual}
      \end{table}

    \subsection{Diagrama de Gantt} 

    \begin{figure}[htbp]
      \centering
        \includegraphics[width=1\textwidth]{imagenes/gantt.png}
    \end{figure} 

%% Disciplina de Análisis
%%%%%%%%%%%%%%%%%%%%%%%%%%%%%%%%%%%%%%%%%%%%%%%%%%%%%%%%
%   |------------------------------------------|       %
%   | Web App embebida en dispositivos móviles |       %
%   |  para la gestión de registros sobre la   |       %
%   |   contaminación de afluentes y ríos.     |       %
%   |                                          |       %
%   |          Proyecto de graduación          |       %
%   |__________________________________________|       %
%                                                      %
%   Autores                                            %
%   -------                                            %
%                                                      %
% * Bruno, Ricardo Hugo (CX 1409686)                   %
%     rburnount@gmail.com                              %
% * Gomez Veliz, Kevin Shionen (CX 1411828)            %
%     ing.gomezvelizkevin@gmail.com                    %
%                                                      %
%   Tutor                                              %
%   -------                                            %
%                                                      %
% * Ing. Cohen, Daniel Eduardo                         %
%        dcohen.tuc@gmail.com                          %
%                                                      %
%   Cotutor                                            %
%   -------                                            %
%                                                      %
% * Ing. Nieto, Luis Eduardo                           %
%        lnieto@herrera.unt.edu.ar                     %
%                                                      %
%                                                      %
%%%%%%%%%%%%%%%%%%%%%%%%%%%%%%%%%%%%%%%%%%%%%%%%%%%%%%%%

\chapter{Disciplina de Análisis}
	\label{chap:analisis}
	\section{Vista de Casos de Uso}
		La vista de casos de uso captura el comportamiento de un sistema, subsistema, clase o componente, como lo ve un usuario externo. Particiona la funcionalidad del sistema en transacciones significativas para los actores (usuarios idealizados) de un sistema. 
		Las piezas de funcionalidad interactiva son llamadas ``casos de uso''. Un caso de uso describe una interacción entre actores como una secuencia de mensajes entre el sistema y uno o más actores. El término \emph{actor} incluye a personas, como también otros sistemas de computadora o procesos.

	\section{Diagramas de Casos de Uso}
		Primeramente se muestra un diagrama de caso de uso general donde se agruparon los casos de uso por las acciones en común, luego se va a explorar cada caso de uso de manera mas descriptiva.
		Por ejemplo, en ``Gestión de Registros'' van a estar todos los casos de uso referidos a los mismos (alta, baja, listar, buscar, etc)

		\subsection{Diagrama General de Casos de Uso}
			\begin{figure}[H]
			\centering
				
				\includegraphics[width=1\textwidth]{imagenes/DiagramasUML/CasoGeneralDeUso.png}
					%%Me parece que queda mejor sin el hfill
					%\hfill
				%\caption{epígrafe}
				\label{fig:casos-de-uso}
			\end{figure}

		\subsection{Gestión de Usuarios}
			\begin{figure}[H]
			\centering
				\includegraphics[width=0.7\textwidth]{imagenes/DiagramasUML/gestionDeUsuarios.png}
					%%Me parece que queda mejor sin el hfill
					%\hfill
				%\caption{epígrafe}
				\label{fig:casos-de-uso-usuario}
			\end{figure}

		\subsection{Gestión de Registros}
			\begin{figure}[H]
			\centering
				\includegraphics[width=0.7\textwidth]{imagenes/DiagramasUML/gestionDeRegistros.png}
					%%Me parece que queda mejor sin el hfill
					%\hfill
				%\caption{epígrafe}
				\label{fig:casos-de-uso-tienda}
			\end{figure}

		\subsection{Casos de Uso Principales}
			\begin{enumerate}[CU1: ]
				\itemsep-1em
				\item Un alumno no registrado desea registrarse en el sistema.
				\item Un alumno registrado desea iniciar sesión.
				\item Un alumno registrado desea crear un registro.
				\item Un alumno registrado desea listar sus registros.
				\item Un alumno registrado desea ver un registro en particular.
				\item Un alumno registrado desea ver el mapa general.
				\item Un alumno registrado desea ver su perfil.
				\item Un alumno registrado desea modificar su perfil.
				\item Un alumno registrado desea cerrar sesión.
				\item Un alumno registrado desea eliminar su cuenta.
				\item Un administrador desea iniciar sesión.
				\item Un administrador desea borrar o dar de baja un alumno.
				\item Un administrador desea listar los alumnos.
				\item Un administrador desea listar los registros.
				\item Un administrador desea ver un registro en particular.
				\item Un administrador desea aprobar o rechazar registros.
				\item Un administrador desea buscar registros por fecha de creación.
				\item Un administrador desea buscar registros por institución.
				\item Un administrador desea buscar registros por indice.
				\item Un administrador desea ver el mapa general.
				\item Un administrador desea exportar a Excel la información.
				\item Un administrador desea cerrar sesión.
			\end{enumerate}


	

%% Disciplina de Diseño
%%%%%%%%%%%%%%%%%%%%%%%%%%%%%%%%%%%%%%%%%%%%%%%%%%%%%%%%
%   |------------------------------------------|       %
%   | Web App embebida en dispositivos móviles |       %
%   |  para la gestión de registros sobre la   |       %
%   |   contaminación de afluentes y ríos.     |       %
%   |                                          |       %
%   |          Proyecto de graduación          |       %
%   |__________________________________________|       %
%                                                      %
%   Autores                                            %
%   -------                                            %
%                                                      %
% * Bruno, Ricardo Hugo (CX 1409686)                   %
%     rburnount@gmail.com                              %
% * Gomez Veliz, Kevin Shionen (CX 1411828)            %
%     ing.gomezvelizkevin@gmail.com                    %
%                                                      %
%   Tutor                                              %
%   -------                                            %
%                                                      %
% * Ing. Cohen, Daniel Eduardo                         %
%        dcohen.tuc@gmail.com                          %
%                                                      %
%   Cotutor                                            %
%   -------                                            %
%                                                      %
% * Ing. Nieto, Luis Eduardo                           %
%        lnieto@herrera.unt.edu.ar                     %
%                                                      %
%                                                      %
%%%%%%%%%%%%%%%%%%%%%%%%%%%%%%%%%%%%%%%%%%%%%%%%%%%%%%%%

\chapter{Disciplina de Diseño}
	\label{chap:disenio}
	Por el Principio de Pareto\footnote{Cuando se habla de los costes de desarrollo de software enunciarse de la siguiente manera: ``El 80\% del esfuerzo de desarrollo (en tiempo y recursos) produce el 20\% del código, mientras que el 80\% restante es producido con tan sólo un 20\% del esfuerzo''}, se hicieron los diagramas relevantes. Como se procura tener homogeneidad en la implementación de las clases, basta un solo diagrama de cada tipo para una clase, para describir también el comportamiento de las otras clases.

	\section{Descripción Textual de Casos de Uso}
		%#############################################################################
		%#   
		%#   Caso de uso 1
		%#
		%#############################################################################
		\subsection{Caso de Uso 01: Un usuario no registrado desea registrarse en el sistema.}
			\begin{longtable}{|l|p{5.5cm}|l|p{2cm}|l|p{1.9cm}|} \hline
					\cellcolor{grisOscuro} CU01 & \multicolumn{4}{|l|}{  \cellcolor{grisOscuro} Registrar} &  \cellcolor{grisClaro}\multirow{2}{1cm}{} \\ \cline{1-5}
					\cellcolor{grisOscuro} Revisa: &  \cellcolor{grisClaro} &  \cellcolor{grisOscuro} Fecha &  \cellcolor{grisClaro} &  \cellcolor{grisOscuro} Firma: & \cellcolor{grisClaro} \\ \hline
					\multicolumn{6}{|p{15cm}|}{ \textbf{Resumen: } Este caso de uso permite a los usuarios registrarse como usuarios de la aplicación, permitiendo introducir sus datos personales

					} \\ \hline

					\multicolumn{6}{|p{15cm}|}{ \textbf{Actores: } Usuario (primario). Servidor (en adelante S. Secundario)

					} \\ \hline

					\multicolumn{6}{|p{15cm}|}{ \textbf{Personal Involucrado y Metas: }
					
					\emph{Usuario:} quiere transformarse en un usuario del sistema, así pueda realizar las transacciones con la aplicación de un modo seguro y personalizado.
					
					\emph{Servidor: } quiere registrar la mayor cantidad de usuarios posibles y que el proceso sea lo más rápido y seguro posible.
					} \\ \hline

					\multicolumn{6}{|p{15cm}|}{ \textbf{Precondiciones: } El usuario no está registrado en la aplicación

					} \\ \hline

					\multicolumn{6}{|p{15cm}|}{ \textbf{Poscondiciones: } Se registra al usuario como usuario de la aplicación. El usuario puede realizar operaciones en la aplicación.

					} \\ \hline

					\multicolumn{6}{|p{15cm}|}{ \textbf{Escenario Principal: }
							\begin{enumerate}
									\item El usuario ejecuta la aplicación móvil (en adelante APP) en su Smartphone y decide registrarse.
									\item APP muestra un formulario de carga donde ingresa sus datos personales y su nombre de usuario y contraseña.
									\item APP verifica los datos ingresados.
									\item APP solicita a S el registro del usuario.
									\item S registra al usuario y lo informa a APP.
									\item APP da la bienvenida al usuario.
							\end{enumerate}

					} \\ \hline

					\multicolumn{6}{|p{15cm}|}{ \textbf{Flujos Alternativos: }

					\textbf{A1: El sistema encuentra algún fallo para comunicarse con S}

					La secuencia A1 comienza en el punto 4 del escenario principal.
					\begin{enumerate}
							\setcounter{enumi}{4}
							\item APP informa al usuario el problema de conexión a través de un mensaje por la pantalla.
					\end{enumerate}

					El escenario vuelve al punto 4.

					\textbf{A2: Nombre de usuario existente}
					
					La secuencia A2 comienza en el punto 4 del escenario principal.
					\begin{enumerate}
							\setcounter{enumi}{4}
							\item S comunica que el nombre de usuario es existente.
					\end{enumerate}

					El escenario vuelve al punto 3.

					\textbf{A3: Contraseña inválida o no coincide con la confirmación}
					
					La secuencia A3 comienza en el punto 3 del escenario principal.
					\begin{enumerate}
							\setcounter{enumi}{3}
							\item APP informa el problema a través de un mensaje por pantalla.
					\end{enumerate}

					El escenario vuelve al punto 2.

					\textbf{A4: Dirección de correo electrónico existente}
					
					La secuencia A4 comienza en el punto 4 del escenario principal.
					\begin{enumerate}
							\setcounter{enumi}{4}
							\item S comunica que la dirección de correo electrónico es existente.
					\end{enumerate}

					El escenario vuelve al punto 2.

					\textbf{A5: Tipo de datos ingresados de manera incorrecta}
					
					La secuencia A5 comienza en el punto 3 del escenario principal.
					\begin{enumerate}
							\setcounter{enumi}{3}
							\item APP informa el problema a través de un mensaje por pantalla.
					\end{enumerate}

					El escenario vuelve al punto 2.

					} \\ \hline

					\multicolumn{6}{|p{15cm}|}{ \textbf{Requisitos de Interfaz de Usuario para todos los casos de uso: }

					Smartphone con SO Android o iOS o Windows Mobile, con pantalla táctil, cámara y GPS integrado.
					
					} \\ \hline

					\multicolumn{6}{|p{15cm}|}{ \textbf{Requisitos No-Funcionales para todos los casos de uso: }

					\emph{Tiempo de respuesta:} la interfaz debe responder dentro de un tiempo máximo de 3 segundos en una velocidad efectiva de conexión con el servidor a través de 3G.

					\emph{Disponibilidad:} debe poder accederse a toda hora, los 365 días del año.

					} \\ \hline

			\end{longtable}



		%#############################################################################
		%#   
		%#   Caso de uso 2
		%#
		%#############################################################################
		\subsection{Caso de Uso 02: Un usuario registrado desea iniciar sesión en la aplicación movil.}
			\begin{longtable}{|l|p{5.5cm}|l|p{2cm}|l|p{1.9cm}|} \hline
				\cellcolor{grisOscuro} CU02 & \multicolumn{4}{|l|}{  \cellcolor{grisOscuro} Iniciar Sesión} &  \cellcolor{grisClaro}\multirow{2}{1cm}{} \\ \cline{1-5}
				\cellcolor{grisOscuro} Revisa: &  \cellcolor{grisClaro} &  \cellcolor{grisOscuro} Fecha &  \cellcolor{grisClaro} &  \cellcolor{grisOscuro} Firma: & \cellcolor{grisClaro} \\ \hline
				\multicolumn{6}{|p{15cm}|}{ \textbf{Resumen: } Este caso de uso permite a los usuarios iniciar sesión con el nombre de usuario y contraseña de manera que el sistema le permita realizar tareas.

				} \\ \hline

				\multicolumn{6}{|p{15cm}|}{ \textbf{Actores: } Usuario (Primario). Servidor (en adelante S. Secundario)

				} \\ \hline

				\multicolumn{6}{|p{15cm}|}{ \textbf{Personal Involucrado y Metas: }

				\emph{Usuario:} quiere que el sistema lo reconozca como tal, así pueda realizar las tareas con la aplicación de un modo seguro y personalizado.

				\emph{App:} requiere identificar, de manera local, confiablemente a sus usuarios de manera de satisfacer sus intereses en cuando a seguridad, accesos a su cuenta personal y datos privados.

				} \\ \hline

				\multicolumn{6}{|p{15cm}|}{ \textbf{Precondiciones: } El usuario está registrado.

				} \\ \hline

				\multicolumn{6}{|p{15cm}|}{ \textbf{Poscondiciones: } Se identifica y autentica al usuario. Se conocen sus datos personales y opciones de personalización.

				} \\ \hline

				\multicolumn{6}{|p{15cm}|}{ \textbf{Escenario Principal: }

				\begin{enumerate}
					\item El usuario ejecuta la aplicación móvil (en adelante APP) en su Smartphone.
					\item APP descarga la informacion de inicio de sesión del servidor.
					\item APP solicita al usuario su nombre de usuario y contraseña.
					\item El usuario ingresa su nombre de usuario y contraseña.
					\item APP realiza el proceso de validación del usuario
					\item APP valida al usuario y comunica sus datos personales.
					\item APP da la bienvenida al usuario
				\end{enumerate}

				} \\ \hline

				\multicolumn{6}{|p{15cm}|}{ \textbf{Flujos Alternativos: }

				\textbf{A1: Nombre de usuario inexistente}
				
				La secuencia A1 comienza en el punto 4 del escenario principal.
				\begin{enumerate}
						\setcounter{enumi}{4}
						\item S comunica que el nombre de usuario es inexistente.
				\end{enumerate}

				El escenario vuelve al punto 2.

				\textbf{A2: Nombre de usuario existente pero contraseña inválida}
				
				La secuencia A2 comienza en el punto 4 del escenario principal.
				\begin{enumerate}
						\setcounter{enumi}{3}
						\item S comunica que la contraseña es inválida.
				\end{enumerate}

				El escenario vuelve al punto 2.

				} \\ \hline
			\end{longtable}

		%#############################################################################
		%#   
		%#   Caso de uso 3
		%#
		%#############################################################################
		\subsection{Caso de Uso 03: Un usuario registrado desea crear un registro}
			\begin{longtable}{|l|p{5.5cm}|l|p{2cm}|l|p{1.9cm}|} \hline
				\cellcolor{grisOscuro} CU03 & \multicolumn{4}{|l|}{  \cellcolor{grisOscuro} Crear Registro} &  \cellcolor{grisClaro}\multirow{2}{1cm}{} \\ \cline{1-5}
				\cellcolor{grisOscuro} Revisa: &  \cellcolor{grisClaro} &  \cellcolor{grisOscuro} Fecha &  \cellcolor{grisClaro} &  \cellcolor{grisOscuro} Firma: & \cellcolor{grisClaro} \\ \hline
				\multicolumn{6}{|p{15cm}|}{ \textbf{Resumen: } Este caso de uso permite al usuario crear un registro de manera que el sistema lo almacene en una base de datos para luego mostrarlo como información del mapa general

				} \\ \hline

				\multicolumn{6}{|p{15cm}|}{ \textbf{Actores: } Usuario (Primario). Servidor (en adelante S. Secundario)

				} \\ \hline

				\multicolumn{6}{|p{15cm}|}{ \textbf{Personal Involucrado y Metas: }

				\emph{Usuario:} quiere que el sistema lo reconozca como tal, así pueda realizar los registros a través de la aplicación móvil (en adelante APP) de un modo seguro y personalizado.

				\emph{Servidor:} quiere identificar confiablemente a sus usuarios de manera de satisfacer sus intereses en cuanto a seguridad y datos privados.

				} \\ \hline

				\multicolumn{6}{|p{15cm}|}{ \textbf{Precondiciones: } Los usuarios deben estar autenticados en APP

				} \\ \hline

				\multicolumn{6}{|p{15cm}|}{ \textbf{Poscondiciones: } Se almacena un nuevo registro en el sistema con los datos correspondientes necesarios para luego aportar información al mapa general

				} \\ \hline

				\multicolumn{6}{|p{15cm}|}{ \textbf{Escenario Principal: }

				\begin{enumerate}
					\item El usuario selecciona la opción para crear registro.
					\item APP solicita al usuario a través de un formulario los datos requeridos para la creacion del registro.
					\item El usuario completa el formulario y presiona un botón para finalizar.
					\item APP verifica que los tipos de datos ingresados en el formulario sean correctos y almacena el registro de forma local.
					\item APP detecta conexion a internet y envía de manera segura los datos a S para que sean validados.
					\item S valida los datos, realiza la creacion del registro y envía una confirmación.
					\item APP informa al usuario que la operación se realizó exitosamente.
				\end{enumerate}

				} \\ \hline

				\multicolumn{6}{|p{15cm}|}{ \textbf{Flujos Alternativos: }

				\textbf{A1: El sistema encuentra algún fallo para comunicarse con S}
				
				La secuencia A1 comienza en el punto 5 del escenario principal.
				\begin{enumerate}
						\setcounter{enumi}{5}
						\item APP reintenta enviar la información de manera automática por tiempo indefinido.
				\end{enumerate}

				El escenario vuelve al punto 5.

				} \\ \hline

			\end{longtable}

		%#############################################################################
		%#   
		%#   Caso de uso 4
		%#
		%#############################################################################
		\subsection{Caso de Uso 04: Un usuario registrado desea listar sus registros.}
			\begin{longtable}{|l|p{5.5cm}|l|p{2cm}|l|p{1.9cm}|} \hline
				\cellcolor{grisOscuro} CU04 & \multicolumn{4}{|l|}{  \cellcolor{grisOscuro} Listar Registros} &  \cellcolor{grisClaro}\multirow{2}{1cm}{} \\ \cline{1-5}
				\cellcolor{grisOscuro} Revisa: &  \cellcolor{grisClaro} &  \cellcolor{grisOscuro} Fecha &  \cellcolor{grisClaro} &  \cellcolor{grisOscuro} Firma: & \cellcolor{grisClaro} \\ \hline
				\multicolumn{6}{|p{15cm}|}{ \textbf{Resumen: } Este caso de uso permite a un usuario listar la información de todos los registros que realizo.

				} \\ \hline

				\multicolumn{6}{|p{15cm}|}{ \textbf{Actores: } Usuario (Primario). Servidor (Secundario).

				} \\ \hline

				\multicolumn{6}{|p{15cm}|}{ \textbf{Personal Involucrado y Metas: }

				\emph{Usuario:} quiere visualizar de manera completa la información de los registros realizados.

				\emph{Servidor:} quiere que el usuario pueda ver la información relacionada con sus registros de manera segura.

				} \\ \hline

				\multicolumn{6}{|p{15cm}|}{ \textbf{Precondiciones: } El usuario debe estar autenticado en APP.

				} \\ \hline

				\multicolumn{6}{|p{15cm}|}{ \textbf{Poscondiciones: } Se muestra un listado de registros realizados.

				} \\ \hline

				\multicolumn{6}{|p{15cm}|}{ \textbf{Escenario Principal: }

				\begin{enumerate}
					\item El usuario selecciona la opción para listar los registros realizados.
					\item APP envía la solicitud a S.
					\item S envía los datos necesarios para generar el listado de registros a APP.
					\item APP muestra el listado al usuario.
				\end{enumerate}

				} \\ \hline

				\multicolumn{6}{|p{15cm}|}{ \textbf{Flujos Alternativos: }
				
				\textbf{A1: El sistema encuentra algún fallo para comunicarse con S}
				
				La secuencia A1 comienza en el punto 2 del escenario principal.
				\begin{enumerate}
						\setcounter{enumi}{2}
						\item APP informa al usuario el problema de conexión a través de un mensaje por la pantalla.
				\end{enumerate}

				El escenario vuelve al punto 2.

				} \\ \hline

			\end{longtable}

		%#############################################################################
		%#   
		%#   Caso de uso 5
		%#
		%#############################################################################
		\subsection{Caso de Uso 05: Un usuario registrado desea ver un registro en particular}
			\begin{longtable}{|l|p{5.5cm}|l|p{2cm}|l|p{1.9cm}|} \hline
				\cellcolor{grisOscuro} CU05 & \multicolumn{4}{|l|}{  \cellcolor{grisOscuro} Ver Registro} &  \cellcolor{grisClaro}\multirow{2}{1cm}{} \\ \cline{1-5}
				\cellcolor{grisOscuro} Revisa: &  \cellcolor{grisClaro} &  \cellcolor{grisOscuro} Fecha &  \cellcolor{grisClaro} &  \cellcolor{grisOscuro} Firma: & \cellcolor{grisClaro} \\ \hline
				\multicolumn{6}{|p{15cm}|}{ \textbf{Resumen: } Este caso de uso permite al usuario ver un registro especifico que él realizo.

				} \\ \hline

				\multicolumn{6}{|p{15cm}|}{ \textbf{Actores: } Usuario (Primario). Servidor (Secundario)

				} \\ \hline

				\multicolumn{6}{|p{15cm}|}{ \textbf{Personal Involucrado y Metas: }
				
				\emph{Usuario:} quiere visualizar de manera completa la información de un registro, realizado por él, en particular.

				\emph{Servidor:} quiere que el usuario pueda ver la información relacionada con su registro de manera segura.

				} \\ \hline

				\multicolumn{6}{|p{15cm}|}{ \textbf{Precondiciones: } El usuario debe estar autenticado en APP.

				} \\ \hline

				\multicolumn{6}{|p{15cm}|}{ \textbf{Poscondiciones: } Se muestra la informacion completa de un registro.

				} \\ \hline

				\multicolumn{6}{|p{15cm}|}{ \textbf{Escenario Principal: }

				\begin{enumerate}
					\item El usuario selecciona la opción para ver un registro realizado.
					\item APP envía la solicitud a S.
					\item S envía los datos necesarios para generar la vista de un registro a APP.
					\item APP muestra el registro al usuario.
				\end{enumerate}

				} \\ \hline

				\multicolumn{6}{|p{15cm}|}{ \textbf{Flujos Alternativos: }
				
				\textbf{A1: El sistema encuentra algún fallo para comunicarse con S}
				
				La secuencia A1 comienza en el punto 2 del escenario principal.
				\begin{enumerate}
					\setcounter{enumi}{2}
					\item APP informa al usuario el problema de conexión a través de un mensaje por la pantalla.
				\end{enumerate}

				El escenario vuelve al punto 2.

				} \\ \hline
			\end{longtable}

		%#############################################################################
		%#   
		%#   Caso de uso 6
		%#
		%#############################################################################
		\subsection{Caso de Uso 06: Un usuario registrado desea ver el mapa general.}
			\begin{longtable}{|l|p{5.5cm}|l|p{2cm}|l|p{1.9cm}|} \hline
				\cellcolor{grisOscuro} CU06 & \multicolumn{4}{|l|}{  \cellcolor{grisOscuro} Ver Mapa} &  \cellcolor{grisClaro}\multirow{2}{1cm}{} \\ \cline{1-5}
				\cellcolor{grisOscuro} Revisa: &  \cellcolor{grisClaro} &  \cellcolor{grisOscuro} Fecha &  \cellcolor{grisClaro} &  \cellcolor{grisOscuro} Firma: & \cellcolor{grisClaro} \\ \hline
				\multicolumn{6}{|p{15cm}|}{ \textbf{Resumen: } Este caso de uso permite al usuario ver el mapa general con la informacion de todos los registros.

				} \\ \hline

				\multicolumn{6}{|p{15cm}|}{ \textbf{Actores: } Usuario (Primario). Servidor (Secundario)

				} \\ \hline

				\multicolumn{6}{|p{15cm}|}{ \textbf{Personal Involucrado y Metas: }
				
				\emph{Usuario:} quiere visualizar de manera completa la información del mapa general.

				\emph{Servidor:} quiere que el usuario pueda ver la información relacionada con el mapa general de forma segura.

				} \\ \hline

				\multicolumn{6}{|p{15cm}|}{ \textbf{Precondiciones: } 
				
				El usuario debe estar autenticado en APP.
				
				Debe existir al menos 1 registro en el servidor.

				} \\ \hline

				\multicolumn{6}{|p{15cm}|}{ \textbf{Poscondiciones: } Se muestra la informacion completa del mapa general.

				} \\ \hline

				\multicolumn{6}{|p{15cm}|}{ \textbf{Escenario Principal: }

				\begin{enumerate}
					\item El usuario selecciona la opción para ver el mapa general.
					\item APP envía la solicitud a S.
					\item S envía los datos necesarios para generar la vista del mapa general a APP.
					\item APP muestra el mapa general al usuario.
				\end{enumerate}

				} \\ \hline

				\multicolumn{6}{|p{15cm}|}{ \textbf{Flujos Alternativos: }
				
				\textbf{A1: El sistema encuentra algún fallo para comunicarse con S}
				
				La secuencia A1 comienza en el punto 2 del escenario principal.
				\begin{enumerate}
					\setcounter{enumi}{2}
					\item APP informa al usuario el problema de conexión a través de un mensaje por la pantalla.
				\end{enumerate}

				El escenario vuelve al punto 2.

				} \\ \hline

			\end{longtable}

		%#############################################################################
		%#   
		%#   Caso de uso 7
		%#
		%#############################################################################
		\subsection{Caso de Uso 07: Un usuario registrado desea ver su perfil.}
			\begin{longtable}{|l|p{5.5cm}|l|p{2cm}|l|p{1.9cm}|} \hline
				\cellcolor{grisOscuro} CU07 & \multicolumn{4}{|l|}{  \cellcolor{grisOscuro} Ver Perfil de Usuario} &  \cellcolor{grisClaro}\multirow{2}{1cm}{} \\ \cline{1-5}
				\cellcolor{grisOscuro} Revisa: &  \cellcolor{grisClaro} &  \cellcolor{grisOscuro} Fecha &  \cellcolor{grisClaro} &  \cellcolor{grisOscuro} Firma: & \cellcolor{grisClaro} \\ \hline
				\multicolumn{6}{|p{15cm}|}{ \textbf{Resumen: } Este caso de uso permite al usuario ver su perfil con la informacion de sus datos.

				} \\ \hline

				\multicolumn{6}{|p{15cm}|}{ \textbf{Actores: } Usuario (Primario). Servidor (Secundario)

				} \\ \hline

				\multicolumn{6}{|p{15cm}|}{ \textbf{Personal Involucrado y Metas: }
				
				\emph{Usuario:} quiere visualizar de manera completa la información de su perfil.

				\emph{Servidor:} quiere que el usuario pueda ver la información relacionada con su perfil de forma segura.

				} \\ \hline

				\multicolumn{6}{|p{15cm}|}{ \textbf{Precondiciones: } El usuario debe estar autenticado en APP.

				} \\ \hline

				\multicolumn{6}{|p{15cm}|}{ \textbf{Poscondiciones: } Se muestra la informacion completa del perfil del usuario.

				} \\ \hline

				\multicolumn{6}{|p{15cm}|}{ \textbf{Escenario Principal: }

				\begin{enumerate}
					\item El usuario selecciona la opción para ver su perfil.
					\item APP envía la solicitud a S.
					\item S envía los datos necesarios para generar la vista del perfil a APP.
					\item APP muestra el perfil de usuario por pantalla.
				\end{enumerate}

				} \\ \hline

				\multicolumn{6}{|p{15cm}|}{ \textbf{Flujos Alternativos: }
				
				\textbf{A1: El sistema encuentra algún fallo para comunicarse con S}
				
				La secuencia A1 comienza en el punto 2 del escenario principal.
				\begin{enumerate}
					\setcounter{enumi}{2}
					\item APP informa al usuario el problema de conexión a través de un mensaje por la pantalla.
				\end{enumerate}

				El escenario vuelve al punto 1.

				} \\ \hline

			\end{longtable}

		%#############################################################################
		%#   
		%#   Caso de uso 8
		%#
		%#############################################################################
		\subsection{Caso de Uso 08: Un usuario registrado desea modificar su perfil.}
			\begin{longtable}{|l|p{5.5cm}|l|p{2cm}|l|p{1.9cm}|} \hline
				\cellcolor{grisOscuro} CU08 & \multicolumn{4}{|l|}{  \cellcolor{grisOscuro} Modificar Perfil de Usuario} &  \cellcolor{grisClaro}\multirow{2}{1cm}{} \\ \cline{1-5}
				\cellcolor{grisOscuro} Revisa: &  \cellcolor{grisClaro} &  \cellcolor{grisOscuro} Fecha &  \cellcolor{grisClaro} &  \cellcolor{grisOscuro} Firma: & \cellcolor{grisClaro} \\ \hline
				\multicolumn{6}{|p{15cm}|}{ \textbf{Resumen: } Este caso de uso permite al usuario editar los datos de su perfil.

				} \\ \hline

				\multicolumn{6}{|p{15cm}|}{ \textbf{Actores: } Usuario (Primario). Servidor (Secundario)

				} \\ \hline

				\multicolumn{6}{|p{15cm}|}{ \textbf{Personal Involucrado y Metas: }
				
				\emph{Usuario:} quiere editar los datos de su perfil.

				\emph{Servidor:} quiere mantener actualizados los datos del usuario.

				} \\ \hline

				\multicolumn{6}{|p{15cm}|}{ \textbf{Precondiciones: } El usuario debe estar autenticado en APP.

				} \\ \hline

				\multicolumn{6}{|p{15cm}|}{ \textbf{Poscondiciones: } Se registran los cambios de los datos del perfil de usuario en el servidor.

				} \\ \hline

				\multicolumn{6}{|p{15cm}|}{ \textbf{Escenario Principal: }

				\begin{enumerate}
					\item El usuario selecciona la opción para editar su perfil.
					\item APP solicita al usuario a través de un formulario los datos de su perfil que pueden ser modificados o mantenidos.
					\item El usuario completa el formulario y presiona un botón para finalizar.
					\item APP envía de manera segura los datos a S para que sean validados.
					\item S valida los datos, realiza la actualización y enviá una confirmación.
					\item APP informa al usuario que la operación se realizó exitosamente.
				\end{enumerate}

				} \\ \hline

				\multicolumn{6}{|p{15cm}|}{ \textbf{Flujos Alternativos: }
				
				\textbf{A1: El sistema encuentra algún fallo para comunicarse con S}
				
				La secuencia A1 comienza en el punto 4 del escenario principal.
				\begin{enumerate}
					\setcounter{enumi}{4}
					\item APP informa al usuario el problema de conexión a través de un mensaje por la pantalla.
				\end{enumerate}

				El escenario vuelve al punto 2.

				} \\ \hline
			\end{longtable}

		%#############################################################################
		%#   
		%#   Caso de uso 10
		%#
		%#############################################################################
		\subsection{Caso de Uso 10: Un usuario registrado desea eliminar su cuenta.}
			\begin{longtable}{|l|p{5.5cm}|l|p{2cm}|l|p{1.9cm}|} \hline
					\cellcolor{grisOscuro} CU10 & \multicolumn{4}{|l|}{  \cellcolor{grisOscuro} Eliminar Cuenta De Usuario} &  \cellcolor{grisClaro}\multirow{2}{1cm}{} \\ \cline{1-5}
					\cellcolor{grisOscuro} Revisa: &  \cellcolor{grisClaro} &  \cellcolor{grisOscuro} Fecha &  \cellcolor{grisClaro} &  \cellcolor{grisOscuro} Firma: & \cellcolor{grisClaro} \\ \hline
					\multicolumn{6}{|p{15cm}|}{ \textbf{Resumen: } Este caso de uso permite al usuario eliminar su cuenta de usuario del servidor.

					} \\ \hline

					\multicolumn{6}{|p{15cm}|}{ \textbf{Actores: } Usuario (Primario). Servidor (Secundario)

					} \\ \hline

					\multicolumn{6}{|p{15cm}|}{ \textbf{Personal Involucrado y Metas: }
					
					\emph{Usuario:} quiere eliminar su cuenta de usuario.

					\emph{Servidor:} quiere mantener actualizados los datos del usuario.

					} \\ \hline

					\multicolumn{6}{|p{15cm}|}{ \textbf{Precondiciones: } El usuario debe estar autenticado en APP.

					} \\ \hline

					\multicolumn{6}{|p{15cm}|}{ \textbf{Poscondiciones: } Se registran los cambios de los datos del perfil de usuario en el servidor.

					} \\ \hline

					\multicolumn{6}{|p{15cm}|}{ \textbf{Escenario Principal: }

					\begin{enumerate}
							\item El usuario selecciona la opción para eliminar su perfil.
							\item APP solicita al usuario a través de una alerta la confirmación para eliminar el perfil.
							\item El usuario presiona el botón aceptar.
							\item APP envía la solicitud a S.
							\item S envía confirmación.
							\item APP informa al usuario que la operación se realizó exitosamente.
					\end{enumerate}

					} \\ \hline

					\multicolumn{6}{|p{15cm}|}{ \textbf{Flujos Alternativos: }
					
					\textbf{A1: El sistema encuentra algún fallo para comunicarse con S}
					
					La secuencia A1 comienza en el punto 4 del escenario principal.
					\begin{enumerate}
							\setcounter{enumi}{4}
							\item APP informa al usuario el problema de conexión a través de un mensaje por la pantalla.
					\end{enumerate}

					El escenario vuelve al punto 1.

					} \\ \hline

			\end{longtable}
		%#############################################################################
		%#   
		%#   Caso de uso 11
		%#
		%#############################################################################
		\subsection{Caso de Uso 11: Un administrador desea iniciar sesión.}
			\begin{longtable}{|l|p{5.5cm}|l|p{2cm}|l|p{1.9cm}|} \hline
					\cellcolor{grisOscuro} CU11 & \multicolumn{4}{|l|}{  \cellcolor{grisOscuro} Iniciar Sesión} &  \cellcolor{grisClaro}\multirow{2}{1cm}{} \\ \cline{1-5}
					\cellcolor{grisOscuro} Revisa: &  \cellcolor{grisClaro} &  \cellcolor{grisOscuro} Fecha &  \cellcolor{grisClaro} &  \cellcolor{grisOscuro} Firma: & \cellcolor{grisClaro} \\ \hline
					\multicolumn{6}{|p{15cm}|}{ \textbf{Resumen: } Este caso de uso permite a los administradores iniciar sesión con el nombre de usuario y contraseña de manera que el sistema le permita realizar tareas.

					} \\ \hline

					\multicolumn{6}{|p{15cm}|}{ \textbf{Actores: } Administrador (Primario). Servidor (en adelante S. Secundario)

					} \\ \hline

					\multicolumn{6}{|p{15cm}|}{ \textbf{Personal Involucrado y Metas: }

					\emph{Administrador:} quiere que el sistema lo reconozca como tal, así pueda realizar las tareas de gestión con la aplicación de un modo seguro y personalizado.

					\emph{Servidor:} requiere identificar confiablemente a sus administradores de manera de satisfacer sus intereses en cuando a seguridad, accesos a su cuenta personal y datos privados.

					} \\ \hline

					\multicolumn{6}{|p{15cm}|}{ \textbf{Precondiciones: } El administrador está registrado.

					} \\ \hline

					\multicolumn{6}{|p{15cm}|}{ \textbf{Poscondiciones: } Se identifica y autentica al administrador. Se conocen sus datos personales y opciones de personalización.

					} \\ \hline

					\multicolumn{6}{|p{15cm}|}{ \textbf{Escenario Principal: }

					\begin{enumerate}
							\item El administrador ejecuta el sistema de gestion de registros mediante (en adelante SGR) una URL en un navegador web.
							\item SGR solicita al administrador su nombre de usuario y contraseña.
							\item El administrador ingresa su nombre de usuario y contraseña.
							\item SGR solicita a S la validación del administrador
							\item S valida al administrador y comunica sus datos personales.
							\item SGR da la bienvenida al administrador
					\end{enumerate}

					} \\ \hline

					\multicolumn{6}{|p{15cm}|}{ \textbf{Flujos Alternativos: }

					\textbf{A1: El sistema encuentra algún fallo para comunicarse con S}
					
					La secuencia A1 comienza en el punto 4 del escenario principal.
					\begin{enumerate}
							\setcounter{enumi}{4}
							\item SGR informa al administrador el problema de conexión a través de un mensaje por la pantalla.
					\end{enumerate}

					El escenario vuelve al punto 2.

					\textbf{A2: Nombre de usuario inexistente}
					
					La secuencia A2 comienza en el punto 4 del escenario principal.
					\begin{enumerate}
							\setcounter{enumi}{4}
							\item S comunica que el nombre de usuario es inexistente.
					\end{enumerate}

					El escenario vuelve al punto 2.

					\textbf{A3: Nombre de usuario existente pero contraseña inválida}
					
					La secuencia A3 comienza en el punto 4 del escenario principal.
					\begin{enumerate}
							\setcounter{enumi}{3}
							\item S comunica que la contraseña es inválida.
					\end{enumerate}

					El escenario vuelve al punto 2.

					} \\ \hline

			\end{longtable}

		%#############################################################################
		%#   
		%#   Caso de uso 12
		%#
		%#############################################################################
		\subsection{Caso de Uso 12: Un administrador desea borrar o dar de baja un alumno.}
		%#############################################################################
		%#   
		%#   Caso de uso 13
		%#
		%#############################################################################
		\subsection{Caso de Uso 13: Un administrador desea cambiar el rol a un usuario.}
			\begin{longtable}{|l|p{5.5cm}|l|p{2cm}|l|p{1.9cm}|} \hline
					\cellcolor{grisOscuro} CU13 & \multicolumn{4}{|l|}{  \cellcolor{grisOscuro} Cambiar el rol a un usuario} &  \cellcolor{grisClaro}\multirow{2}{1cm}{} \\ \cline{1-5}
					\cellcolor{grisOscuro} Revisa: &  \cellcolor{grisClaro} &  \cellcolor{grisOscuro} Fecha &  \cellcolor{grisClaro} &  \cellcolor{grisOscuro} Firma: & \cellcolor{grisClaro} \\ \hline
					\multicolumn{6}{|p{15cm}|}{ \textbf{Resumen: } Este caso de uso permite al administrador cambiar el rol a un usuario en particular.

					} \\ \hline

					\multicolumn{6}{|p{15cm}|}{ \textbf{Actores: } Administrador (Primario). Servidor (Secundario)

					} \\ \hline

					\multicolumn{6}{|p{15cm}|}{ \textbf{Personal Involucrado y Metas: }
					
					\emph{Administrador:} quiere cambiar el rol a un usuario.

					\emph{Servidor:} quiere mantener actualizados los datos del usuario.

					} \\ \hline

					\multicolumn{6}{|p{15cm}|}{ \textbf{Precondiciones: } El administrador debe estar autenticado en SGR.

					} \\ \hline

					\multicolumn{6}{|p{15cm}|}{ \textbf{Poscondiciones: } Se registran los cambios de los datos de la cuenta de usuario en el servidor.

					} \\ \hline

					\multicolumn{6}{|p{15cm}|}{ \textbf{Escenario Principal: }

					\begin{enumerate}
							\item El administrador selecciona la opción para cambiar el rol a un usuario.
							\item SGR solicita al administrador a través de una alerta la confirmación para dicha accion.
							\item El administrador presiona el botón aceptar.
							\item SGR envía la solicitud a S.
							\item S envía confirmación.
							\item SGR informa al administrador que la operación se realizó exitosamente.
					\end{enumerate}

					} \\ \hline

					\multicolumn{6}{|p{15cm}|}{ \textbf{Flujos Alternativos: }
					
					\textbf{A1: El sistema encuentra algún fallo para comunicarse con S}
					
					La secuencia A1 comienza en el punto 4 del escenario principal.
					\begin{enumerate}
							\setcounter{enumi}{4}
							\item SGR informa al administrador el problema de conexión a través de un mensaje por la pantalla.
					\end{enumerate}

					El escenario vuelve al punto 1.

					} \\ \hline

			\end{longtable}

		%#############################################################################
		%#   
		%#   Caso de uso 14
		%#
		%#############################################################################
		\subsection{Caso de Uso 14: Un administrador desea listar los alumnos.}
			\begin{longtable}{|l|p{5.5cm}|l|p{2cm}|l|p{1.9cm}|} \hline
					\cellcolor{grisOscuro} CU14 & \multicolumn{4}{|l|}{  \cellcolor{grisOscuro} Listar Usuarios} &  \cellcolor{grisClaro}\multirow{2}{1cm}{} \\ \cline{1-5}
					\cellcolor{grisOscuro} Revisa: &  \cellcolor{grisClaro} &  \cellcolor{grisOscuro} Fecha &  \cellcolor{grisClaro} &  \cellcolor{grisOscuro} Firma: & \cellcolor{grisClaro} \\ \hline
					\multicolumn{6}{|p{15cm}|}{ \textbf{Resumen: } Este caso de uso permite a un administrador listar la información de todos usuarios registrados en el sistema.

					} \\ \hline

					\multicolumn{6}{|p{15cm}|}{ \textbf{Actores: } Administrador (Primario). Servidor (Secundario).

					} \\ \hline

					\multicolumn{6}{|p{15cm}|}{ \textbf{Personal Involucrado y Metas: }

					\emph{Administrador:} quiere visualizar de manera completa la información de los usuarios registrados.

					\emph{Servidor:} quiere que el administrador pueda ver la información relacionada con los perfiles de usuarios de manera segura.

					} \\ \hline

					\multicolumn{6}{|p{15cm}|}{ \textbf{Precondiciones: } El administrador debe estar autenticado en SGR.

					} \\ \hline

					\multicolumn{6}{|p{15cm}|}{ \textbf{Poscondiciones: } Se muestra un listado de usuarios registrados.

					} \\ \hline

					\multicolumn{6}{|p{15cm}|}{ \textbf{Escenario Principal: }

					\begin{enumerate}
							\item El administrador selecciona la opción para listar los usuarios registrados.
							\item SGR envía la solicitud a S.
							\item S envía los datos necesarios para generar el listado de usuarios a SGR.
							\item SGR muestra el listado de usuarios al administrador.
					\end{enumerate}

					} \\ \hline

					\multicolumn{6}{|p{15cm}|}{ \textbf{Flujos Alternativos: }
					
					\textbf{A1: El sistema encuentra algún fallo para comunicarse con S}
					
					La secuencia A1 comienza en el punto 2 del escenario principal.
					\begin{enumerate}
							\setcounter{enumi}{2}
							\item SGR informa al usuario el problema de conexión a través de un mensaje por la pantalla.
					\end{enumerate}

					El escenario vuelve al punto 2.

					} \\ \hline

			\end{longtable}

		%#############################################################################
		%#   
		%#   Caso de uso 15
		%#
		%#############################################################################
		\subsection{Caso de Uso 15: Un administrador desea listar los registros.}
			\begin{longtable}{|l|p{5.5cm}|l|p{2cm}|l|p{1.9cm}|} \hline
					\cellcolor{grisOscuro} CU15 & \multicolumn{4}{|l|}{  \cellcolor{grisOscuro} Listar Registros} &  \cellcolor{grisClaro}\multirow{2}{1cm}{} \\ \cline{1-5}
					\cellcolor{grisOscuro} Revisa: &  \cellcolor{grisClaro} &  \cellcolor{grisOscuro} Fecha &  \cellcolor{grisClaro} &  \cellcolor{grisOscuro} Firma: & \cellcolor{grisClaro} \\ \hline
					\multicolumn{6}{|p{15cm}|}{ \textbf{Resumen: } Este caso de uso permite a un administrador listar la información de todos los registros realizados por la totalidad de los usuarios.

					} \\ \hline

					\multicolumn{6}{|p{15cm}|}{ \textbf{Actores: } Administrador (Primario). Servidor (Secundario).

					} \\ \hline

					\multicolumn{6}{|p{15cm}|}{ \textbf{Personal Involucrado y Metas: }

					\emph{Administrador:} quiere visualizar de manera completa la información de los registros realizados.

					\emph{Servidor:} quiere que el administrador pueda ver la información relacionada con los registros de manera segura.

					} \\ \hline

					\multicolumn{6}{|p{15cm}|}{ \textbf{Precondiciones: } El administrador debe estar autenticado en SGR.

					} \\ \hline

					\multicolumn{6}{|p{15cm}|}{ \textbf{Poscondiciones: } Se muestra un listado de registros realizados.

					} \\ \hline

					\multicolumn{6}{|p{15cm}|}{ \textbf{Escenario Principal: }

					\begin{enumerate}
							\item El administrador selecciona la opción para listar los registros realizados.
							\item SGR envía la solicitud a S.
							\item S envía los datos necesarios para generar el listado de registros a SGR.
							\item SGR muestra en pantalla el listado de registros al administrador.
					\end{enumerate}

					} \\ \hline

					\multicolumn{6}{|p{15cm}|}{ \textbf{Flujos Alternativos: }
					
					\textbf{A1: El sistema encuentra algún fallo para comunicarse con S}
					
					La secuencia A1 comienza en el punto 2 del escenario principal.
					\begin{enumerate}
							\setcounter{enumi}{2}
							\item SGR informa al administrador el problema de conexión a través de un mensaje por la pantalla.
					\end{enumerate}

					El escenario vuelve al punto 2.

					} \\ \hline

			\end{longtable}

		%#############################################################################
		%#   
		%#   Caso de uso 16
		%#
		%#############################################################################
		\subsection{Caso de Uso 16: Un administrador desea ver un registro en particular.}
			\begin{longtable}{|l|p{5.5cm}|l|p{2cm}|l|p{1.9cm}|} \hline
					\cellcolor{grisOscuro} CU16 & \multicolumn{4}{|l|}{  \cellcolor{grisOscuro} Ver Registro} &  \cellcolor{grisClaro}\multirow{2}{1cm}{} \\ \cline{1-5}
					\cellcolor{grisOscuro} Revisa: &  \cellcolor{grisClaro} &  \cellcolor{grisOscuro} Fecha &  \cellcolor{grisClaro} &  \cellcolor{grisOscuro} Firma: & \cellcolor{grisClaro} \\ \hline
					\multicolumn{6}{|p{15cm}|}{ \textbf{Resumen: } Este caso de uso permite al administrador ver un registro en particular con toda su informacion.

					} \\ \hline

					\multicolumn{6}{|p{15cm}|}{ \textbf{Actores: } Administrador (Primario). Servidor (Secundario)

					} \\ \hline

					\multicolumn{6}{|p{15cm}|}{ \textbf{Personal Involucrado y Metas: }
					
					\emph{Administrador:} quiere visualizar de manera completa la información de un registro.

					\emph{Servidor:} quiere que el administrador pueda ver la información relacionada con un registro de forma segura.

					} \\ \hline

					\multicolumn{6}{|p{15cm}|}{ \textbf{Precondiciones: } El administrador debe estar autenticado en SGR.

					} \\ \hline

					\multicolumn{6}{|p{15cm}|}{ \textbf{Poscondiciones: } Se muestra la informacion completa del registro de un usuario.

					} \\ \hline

					\multicolumn{6}{|p{15cm}|}{ \textbf{Escenario Principal: }

					\begin{enumerate}
							\item El administrador selecciona la opción de listar los registros.
							\item El administrador, sobre un registro de la lista, selecciona la opción de ver registro.
							\item SGR envía la solicitud a S.
							\item S envía los datos necesarios para generar la vista del registro a SGR.
							\item SGR muestra el registro completo por pantalla.
					\end{enumerate}

					} \\ \hline

					\multicolumn{6}{|p{15cm}|}{ \textbf{Flujos Alternativos: }
					
					\textbf{A1: El sistema encuentra algún fallo para comunicarse con S}
					
					La secuencia A1 comienza en el punto 3 del escenario principal.
					\begin{enumerate}
							\setcounter{enumi}{3}
							\item SGR informa al usuario el problema de conexión a través de un mensaje por la pantalla.
					\end{enumerate}

					El escenario vuelve al punto 2.

					} \\ \hline

			\end{longtable}

		%#############################################################################
		%#   
		%#   Caso de uso 17
		%#
		%#############################################################################
		\subsection{Caso de Uso UC17: Un administrador desea validar o invalidar registros.}
			\begin{longtable}{|l|p{5.5cm}|l|p{2cm}|l|p{1.9cm}|} \hline
					\cellcolor{grisOscuro} CU17 & \multicolumn{4}{|l|}{  \cellcolor{grisOscuro} Validar o Invalidar Registro} &  \cellcolor{grisClaro}\multirow{2}{1cm}{} \\ \cline{1-5}
					\cellcolor{grisOscuro} Revisa: &  \cellcolor{grisClaro} &  \cellcolor{grisOscuro} Fecha &  \cellcolor{grisClaro} &  \cellcolor{grisOscuro} Firma: & \cellcolor{grisClaro} \\ \hline
					\multicolumn{6}{|p{15cm}|}{ \textbf{Resumen: } Este caso de uso permite al administrador validar o invalidar un registro en particular.

					} \\ \hline

					\multicolumn{6}{|p{15cm}|}{ \textbf{Actores: } Administrador (Primario). Servidor (Secundario)

					} \\ \hline

					\multicolumn{6}{|p{15cm}|}{ \textbf{Personal Involucrado y Metas: }
					
					\emph{Administrador:} quiere validar o invalidar un registro creado por un usuario.

					\emph{Servidor:} quiere mantener actualizados los datos de los registros.

					} \\ \hline

					\multicolumn{6}{|p{15cm}|}{ \textbf{Precondiciones: } El administrador debe estar autenticado en SGR.

					} \\ \hline

					\multicolumn{6}{|p{15cm}|}{ \textbf{Poscondiciones: } Se registran los cambios de los datos del registro en cuestión en el servidor.

					} \\ \hline

					\multicolumn{6}{|p{15cm}|}{ \textbf{Escenario Principal: }

					\begin{enumerate}
							\item El administrador ingresa a la opción para ver un registro en particular.
							\item El administrador selecciona la opción de validar o invalidar registro.
							\item SGR solicita al administrador a través de una alerta la confirmación para validar o invalidar el registro.
							\item El administrador presiona el botón aceptar.
							\item SGR envía la solicitud a S.
							\item S envía confirmación.
							\item SGR informa al administrador que la operación se realizó exitosamente.
					\end{enumerate}

					} \\ \hline

					\multicolumn{6}{|p{15cm}|}{ \textbf{Flujos Alternativos: }
					
					\textbf{A1: El sistema encuentra algún fallo para comunicarse con S}
					
					La secuencia A1 comienza en el punto 5 del escenario principal.
					\begin{enumerate}
							\setcounter{enumi}{5}
							\item SGR informa al administrador el problema de conexión a través de un mensaje por la pantalla.
					\end{enumerate}

					El escenario vuelve al punto 2.

					} \\ \hline

			\end{longtable}

		%#############################################################################
		%#   
		%#   Caso de uso 18/19/20
		%#
		%#############################################################################
		\subsection{Caso de Uso UC18/19/20: Un administrador desea buscar registros por fecha de creación/Institución/Indice.}
			\begin{longtable}{|l|p{5.5cm}|l|p{2cm}|l|p{1.9cm}|} \hline
					\cellcolor{grisOscuro} CU18/19/20 & \multicolumn{4}{|l|}{  \cellcolor{grisOscuro} Buscar Registro} &  \cellcolor{grisClaro}\multirow{2}{1cm}{} \\ \cline{1-5}
					\cellcolor{grisOscuro} Revisa: &  \cellcolor{grisClaro} &  \cellcolor{grisOscuro} Fecha &  \cellcolor{grisClaro} &  \cellcolor{grisOscuro} Firma: & \cellcolor{grisClaro} \\ \hline
					\multicolumn{6}{|p{15cm}|}{ \textbf{Resumen: } Este caso de uso permite al administrador realizar una busqueda de registros filtrados por un rango de fechas.

					} \\ \hline

					\multicolumn{6}{|p{15cm}|}{ \textbf{Actores: } Administrador (Primario). Servidor (Secundario)

					} \\ \hline

					\multicolumn{6}{|p{15cm}|}{ \textbf{Personal Involucrado y Metas: }
					
					\emph{Administrador:} quiere obtener los registros que coincidan con un determinado criterio de busqueda.

					\emph{Servidor:} quiere ofrecer al administrador los registros que mejor se ajustan a su búsqueda

					} \\ \hline

					\multicolumn{6}{|p{15cm}|}{ \textbf{Precondiciones: } El administrador debe estar autenticado en SGR.

					} \\ \hline

					\multicolumn{6}{|p{15cm}|}{ \textbf{Poscondiciones: } Se obtiene un listado de registros que cumplen con la condición de busqueda.

					} \\ \hline

					\multicolumn{6}{|p{15cm}|}{ \textbf{Escenario Principal: }

					\begin{enumerate}
							\item El administrador ingresa el texto que buscará (rango de fechas o institución o indice)
							\item El administrador presiona un botón para comenzar la búsqueda.
							\item SGR envía la solicitud a S.
							\item S verifica los registros que cumplen con el criterio buscado.
							\item S retorna el listado de registros
							\item SGR muestra el listado al administrador.
					\end{enumerate}

					} \\ \hline

					\multicolumn{6}{|p{15cm}|}{ \textbf{Flujos Alternativos: }
					
					\textbf{A1: El sistema encuentra algún fallo para comunicarse con S}
					
					La secuencia A1 comienza en el punto 3 del escenario principal.
					\begin{enumerate}
							\setcounter{enumi}{3}
							\item SGR informa al administrador el problema de conexión a través de un mensaje por la pantalla.
					\end{enumerate}

					El escenario vuelve al punto 2.

					} \\ \hline

			\end{longtable}

		%#############################################################################
		%#   
		%#   Caso de uso 22
		%#
		%#############################################################################
		\subsection{Caso de Uso UC22: Un administrador desea exportar a Excel la información.}
			\begin{longtable}{|l|p{5.5cm}|l|p{2cm}|l|p{1.9cm}|} \hline
					\cellcolor{grisOscuro} CU22 & \multicolumn{4}{|l|}{  \cellcolor{grisOscuro} Exportar Excel} &  \cellcolor{grisClaro}\multirow{2}{1cm}{} \\ \cline{1-5}
					\cellcolor{grisOscuro} Revisa: &  \cellcolor{grisClaro} &  \cellcolor{grisOscuro} Fecha &  \cellcolor{grisClaro} &  \cellcolor{grisOscuro} Firma: & \cellcolor{grisClaro} \\ \hline
					\multicolumn{6}{|p{15cm}|}{ \textbf{Resumen: } Este caso de uso permite al administrador exportar un archivo formato excel con la informacion de los registros.

					} \\ \hline

					\multicolumn{6}{|p{15cm}|}{ \textbf{Actores: } Administrador (Primario). Servidor (Secundario)

					} \\ \hline

					\multicolumn{6}{|p{15cm}|}{ \textbf{Personal Involucrado y Metas: }
					
					\emph{Administrador:} quiere exportar un archivo excel con la informacion de los registros.

					\emph{Servidor:} quiere ofrecer al administrador el archivo excel con los registros.
					} \\ \hline

					\multicolumn{6}{|p{15cm}|}{ \textbf{Precondiciones: } 
					
					El administrador debe estar autenticado en SGR.

					} \\ \hline

					\multicolumn{6}{|p{15cm}|}{ \textbf{Poscondiciones: } Se facilita un archivo excel para la descarga.

					} \\ \hline

					\multicolumn{6}{|p{15cm}|}{ \textbf{Escenario Principal: }

					\begin{enumerate}
							\item El administrador selecciona la opción para ver exportar un archivo excel.
							\item SGR envía la solicitud a S.
							\item S genera el archivo excel con los registros.
							\item S envía el archivo a SGR.
							\item SGR proporciona una ventana para la descarga del archivo excel.
					\end{enumerate}

					} \\ \hline

					\multicolumn{6}{|p{15cm}|}{ \textbf{Flujos Alternativos: }
					
					\textbf{A1: El sistema encuentra algún fallo para comunicarse con S}
					
					La secuencia A1 comienza en el punto 2 del escenario principal.
					\begin{enumerate}
							\setcounter{enumi}{2}
							\item SGR informa al usuario el problema de conexión a través de un mensaje por la pantalla.
					\end{enumerate}

					El escenario vuelve al punto 2.

					} \\ \hline

			\end{longtable}
	
	
	%#############################################################################
	%#   
	%#   DIAGRAMAS DE CLASES DE DISEÑO 
	%#
	%#############################################################################
	\section{Diagrama de Clases}
		Un diagrama de clases sirve para visualizar las relaciones entre las clases que involucran el sistema, las cuales pueden ser asociativas, de herencia, de uso y de contenimiento. Se puede utilizar un diagrama de clases para describir los tipos de datos y sus relaciones con independencia de su implementación. El diagrama se utiliza para que la atención se centre en los aspectos lógicos de las clases en lugar de en su implementación.


		\subsection{Vista general}
			\begin{figure}[H]
				\centering
					\includegraphics[width=1\textwidth]{imagenes/DiagramasUML/clase.png}
							%%Me parece que queda mejor sin el hfill
							%\hfill
					\label{fig:diagrama-clases-disenio}
			\end{figure}
	%#############################################################################
	%#   
	%#   DIAGRAMAS DE ACTIVIDAD
	%#
	%#############################################################################
	\section{Diagramas de Actividad}
		\subsection{Inicio de Sesión}
			\begin{figure}[H]
			\centering
				\includegraphics[width=1\textwidth]{imagenes/analisis/diagrama-actividad-inicioSesion.png}
					%%Me parece que queda mejor sin el hfill 
					%\hfill
				\label{fig:diagrama-actividad-autenticar}
			\end{figure}

		\subsection{Registrar Usuario}
			\begin{figure}[H]
			\centering
				\includegraphics[width=1\textwidth]{imagenes/analisis/diagrama-actividad-registrar.png}
					%%Me parece que queda mejor sin el hfill
					%\hfill 
				\label{fig:diagrama-actividad-registrar}
			\end{figure}

		\subsection{Crear Registro }
			\begin{figure}[H]
			\centering
				\includegraphics{imagenes/analisis/diagrama-actividad-crear-registro.png}
					%%Me parece que queda mejor sin el hfill
					%\hfill
				\label{fig:diagrama-actividad-crear-tienda}
			\end{figure}

		\subsection{Ver Mapa Interactivo}
			\begin{figure}[H]
			\centering
				\includegraphics[width=1\textwidth]{imagenes/analisis/diagrama-actividad-ver-mapa.png}
					%%Me parece que queda mejor sin el hfill
					%\hfill
				\label{fig:diagrama-actividad-comprar-producto}
				\end{figure}
			
	%#############################################################################
	%#   
	%#   DIAGRAMAS DE SECUENCIA 
	%#
	%#############################################################################

	\section{Diagramas de Secuencia}
		\subsection{Iniciar Sesion}
			\begin{figure}[H]
				\centering
				\includegraphics[width=1\textwidth]{imagenes/DiagramasUML/sdIniciarSesion.png}
							%%Me parece que queda mejor sin el hfill
							%\hfill
					\label{fig:diagrama-secuencia-autenticar}
			\end{figure}

		\subsection{Registrar}
			\begin{figure}[H]
				\centering
				\includegraphics[width=1\textwidth]{imagenes/DiagramasUML/sdRegistrar.png}
							%%Me parece que queda mejor sin el hfill
							%\hfill
					\label{fig:diagrama-secuencia-registrar}
			\end{figure}

		\subsection{Crear Registro}
			\begin{figure}[H]
        \centering
        \includegraphics[width=1\textwidth]{imagenes/DiagramasUML/sdCrearRegistro.png}
							%%Me parece que queda mejor sin el hfill
							%\hfill
					\label{fig:diagrama-secuencia-crear-tienda}
			\end{figure}

	%#############################################################################
	%#   
	%#   INTERFAZ DE USUARIO
	%#
	%#############################################################################

	\section{Interfaz de usuario}
		\begin{itemize}
			\item \subsection{Aplicacion movil}
			
				\subsection{Iniciar Sesion}
					\begin{figure}[H]
						\centering
							\includegraphics[width=0.6\textwidth]{Screenshots/login.png}
									%%Me parece que queda mejor sin el hfill
									%\hfill
									\caption{Incio de sesion mediante nombre de usuario y contraseNIA}
							\label{fig:login}
					\end{figure}

				\subsection{Configuracion}
					\begin{figure}
						\centering
							\includegraphics[width=0.6\textwidth]{Screenshots/configuracion.png}
									%%Me parece que queda mejor sin el hfill
									%\hfill
									\caption{Pantalla de uso general}
							\label{fig:configuracion}
					\end{figure}

				\subsection{Vista Perfil}
					\begin{figure}
						\centering
							\includegraphics[width=0.6\textwidth]{Screenshots/perfil.png}
									%%Me parece que queda mejor sin el hfill
									%\hfill
									\caption{Pantalla para ver/editar/eliminar el usuario.}
							\label{fig:perfil}
					\end{figure}

				\subsection{Crear Registro}

					\begin{itemize}
						\item Paso 1: Obtencion automatica de coordenadas con vista satelital del mapa
							\begin{figure}
								\centering
									\includegraphics[width=0.6\textwidth]{Screenshots/registroPaso1.png}
											%%Me parece que queda mejor sin el hfill
											%\hfill
											\caption{Paso automatico obligatorio al iniciar un nuevo registro}
									\label{fig:registroPaso1}
							\end{figure}

						\item Paso 2: Capturar fotos correspondientes
							\begin{figure}
								\centering
									\includegraphics[width=0.6\textwidth]{Screenshots/registroPaso2A.png}
											%%Me parece que queda mejor sin el hfill
											%\hfill
											\caption{Vista sin fotos capturadas}
									\label{fig:registroPaso2A}
							\end{figure}
							\begin{figure}
								\centering
									\includegraphics[width=0.6\textwidth]{Screenshots/registroPaso2B.png}
											%%Me parece que queda mejor sin el hfill
											%\hfill
											\caption{Vista con fotos capturadas}
									\label{fig:registroPaso2B}
							\end{figure}

						\item Paso 3: Seleccionar insectos encontrados
							\begin{figure}
									\includegraphics[width=0.3\textwidth]{Screenshots/registroPaso3A.png}
											%%Me parece que queda mejor sin el hfill
											%\hfill
									\label{fig:registroPaso3A}
							\end{figure}
							\begin{figure}
									\includegraphics[width=0.3\textwidth]{Screenshots/registroPaso3B.png}
											%%Me parece que queda mejor sin el hfill
											%\hfill
									\label{fig:registroPaso3B}
							\end{figure}
							\begin{figure}
									\includegraphics[width=0.3\textwidth]{Screenshots/registroPaso3C.png}
											%%Me parece que queda mejor sin el hfill
											%\hfill
									\label{fig:registroPaso3C}
							\end{figure}
							\begin{figure}
								\centering
									\includegraphics[width=0.6\textwidth]{Screenshots/registroPaso3Completo.png}
											%%Me parece que queda mejor sin el hfill
											%\hfill
											\caption{Formulario vista vertical completa. Modo ilustrativo}
									\label{fig:registroPaso3Completo}
							\end{figure}
						\item Paso 4: Ruleta animada
							\begin{figure}
								\centering
									\includegraphics[width=0.6\textwidth]{Screenshots/ruedita.png}
											%%Me parece que queda mejor sin el hfill
											%\hfill
											\caption{Ruleta indicadora de indice segun insectos encontrados}
									\label{fig:ruedita}
							\end{figure}

					\end{itemize}

				\subsection{Lista de registros creados por el usuario}
					\begin{figure}
						\centering
							\includegraphics[width=0.6\textwidth]{Screenshots/registrosLocales.png}
									%%Me parece que queda mejor sin el hfill
									%\hfill
									\caption{Lista con desplazamiento vertical para ver registros correspondientes al usuario logueado}
							\label{fig:registrosLocales}
					\end{figure}

				\subsection{Ver registro}
					\begin{figure}
							\includegraphics[width=0.3\textwidth]{Screenshots/verRegistro1.png}
									%%Me parece que queda mejor sin el hfill
									%\hfill
							\label{fig:verRegistro1}
					\end{figure}
					\begin{figure}
							\includegraphics[width=0.3\textwidth]{Screenshots/verRegistro2.png}
									%%Me parece que queda mejor sin el hfill
									%\hfill
							\label{fig:verRegistro2}
					\end{figure}
					\begin{figure}
							\includegraphics[width=0.3\textwidth]{Screenshots/verRegistro3.png}
									%%Me parece que queda mejor sin el hfill
									%\hfill
							\label{fig:verRegistro3}
					\end{figure}

					\begin{figure}
						\centering
							\includegraphics[width=0.6\textwidth]{Screenshots/verRegistroCompleto.png}
									%%Me parece que queda mejor sin el hfill
									%\hfill
									\caption{Vista completa de un registro con desplazamiento vertical}
							\label{fig:verRegistroCompleto}
					\end{figure}


			\item \subsection{Sistema de gestion WEB}

				\subsection{Iniciar Sesion}
					\begin{figure}
						\centering
							\includegraphics[width=0.6\textwidth]{Screenshots/web/login.png}
									%%Me parece que queda mejor sin el hfill
									%\hfill
							\label{fig:login}
					\end{figure}

				\subsection{Lista de registros global con filtros}
					\begin{figure}
						\centering
							\includegraphics[width=0.6\textwidth]{Screenshots/web/registroListar.png}
									%%Me parece que queda mejor sin el hfill
									%\hfill
									\caption{Pantalla de resultados de la busqueda de registros con filtros}
							\label{fig:busqueda}
					\end{figure}

				\subsection{Ver registro}
					\begin{figure}
						\centering
							\includegraphics[width=0.6\textwidth]{Screenshots/web/registroVer1.png}
									%%Me parece que queda mejor sin el hfill
									%\hfill
									\caption{Visualizacion rapida de registro en el sector derecho de la pantalla}
							\label{fig:verRegistro}
					\end{figure}
					\begin{figure}
						\centering
							\includegraphics[width=0.6\textwidth]{Screenshots/web/registroVer2.png}
									%%Me parece que queda mejor sin el hfill
									%\hfill
									\caption{Visualizacion rapida de registro en el sector derecho de la pantalla}
							\label{fig:verRegistro}
					\end{figure}

				\subsection{Lista de usuarios global con filtros}
					\begin{figure}
						\centering
							\includegraphics[width=0.6\textwidth]{Screenshots/web/usuarioListar.png}
									%%Me parece que queda mejor sin el hfill
									%\hfill
									\caption{Pantalla de resultados de la busqueda de usuarios con filtros}
							\label{fig:busqueda}
					\end{figure}

				\subsection{Ver registro}
					\begin{figure}
						\centering
							\includegraphics[width=0.6\textwidth]{Screenshots/web/usuarioVer.png}
									%%Me parece que queda mejor sin el hfill
									%\hfill
									\caption{Visualizacion rapida del perfil de usuario en el sector derecho de la pantalla}
							\label{fig:verRegistro}
					\end{figure}

		\end{itemize}

%% Disciplina de Implementación
%%%%%%%%%%%%%%%%%%%%%%%%%%%%%%%%%%%%%%%%%%%%%%%%%%%%%%%%
%   |------------------------------------------|       %
%   |Aplicación de comercio electrónico para   |       %
%   |teléfonos móviles con S.O. Android        |       %
%   |                                          |       %
%   | Proyecto de graduación                   |       %
%   |__________________________________________|       %
%                                                      %
%   Autores                                            %
%   -------                                            %
%                                                      %
% * Soto, Paula Fabiana (CX05-0967-4)                  %
%     paulette255@gmail.com                            %
% * Vallejo, Sergio Daniel (CX05-0392-4)               %
%     vallejosergio@gmail.com                          %
%                                                      %
%   Tutor                                              %
%   -------                                            %
%                                                      %
% * Ing. Augusto Maximiliano Odstrcil                  %
%        modstrcil@gmail.com                           %
%                                                      %
%                                                      %
%%%%%%%%%%%%%%%%%%%%%%%%%%%%%%%%%%%%%%%%%%%%%%%%%%%%%%%%

\chapter{Disciplina de Implementación}
\label{chap:implementacion}

\section{Arquitectura de la aplicación}

\begin{figure}[H]
  \centering
    %lo que agregué entre corchetes hace que el ancho de la imagen ocupe el 80% del área de texto. Si sacás eso la imagen no se redimensiona y se va de la hoja. Se puede usar algo parecido para limitar el alto si es necesario.
    \includegraphics[width=1\textwidth]{imagenes/implementacion/arquitectura.jpg}
    
     \caption{Arquitectura de la aplicación, mostrando los frameworks utilizados}
    \label{fig:arquitectura-aplicacion}
\end{figure}

\section{Diagrama de Despliegue}

 \begin{figure}[H]
  \centering
    %lo que agregué entre corchetes hace que el ancho de la imagen ocupe el 80% del área de texto. Si sacás eso la imagen no se redimensiona y se va de la hoja. Se puede usar algo parecido para limitar el alto si es necesario.
    \includegraphics[width=1\textwidth]{imagenes/implementacion/despliegue.png}
    \label{diagrama-despliegue}
\end{figure}

\section{Elección del Lenguaje}

    Independientemente del paradigma de ingeniería de software, el lenguaje de programación tendrá impacto en la planificación, el análisis, el diseño, la codificación, la prueba y el mantenimiento de un proyecto. Para la construcción de la aplicación se eligió la utilización de los lenguajes web HTML5, CSS3 y JavaScript.

    La elección de estos lenguajes para la construcción de la apliación se debe a las siguientes ventajas que ofrecen:
    \begin{itemize}
        \item \emph{Mayor portabilidad:} Al ser tecnologías estándares y soportadas por la mayoría de los teléfonos celulares modernos, es posible que una misma aplicación sea muy fácilmente adaptable a varias plataformas móviles.
        \item \emph{Soporte futuro:} Todas las plataformas móviles están trabajando para mejorar el soporte que ofrecen a las tecnologías web, ofreciendo una mejor experiencia al usuario.
        \item \emph{Aprovechamiento de conocimiento de desarrollo de aplicaciones web:} Desarrollando aplicaciones móviles en \gls{HTML}5, \gls{CSS}3 y \gls{JavaScript} es posible aplicar el conocimiento en el desarrollo de aplicaciones web desarrolladas para navegadores en equipos de escritorio 
    \end{itemize}
   
\section{Arquitectura de Android}

Los componentes principales del sistema operativo Android, que pueden verse en la figura \ref{fig:arquitectura-android}, son:

\begin{itemize}
    \item \emph{Aplicaciones:} PRUEBA GLOSARIO PARA SO \gls{SO} las aplicaciones base incluyen un cliente de correo electrónico, programa de SMS, calendario, mapas, navegador, contactos y otros. La mayoría de las aplicaciones están escritas en lenguaje de programación \gls{Java}, aunque existen \gls{framework}s para desarrollar aplicaciones en otros lenguajes como \gls{HTML}, \gls{CSS} y \gls{JavaScript} o \gls{Python}.
    \item \emph{Marco de trabajo de aplicaciones:} los desarrolladores tienen acceso completo a los mismos APIs del framework usados por las aplicaciones base. La arquitectura está diseñada para simplificar la reutilización de componentes; cualquier aplicación puede publicar sus capacidades y cualquier otra aplicación puede luego hacer uso de esas capacidades (sujeto a reglas de seguridad del framework). Este mismo mecanismo permite que los componentes sean reemplazados por el usuario.
    \item \emph{Bibliotecas:} Android incluye un conjunto de bibliotecas de C/C++ usadas por varios componentes del sistema. Estas características se exponen a los desarrolladores a través del marco de trabajo de aplicaciones de Android; algunas son: System C library (implementación biblioteca C estándar), bibliotecas de medios, bibliotecas de gráficos, 3D y SQLite, entre otras.
    \item \emph{Runtime de Android:} Android incluye un set de bibliotecas base que proporcionan la mayor parte de las funciones disponibles en las bibliotecas base del lenguaje Java. Cada aplicación Android corre su propio proceso, con su propia instancia de la máquina virtual Dalvik. Dalvik ha sido escrito de forma que un dispositivo puede correr múltiples máquinas virtuales de forma eficiente. Dalvik ejecuta archivos en el formato Dalvik Executable (.dex), el cual está optimizado para minimizar el consumo de memoria. La Máquina Virtual está basada en registros y corre clases compiladas por el compilador de Java que han sido transformadas al formato.dex por la herramienta incluida "dx".
    
    \item \emph{Núcleo \gls{Linux}:} Android depende de Linux para los servicios base del sistema como seguridad, gestión de memoria, gestión de procesos, pila de red y modelo de controladores. El núcleo también actúa como una capa de abstracción entre el hardware y el resto de la pila de software.
   
 \end{itemize}

\begin{figure}[htbp]
  \centering
    %lo que agregué entre corchetes hace que el ancho de la imagen ocupe el 80% del área de texto. Si sacás eso la imagen no se redimensiona y se va de la hoja. Se puede usar algo parecido para limitar el alto si es necesario.
    \includegraphics[width=0.8\textwidth]{imagenes/arquitecturaAndroid.png}
    
     \caption{Arquitectura de Android}
    \label{fig:arquitectura-android}
\end{figure}

\section{PhoneGap}

\begin{figure}[htbp]
  \centering
    %lo que agregué entre corchetes hace que el ancho de la imagen ocupe el 80% del área de texto. Si sacás eso la imagen no se redimensiona y se va de la hoja. Se puede usar algo parecido para limitar el alto si es necesario.
    \includegraphics[width=0.5\textwidth]{imagenes/phonegap.png}
    
     \caption{Arquitectura de PhoneGap}
    \label{arquitectura-phonegap}
\end{figure}

PhoneGap es un \gls{framework} de código abierto que actúa como un intermediario entre las aplicaciones web y los dispositivos móviles. Permite crear aplicaciones móviles instalables utilizando tecnología web: \gls{JavaScript}, \gls{HTML}5 y \gls{CSS}3.

Las aplicaciones resultantes no son totalmente nativas, ni puramente basado en la web. La desventaja de que una aplicación sea totalmente nativa es que sólo se podrá utilizar para la plataforma para la que fue realizada, es decir si se hace una aplicación para Android luego no se podrá reutilizar el código para hacer la misma aplicación para iOS.

Con PhoneGap se puede reutilizar el código de una aplicación para crear el paquete instalable de cualquiera de las 7 plataformas móviles soportadas: iOS, Android, Blackberry, Windows Phone, WebOS de Palm, Symbian y Bada.

PhoneGap permite acceder a funciones nativas como el acelerómetro, cámara, brújula, contactos, archivos, ubicación geográfica, almacenamiento  y notificaciones.

PhoneGap para Android esta dividido en dos partes:

\begin{itemize}
    \item \emph{Librerías nativas (phonegap.jar):} Agrega acceso JavaScript para APIs nativas.
    
    \item \emph{Archivos javascript (phonegap.js):} Contenedores JavaScript para llamados de APIs nativas.
    
   
 \end{itemize}
\subsection{Ventajas de PhoneGap}

\begin{itemize}

    \item Soporta 7 plataformas móviles: iOS, Android, Blackberry, Windows Phone, WebOS de Palm, Symbian y Bada.
    
    \item Acceso a características nativas de cada plataforma a través de su API, a las que una aplicación web visitada desde el navegador no podría acceder, como acceso a la cámara de fotos, acelerómetro, notificaciones, etc.
    
    \item Permite ejecutar a través de JavaScript plugins escritos en código nativo.
    
    \item Permite distribuir aplicaciones realizadas utilizando HTML5 y JavaScript a través de las tiendas de aplicaciones oficiales de cada plataforma.
    
    \end{itemize}
    \subsection{Desventaja de phoneGap}
    \begin{itemize}
\item Normalmente las aplicaciones realizadas con PhoneGap tienen un menor rendimiento en tareas que requieren alta capacidad de procesamiento, sobre todo en versiones antiguas de las plataformas sobre las que se usa.

\item Se pierde la posibilidad de acceder a algunas características nativas, como los diferentes elementos de interfaz de usuario propios de cada plataforma, aunque estos pueden imitarse mediante el uso de CSS.
\end{itemize}

\section{Web services}
\label{sec:webservices}

Un servicio web (en inglés, Web service) es una tecnología que utiliza un conjunto de protocolos y estándares que sirven para intercambiar datos entre aplicaciones. Distintas aplicaciones de software desarrolladas en lenguajes de programación diferentes, y ejecutadas sobre cualquier plataforma, pueden utilizar los servicios web para intercambiar datos en redes de ordenadores como Internet. La interoperabilidad se consigue mediante la adopción de estándares abiertos.

Las ventajas de los servicios web son:

\begin{itemize}
    \item Aportan interoperabilidad entre aplicaciones de software independientemente de sus propiedades o de las plataformas sobre las que se instalen.
    \item Los servicios Web fomentan los estándares y protocolos basados en texto, que hacen más fácil acceder a su contenido y entender su funcionamiento.
    \item Permiten que servicios y software de diferentes compañías ubicadas en diferentes lugares geográficos puedan ser combinados fácilmente para proveer servicios integrados.
   
 \end{itemize}
 
\subsection{Razones para crear servicios Web}

La principal razón para usar servicios Web es que se pueden utilizar con HTTP sobre \gls{TCP} en el puerto 80. Dado que las organizaciones protegen sus redes mediante firewalls -que filtran y bloquean gran parte del tráfico de Internet, cierran casi todos los puertos TCP salvo el 80, que es, precisamente, el que usan los navegadores. Los servicios Web utilizan este puerto, por la simple razón de que no resultan bloqueados. Es importante señalar que los servicios web se pueden utilizar sobre cualquier protocolo, sin embargo, TCP es el más común.

Otra razón por la que los servicios Web son muy prácticos es que pueden aportar gran independencia entre la aplicación que usa el servicio Web y el propio servicio. De esta forma, los cambios a lo largo del tiempo en uno no deben afectar al otro. Esta flexibilidad será cada vez más importante, dado que la tendencia a construir grandes aplicaciones a partir de componentes distribuidos más pequeños es cada día más utilizada.
Se pueden desarrollar servicios web como parte de una aplicación web, permitiendo acceder a los mismos datos que esta.


\subsection{REST}



REST es una técnica de arquitectura software para sistemas hipermedia distribuidos como la World Wide Web.
REST describe cualquier interfaz web simple que utiliza \gls{XML} (o \gls{JSON}) y HTTP, sin las abstracciones adicionales de los protocolos basados en patrones de intercambio de mensajes como el protocolo de servicios web \gls{SOAP}. 

Los sistemas que siguen los principios REST se llaman con frecuencia RESTful.

\begin{figure}[htbp]
  \centering
    %lo que agregué entre corchetes hace que el ancho de la imagen ocupe el 80% del área de texto. Si sacás eso la imagen no se redimensiona y se va de la hoja. Se puede usar algo parecido para limitar el alto si es necesario.
    \includegraphics[width=0.8\textwidth]{imagenes/REST.jpg}
    
     \caption{Web Services REST}
    \label{fig:REST}
\end{figure}

REST afirma que la web ha disfrutado de escalabilidad como resultado de una serie de diseños fundamentales clave:

\begin{itemize}
    \item \emph{Un protocolo cliente/servidor sin estado:} cada mensaje HTTP contiene toda la información necesaria para comprender la petición. Como resultado, ni el cliente ni el servidor necesitan recordar ningún estado de las comunicaciones entre mensajes. Sin embargo, en la práctica, muchas aplicaciones basadas en HTTP utilizan cookies y otros mecanismos para mantener el estado de la sesión (algunas de estas prácticas, como la reescritura de URLs, no son permitidas por REST)
    
    \item \emph{Un conjunto de operaciones bien definidas que se aplican a todos los recursos de información:} HTTP en sí define un conjunto pequeño de operaciones, las más importantes son POST, GET, PUT y DELETE.
    
    \item Una sintaxis universal para identificar los recursos. En un sistema REST, cada recurso es direccionable únicamente a través de su \gls{URI}.
    \item El uso de hipermedios, tanto para la información de la aplicación como para las transiciones de estado de la aplicación: la representación de este estado en un sistema REST son típicamente \gls{HTML}, XML o JSON. Como resultado de esto, es posible navegar de un recurso REST a muchos otros, simplemente siguiendo enlaces sin requerir el uso de registros u otra infraestructura adicional.
   
 \end{itemize}
 
 \subsubsection{Recursos}
 Un concepto importante en REST es la existencia de recursos (elementos de información), que pueden ser accedidos utilizando un identificador global (un Identificador Uniforme de Recurso).
 
 Para manipular estos recursos, los componentes de la red (clientes y servidores) se comunican a través de una interfaz estándar (HTTP) e intercambian representaciones de estos recursos (los ficheros que se descargan y se envían.
 
La petición puede ser transmitida por cualquier número de conectores (por ejemplo clientes, servidores, cachés, túneles, etc.) pero cada uno lo hace sin "ver más allá" de su propia petición. Así, una aplicación puede interactuar con un recurso conociendo el identificador del recurso y la acción requerida, no necesitando conocer si existen cachés, proxys, cortafuegos, túneles o cualquier otra cosa entre ella y el servidor que guarda la información. La aplicación, sin embargo, debe comprender el formato de la información devuelta (la representación), que es por lo general un documento \gls{HTML}, \gls{XML} o \gls{JSON}, aunque también puede ser una imagen o cualquier otro contenido.


\section{Características de seguridad}

\subsection{Canal de comunicación cifrado}

En todos los casos en los que se realiza transferencia de datos confidenciales del usuario, como sus credenciales de acceso, datos personales u operaciones realizadas, es necesario asegurar que tanto las solicitudes y las respuestas se envían a través de un canal de comunicación cifrado.

La utilización de protocolos de comunicación inseguros, como HTTP, pueden hacer que la comunicación pueda ser interceptada utilizando ataques \gls{Man-in-the-middle}, permitiendo que el atacante pueda ver todos los datos intercambiados e incluso manipularlos o generar solicitudes falsas.

Para mitigar este riesgo, las comunicaciones de la aplicación se realizan utilizando el protocolo de comunicación segura \gls{HTTPS}. Este está basado en HTTP, pero  utiliza un cifrado basado en \gls{SSL/TLS} para crear un canal cifrado (cuyo nivel de cifrado depende del servidor remoto) más apropiado para el tráfico de información sensible que el protocolo HTTP. De este modo se consigue que la información sensible no pueda ser usada por un atacante que haya conseguido interceptar la transferencia de datos de la conexión, ya que lo único que obtendrá será un flujo de datos cifrados que le resultará imposible de descifrar, como se muestra en la figura \ref{fig:https}.

\begin{figure}[htbp]
  \centering
    \includegraphics[width=0.9\textwidth]{imagenes/implementacion/https.png}
     \caption{Captura de la solicitud cifrada como lo vería un atacante utilizando Wireshark}
    \label{fig:https}
\end{figure}

\subsection{ACL para cada objeto}

Como ya se habló anteriormente, los Web services pueden ser públicamente accedidos a través de Internet, por lo que es imprescindible contar con un mecanismo que impida la realización de consultas o modificaciones no autorizadas. En la aplicación es necesario que los datos estén definidos correctamente: sólo un vendedor puede realizar la modificación de los datos de una tienda y de sus productos, sólo puede modificar los datos personales el usuario al que pertenece la cuenta activa y las órdenes no pueden modificarse una vez creadas y sólo pueden ser vistas por el usuario comprador o el vendedor.

Para asegurar que los datos sólo son accesibles por usuarios que están autorizados a leerlos o modificarlos se implementó un \gls{ACL} para cada objeto. Este dato se almacena cuando el objeto es creado, y está representado con un listado en formato \gls{JSON} incluyendo los permisos de los usuarios sobre ese objeto. En el siguiente ejemplo se muestra el caso de un objeto que es visible públicamente pero sólo el usuario con id \texttt{idUsuario} puede realizar modificaciones:

\begin{verbatim}
{
    "idUsuario":
        {
            "read":true,
            "write":true
        },
    "*":
        {
            "read":true
        }
}
\end{verbatim}

\begin{figure}[htbp]
  \centering
    \includegraphics[width=0.9\textwidth]{imagenes/implementacion/acl.png}
     \caption{Solicitud de modificación rechazada a un objeto existente utilizando REST Client}
    \label{fig:acl}
\end{figure}

Esta medida de seguridad debe ser controlada en el servidor web que recibe las peticiones, debido a que controlarlo en el cliente no soluciona el problema de las modificaciones no autorizadas. Una solicitud a un objeto sin usar las credenciales de un usuario autorizado debe ser denegada por el servidor, como se muestra en la figura \ref{fig:acl}. Es importante también que el mensaje retornado por el servidor, en caso de que el objeto no tenga permisos de visualización por el usuario actual, no revele evidencia de la existencia del objeto.

\section{Backbone.js}

Backbone.js es un \gls{framework} para Javascript con un interfaz RESTful por \gls{JSON} , basada en el paradigma de diseño de aplicaciones Modelo Vista Presentador (MVP).
Está diseñado para desarrollar aplicaciones de una única página y para mantener las diferentes partes de las aplicaciones web (p.e. múltiples clientes y un servidor) sincronizadas.

Backbone.js posee cuatro clases principales:

\begin{itemize}
    \item Model 
    \item View
    \item Router
    \item Collection
 \end{itemize}
 
 \subsection{Modelo Vista Presentador (MVP)}
 
 El patrón Modelo-Vista-Presentador (MVP) surge como una variación del patrón Modelo-Vista-Controlador (MVC).
 
 Los componentes básicos de este patrón son:
 
\begin{figure}[htbp]
  \centering
    %lo que agregué entre corchetes hace que el ancho de la imagen ocupe el 80% del área de texto. Si sacás eso la imagen no se redimensiona y se va de la hoja. Se puede usar algo parecido para limitar el alto si es necesario.
    \includegraphics{imagenes/MVP.png}
    
     \caption{Componentes MVP}
    \label{MVP}
\end{figure}

 \begin{itemize}
    \item \emph {Modelo:} El modelo es normalmente los datos de la aplicación y la lógica para recuperar y conservar los datos. A menudo, se trata de un modelo de dominio que puede basarse en una base de datos o los resultados de los servicios web. En algunos casos, que el modelo de dominio corresponde perfectamente a lo que se ve en la pantalla, pero en otros casos ha de ser adaptada, agregados o extendido para ser utilizable.
    
    \item \emph {Vista:} La vista es típicamente un control de usuario o formulario que combina varios en una interfaz de usuario. El usuario puede interactuar con los controles en la vista
    %, pero cuando se necesita cierta lógica para iniciarse, la vista este delegado al presentador.
    
   \item \emph {Presentador:} El presentador tiene toda la lógica de la vista y es responsable de sincronizar el modelo y la vista. Cuando la vista notifica el presentador que el usuario ha hecho algo (por ejemplo, hacer clic en un botón), el presentador a continuación, actualizar el modelo y sincronizar los cambios entre el modelo y la vista.

      
 \end{itemize}


\section{jQuery}

jQuery es una biblioteca de \gls{JavaScript} que permite simplificar la manera de interactuar con los documentos \gls{HTML}, manipular el árbol \gls{DOM}, manejar eventos, desarrollar animaciones y agregar interacción con la técnica \gls{AJAX} a páginas web.

Es software libre y de código abierto, posee un doble licenciamiento bajo la Licencia MIT y la Licencia Pública General de GNU v2, permitiendo su uso en proyectos libres y cerrados.

Al igual que otras bibliotecas, ofrece una serie de funcionalidades basadas en JavaScript que de otra manera requerirían de mucho más código, es decir, con las funciones propias de esta biblioteca se logran grandes resultados en menos tiempo y líneas de código.

\section{jQuery Mobile}
jQuery Mobile es un \gls{framework} para Javascript utilizado en el desarrollo de la interfaz de usuario para aplicaciones web adaptadas a dispositivos móviles. Incluye elementos de UI y soporte a eventos relacionados con el uso de pantallas táctiles.

Las principales características de jQuery Mobile son:

\begin{itemize}
    \item Es compatible con otros frameworks que utilizamos para el desarrollo de la aplicación móvil, tales como Phonegap o Backbone.js.
    \item Es compatible con las principales plataformas móviles, así como todos los navegadores de escritorio principales, incluyendo Android.
    \item Construido sobre jQuery.
    \item Permite la creación de temas personalizados mediante el agregado de estilos CSS. Proporciona además una base de temas que permite a los desarrolladores personalizar las combinaciones de colores y determinados aspectos de las características de interfaz de usuario.
    \item Tiene mínimas dependencias, aumentando la velocidad y reduciendo el consumo de memoria.
    \item Adaptación automática del diseño al tamaño de la pantalla.
 \end{itemize}
 

\section{Herramientas de desarrollo}

\begin{itemize}

\item \emph {Eclipse:} Es un entorno de desarrollo integrado (IDE) de código abierto multiplataforma.

\item \emph {Visual Paradigm para UML:} Es una herramienta UML profesional que soporta el ciclo de vida completo del desarrollo de software: análisis y diseño orientados a objetos, construcción, pruebas y despliegue. 

Permite dibujar todos los tipos de diagramas de clases, código inverso, generar código desde diagramas y generar documentación.

\item \emph{\LaTeX:} Es un sistema de composición de textos, orientado especialmente a la creación de libros, documentos científicos y técnicos que contengan fórmulas matemáticas.

\LaTeX facilita el uso del lenguaje de composición tipográfica. Es muy utilizado para la composición de artículos académicos, tesis y libros técnicos, dado que la calidad tipográfica de los documentos realizados con \LaTeX es comparable a la de una editorial científica de primera línea.



\item \emph{ShareLaTeX:} Es un editor online (en tiempo real)de \LaTeX, que puede ser utilizado por varios usuarios a la vez. 

\item \emph{Dropbox y Google Drive:} Son servicios de alojamiento de archivos multiplataforma en la nube, operado por las compañías Dropbox y Google. 

Permiten a los usuarios almacenar y sincronizar archivos en línea y entre computadoras y compartir archivos y carpetas con otros.

\end{itemize}



%% Disciplina de Pruebas
%%%%%%%%%%%%%%%%%%%%%%%%%%%%%%%%%%%%%%%%%%%%%%%%%%%%%%%%
%   |------------------------------------------|       %
%   | Web App embebida en dispositivos móviles |       %
%   |  para la gestión de registros sobre la   |       %
%   |   contaminación de afluentes y ríos.     |       %
%   |                                          |       %
%   |          Proyecto de graduación          |       %
%   |__________________________________________|       %
%                                                      %
%   Autores                                            %
%   -------                                            %
%                                                      %
% * Bruno, Ricardo Hugo (CX 1409686)                   %
%     rburnount@gmail.com                              %
% * Gómez Veliz, Kevin Shionen (CX 1411828)            %
%     ing.gomezvelizkevin@gmail.com                    %
%                                                      %
%   Tutor                                              %
%   -------                                            %
%                                                      %
% * Ing. Cohen, Daniel Eduardo                         %
%        dcohen.tuc@gmail.com                          %
%                                                      %
%   Cotutor                                            %
%   -------                                            %
%                                                      %
% * Ing. Nieto, Luis Eduardo                           %
%        lnieto@herrera.unt.edu.ar                     %
%                                                      %
%                                                      %
%%%%%%%%%%%%%%%%%%%%%%%%%%%%%%%%%%%%%%%%%%%%%%%%%%%%%%%%

\chapter{Disciplina de Pruebas}
\label{chap:pruebas}

\section{Test de Unidades}
	\subsection{Introducción}

		El Test de Unidades consiste en realizar pruebas de las unidades individuales de código. En esta fase se realizan las pruebas de caja blanca. 

	\subsection{Pruebas de Caja Blanca}
		Es un tipo de método de prueba que permite detectar errores internos del código de cada módulo. 

		Con estas pruebas se pueden garantizar que se ejercitan por lo menos una vez todos los caminos independientes de cada módulo, que las decisiones lógicas se evalúan en sus dos variantes (verdadera y falsa), que se ejecutan todos los bucles en sus límites operacionales y que se ejercitan las estructuras internas de datos para asegurar su validez.

\section{Test de Módulos}

	\subsection{Introducción}
		El Test de Módulos consiste en realizar pruebas de los módulos funcionales del sistema. En esta fase se realizan las pruebas de caja negra y las pruebas de estrés. 

	\subsection{Pruebas de Caja Negra}

		En este método de prueba se ve a cada módulo como una caja negra y se generan conjuntos de condiciones de entrada que ejerciten completamente todos los requisitos funcionales del programa, observando las salidas.

		Con estas pruebas se pueden detectar funciones incorrectas o ausentes, errores de interfaz, errores de rendimiento, etc.
			
	\subsection{Pruebas de Estrés}

		Esta prueba se centra en realizar el análisis de valores límites, y en condiciones límites, ya que se ha demostrado que los errores tienden a darse más en los límites del campo de entrada y sometidos a condiciones límites.

\section{Test de Integración}

	\subsection{Introducción}

		El Test de Integración consiste en realizar pruebas de la estructura modular del programa y su interacción a través de la prueba de integración.

	\subsection{Pruebas de Integración}

		En este tipo de prueba los errores surgen al integrar los módulos. En esta fase se pueden detectar errores como por ejemplo que las subfunción, es cuando se combinan pueden no producir la función principal, un módulo puede tener un efecto adverso e inadvertido sobre otro, etc.

		El objetivo es tomar los módulos probados y construir una estructura de programa que esté de acuerdo con lo que dicta la especificación C.
			
		Existen dos tipos de integración:
			\begin{itemize}
				\item \textbf{Integración descendente:} En este tipo se integran los módulos moviéndose hacia abajo por la jerarquía de control, comenzando con el módulo de control inicial.
				\item \textbf{Integración ascendente:} En este tipo se integran los módulos atómicos primero y luego se continúa con el nivel inmediato superior.
			\end{itemize}

	En el desarrollo de este sistema se utilizó la integración descendente.

\section{Test de Aceptación}

	\subsection{Introducción}

		El Test de Aceptación consiste en realizar la prueba del software para validar si funciona de acuerdo con las expectativas razonables del cliente. En esta fase se llevan a cabo las pruebas Alfa y Beta.

	\subsection{Prueba Alfa}

		Esta prueba es conducida por el cliente en el lugar de desarrollo. Se usa el software de forma natural (previa capacitación), con el encargado de desarrollo mirando “por encima del hombro” del usuario y registrando errores y problemas de uso. Se lleva acabo en un entorno controlado.
		
	\subsection{Prueba Beta}

		Esta prueba se lleva a cabo en uno o más lugares de clientes, por los usuarios finales de software. El encargado de desarrollo no está presente. El cliente registra todos los problemas (reales o imaginarios) que  encuentra durante la prueba o informa a intervalos regulares al equipo de desarrollo. Se lleva a cabo en un entorno no controlado.

		Los procedimientos de prueba se diseñaron para asegurar que se satisfacen todos los requisitos funcionales y que se alcanzan todos los requisitos de rendimiento.

%% Conclusiones
%%%%%%%%%%%%%%%%%%%%%%%%%%%%%%%%%%%%%%%%%%%%%%%%%%%%%%%%
%   |------------------------------------------|       %
%   | Web App embebida en dispositivos móviles |       %
%   |  para la gestión de registros sobre la   |       %
%   |   contaminación de afluentes y ríos.     |       %
%   |                                          |       %
%   |          Proyecto de graduación          |       %
%   |__________________________________________|       %
%                                                      %
%   Autores                                            %
%   -------                                            %
%                                                      %
% * Bruno, Ricardo Hugo (CX 1409686)                   %
%     rburnount@gmail.com                              %
% * Gómez Veliz, Kevin Shionen (CX 1411828)            %
%     ing.gomezvelizkevin@gmail.com                    %
%                                                      %
%   Tutor                                              %
%   -------                                            %
%                                                      %
% * Ing. Cohen, Daniel Eduardo                         %
%        dcohen.tuc@gmail.com                          %
%                                                      %
%   Cotutor                                            %
%   -------                                            %
%                                                      %
% * Ing. Nieto, Luis Eduardo                           %
%        lnieto@herrera.unt.edu.ar                     %
%                                                      %
%                                                      %
%%%%%%%%%%%%%%%%%%%%%%%%%%%%%%%%%%%%%%%%%%%%%%%%%%%%%%%%

\chapter{Conclusiones}

El desarrollo de nuestro proyecto final nos permitió poner en práctica temas que aprendimos en distintas asignaturas durante el transcurso de nuestra carrera, el aprendizaje y la experiencia de implementar nuevas tecnologías, enfrentándonos a problemas reales de diseño e integración que nos forzaron a tomar decisiones a fin de encontrar soluciones eficientes a los mismos. Además, nos dio la posibilidad de aprender a trabajar con herramientas con los cuales no estábamos familiarizados.

En cuanto a la dificultad, consideramos que fue media/alta debido a varias razones, una de ellas fue acceder al hardware del dispositivo mediante herramientas ``web'' requirió mucha investigación, pruebas y corrección de mínimos detalles para facilitar la interacción del usuario final con la aplicación. Otra de las razones fue implementar CouchDB (Base de datos no relacional) por ser robusta en lo que se refiere a replicaciones, dado que se amoldaba a nuestro requisito de generar registros de manera offline y al detectar conexión a internet, subirlos automáticamente. Pero esto nos genero un gran problema, dicha base de datos realizaba versionados, por lo que para cada actualización y replicación, la misma crecía exponencialmente su tamaño en disco, lo que iba a ser un problema en un futuro por la cantidad de registros que el sistema gestionaría. Consideramos que tomamos una decisión errónea al seleccionar e implementar esa base de datos en el sistema, solo habiendo investigado y leído la documentación oficial, de manera parcial y no en su totalidad.
Este error nos costo bastante en lo que a tiempo se refiere, ya que tuvimos que rediseñar la lógica de la aplicación para poder implementar una base de datos relacional (MySQL) y esto implico crear un protocolo de auto-sincronización, lo que genero mas trabajo y mas tiempo de desarrollo (aproximadamente 18 meses).

En las pruebas alpha realizadas, pudimos observar una buena respuesta al uso de la aplicación de los docentes de la Facultad de Ciencias Naturales e Instituto Miguel Lillo, sin embargo, se detectaron algunas fallas y falta de funcionalidades. Esto genero una lista con ítems a corregir e implementar, lo que genero aun mas tiempo de desarrollo para la puesta en producción del sistema en su versión final.

Con la futura puesta en funcionamiento de este proyecto podremos generar un producto que da valor agregado a la metodología de trabajo existente de los docentes de la Facultad de Ciencias Naturales e Instituto Miguel Lillo en conjunto con alumnos de escuelas rurales, que no poseen automatismo en dicha metodología, cumpliendo así, con los objetivos planteados inicialmente, obteniendo un producto seguro, escalable, de fácil mantenimiento y simple de usar.

También podemos decir que este sistema queda abierto a la incorporación de futuras actualizaciones para satisfacer las necesidades de los usuarios cuando los procesos así lo requieran.

Algunas de nuestras ideas para una actualización y agregación de funciones al sistema seria incorporar inteligencia artificial para el reconocimiento automático de imágenes al sacar fotos de los insectos. Ademas se podría desarrollar un termómetro con una conexión mediante Bluetooth al dispositivo, para medir la temperatura del agua al momento de realizar un registro, ya que este parámetro es importante para el análisis futuro de la biodiversidad.

\label{chap:conclusiones}



% Glosario
\printnoidxglossary[style=altlist]


%% Bibliografía
%%%%%%%%%%%%%%%%%%%%%%%%%%%%%%%%%%%%%%%%%%%%%%%%%%%%%%%%
%   |------------------------------------------|       %
%   | Web App embebida en dispositivos móviles |       %
%   |  para la gestión de registros sobre la   |       %
%   |   contaminación de afluentes y ríos.     |       %
%   |                                          |       %
%   |          Proyecto de graduación          |       %
%   |__________________________________________|       %
%                                                      %
%   Autores                                            %
%   -------                                            %
%                                                      %
% * Bruno, Ricardo Hugo (CX 1409686)                   %
%     rburnount@gmail.com                              %
% * Gómez Veliz, Kevin Shionen (CX 1411828)            %
%     ing.gomezvelizkevin@gmail.com                    %
%                                                      %
%   Tutor                                              %
%   -------                                            %
%                                                      %
% * Ing. Cohen, Daniel Eduardo                         %
%        dcohen.tuc@gmail.com                          %
%                                                      %
%   Cotutor                                            %
%   -------                                            %
%                                                      %
% * Ing. Nieto, Luis Eduardo                           %
%        lnieto@herrera.unt.edu.ar                     %
%                                                      %
%                                                      %
%%%%%%%%%%%%%%%%%%%%%%%%%%%%%%%%%%%%%%%%%%%%%%%%%%%%%%%%

% ********* Bibliografías ********** %

\begin{thebibliography}{99}

\bibitem{software1} Craig Larman. Applying UML and Patterns second edition.

\bibitem{ionic} Documentación oficial de Ionic.(\href{https://ionicframework.com/docs/}{https://ionicframework.com/docs/})

\bibitem{cordova} Documentación oficial de Apache Cordova.(\hrefhttps://cordova.apache.org/docs/en/latest/}{https://cordova.apache.org/docs/en/latest/})

\bibitem{express} Documentación oficial de Express.js.(\href{http://expressjs.com/}{http://expressjs.com/})

\bibitem{angular} Documentación oficial de Angular.(\href{https://angular.io/docs}{https://angular.io/docs})

\bibitem{software} Roger S. Pressman. Software Enginneering sixth edition.

\bibitem{wiki-en} Wikipedia en inglés (\href{http://en.wikipedia.org/}{http://en.wikipedia.org/}).

\bibitem{wiki-es} Wikipedia en castellano (\href{http://es.wikipedia.org/}{http://en.wikipedia.org/}).

\bibitem{wikilibroslatex} Wikilibros: Manual de \LaTeX.

\bibitem{latexcientifico} Gabriel Valiente Feruglio. Composición de textos científicos con \LaTeX. 1999.

\end{thebibliography}


\end{document}
