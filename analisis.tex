%%%%%%%%%%%%%%%%%%%%%%%%%%%%%%%%%%%%%%%%%%%%%%%%%%%%%%%%
%   |------------------------------------------|       %
%   |Aplicación de comercio electrónico para   |       %
%   |teléfonos móviles con S.O. Android        |       %
%   |                                          |       %
%   | Proyecto de graduación                   |       %
%   |__________________________________________|       %
%                                                      %
%   Autores                                            %
%   -------                                            %
%                                                      %
% * Soto, Paula Fabiana (CX05-0967-4)                  %
%     paulette255@gmail.com                            %
% * Vallejo, Sergio Daniel (CX05-0392-4)               %
%     vallejosergio@gmail.com                          %
%                                                      %
%   Tutor                                              %
%   -------                                            %
%                                                      %
% * Ing. Augusto Maximiliano Odstrcil                  %
%        modstrcil@gmail.com                           %
%                                                      %
%                                                      %
%%%%%%%%%%%%%%%%%%%%%%%%%%%%%%%%%%%%%%%%%%%%%%%%%%%%%%%%

\chapter{Disciplina de Análisis}
\label{chap:analisis}

\section{Vista de Casos de Uso}

La vista de casos de uso captura el comportamiento de un sistema, subsistema, clase o componente, como lo ve un usuario externo. Particiona la funcionalidad del sistema en transacciones significativas para los actores (usuarios idealizados) de un sistema. Las piezas de funcionalidad interactiva son llamadas ``casos de uso''. Un caso de uso describe una interacción entre actores como una secuencia de mensajes entre el sistema y uno o más actores. El término \emph{actor} incluye a personas, como también otros sistemas de computadora o procesos.

\section{Diagrama de Contexto}

\begin{figure}[H]
  \centering
   
    \includegraphics[width=1\textwidth]{imagenes/analisis/diagrama-contexto.png}
        %%Me parece que queda mejor sin el hfill
        %\hfill
    %\caption{epígrafe}
	\label{fig:casos-de-uso}
\end{figure}

\section{Diagramas de Casos de Uso}

\subsection{Diagrama de Casos de Uso general}

\begin{figure}[H]
  \centering
    \includegraphics[width=0.7\textwidth]{imagenes/analisis/casos-uso-general.png}
        %%Me parece que queda mejor sin el hfill
        %\hfill
	%\caption{epígrafe}
	\label{fig:casos-de-uso}
\end{figure}

\subsection{Diagrama de Casos de Uso de gestión de usuarios}

\begin{figure}[H]
  \centering
    \includegraphics[width=0.7\textwidth]{imagenes/analisis/casos-uso-usuario.png}
        %%Me parece que queda mejor sin el hfill
        %\hfill
    %\caption{epígrafe}
	\label{fig:casos-de-uso-usuario}
\end{figure}

\subsection{Diagrama de Casos de Uso de gestión de tienda}

\begin{figure}[H]
  \centering
    \includegraphics[width=0.7\textwidth]{imagenes/analisis/casos-uso-tienda.png}
        %%Me parece que queda mejor sin el hfill
        %\hfill
    %\caption{epígrafe}
    \label{fig:casos-de-uso-tienda}
\end{figure}


\section{Diagramas de Actividad}

\subsection{Autenticar}
\begin{figure}[H]
  \centering
    \includegraphics{imagenes/analisis/diagrama-actividad-autenticar.png}
        %%Me parece que queda mejor sin el hfill
        %\hfill
    \label{fig:diagrama-actividad-autenticar}
\end{figure}

\subsection{Registrar}

\begin{figure}[H]
  \centering
    \includegraphics[width=1\textwidth]{imagenes/analisis/diagrama-actividad-registrar.png}
        %%Me parece que queda mejor sin el hfill
        %\hfill
	\label{fig:diagrama-actividad-registrar}
\end{figure}

\subsection{Crear Tienda}

\begin{figure}[H]
  \centering
    \includegraphics{imagenes/analisis/diagrama-actividad-crear-tienda.png}
        %%Me parece que queda mejor sin el hfill
        %\hfill
    \label{fig:diagrama-actividad-crear-tienda}
\end{figure}

\subsection{Comprar Producto}

\begin{figure}[H]
  \centering
    \includegraphics[width=1\textwidth]{imagenes/analisis/actividad-comprar-producto.png}
        %%Me parece que queda mejor sin el hfill
        %\hfill
    \label{fig:diagrama-actividad-comprar-producto}
\end{figure}


\section{Diagrama de Clases del Dominio}

\begin{figure}[H]
  \centering
    \includegraphics[width=0.9\textwidth]{imagenes/analisis/clases-dominio.png}
        %%Me parece que queda mejor sin el hfill
        %\hfill
    %\caption{epígrafe}
    \label{fig:clases-dominio}
\end{figure}