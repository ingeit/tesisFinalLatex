%%%%%%%%%%%%%%%%%%%%%%%%%%%%%%%%%%%%%%%%%%%%%%%%%%%%%%%%
%   |------------------------------------------|       %
%   | Web App embebida en dispositivos móviles |       %
%   |  para la gestión de registros sobre la   |       %
%   |   contaminación de afluentes y ríos.     |       %
%   |                                          |       %
%   |          Proyecto de graduación          |       %
%   |__________________________________________|       %
%                                                      %
%   Autores                                            %
%   -------                                            %
%                                                      %
% * Bruno, Ricardo Hugo (CX 1409686)                   %
%     rburnount@gmail.com                              %
% * Gómez Véliz, Kevin Shionen (CX 1411828)            %
%     ing.gomezvelizkevin@gmail.com                    %
%                                                      %
%   Tutor                                              %
%   -------                                            %
%                                                      %
% * Ing. Cohen, Daniel Eduardo                         %
%        dcohen.tuc@gmail.com                          %
%                                                      %
%   Cotutor                                            %
%   -------                                            %
%                                                      %
% * Ing. Nieto, Luis Eduardo                           %
%        lnieto@herrera.unt.edu.ar                     %
%                                                      %
%                                                      %
%%%%%%%%%%%%%%%%%%%%%%%%%%%%%%%%%%%%%%%%%%%%%%%%%%%%%%%%

% ********* Introducción ********** %

%TODO: *Agregar Secciones de Android

\chapter{Introducción}
\label{chap:introduccion}

\begin{comment}
Requisitos del proyecto y los problemas a encarar
Razón de ser del proyecto: aplicación, investigación, etc.
Especificaciones generales del proyecto.


introducción es una sección inicial que establece el propósito y los objetivos de todo el contenido posterior del escrito. En general va seguido del cuerpo o desarrollo del tema, y de las conclusiones.

En la introducción normalmente se describe el alcance del documento, y se da una breve explicación o resumen de éste. También puede explicar algunos antecedentes que son importantes para el posterior desarrollo del tema central. Un lector al leer una introducción debería poder hacerse una idea sobre el contenido del texto, antes de comenzar su lectura propiamente dicha.

En artículos técnicos, la introducción generalmente incluye una o más sub-secciones estándar, como lo son el resumen o síntesis, el prefacio y los agradecimientos. Alternativamente, la sección de introducción puede ser un capítulo más del trabajo en sí, dividido en las sub-secciones anteriormente mencionadas. Cuando el libro se divide en capítulos numerados, por convención la introducción y cualquier otro asunto delante de las secciones de cuerpo o desarrollo no se enumeran (o se enumeran de manera distinta) y preceden al capítulo 1.

El concepto de introducción es independiente del contenido del documento al cual introduce. Siempre debe presentar el objeto o problema a desarrollar, ya este se trate de una especificación formal, un producto, un personaje o un ente cualquiera.
\end{comment}

La tecnología de los dispositivos móviles ha avanzado rápidamente en los últimos años, llegando a ser actualmente auténticas computadoras de bolsillo. La gran demanda por este tipo de dispositivos genera un gran interés por parte de empresas/instituciones que desean crear aplicaciones para un mercado en pleno auge, buscando aprovechar no sólo la gran cantidad de usuarios de estas plataformas, sino también la posibilidad de ofrecer funcionalidades y capacidades imposibles para sus procesos actuales.

La plataforma que actualmente posee una mayor cantidad de usuarios y mayor crecimiento es Android, debido a que se trata de un Sistema Operativo abierto que cualquier fabricante puede adaptar e instalar en sus dispositivos, que está en constante evolución, y que aporta gran cantidad de servicios y aplicaciones. 

Por ello, el objetivo del presente trabajo de graduación es aplicar dichas tecnologías, los conocimientos y competencias adquiridas a lo largo de la carrera, en la construcción de un producto de software que satisface las necesidades y genera valor agregado a un cliente determinado. Dicho producto es una aplicación web embebida en dispositivos móviles, que es el tipo de producto que experimenta un vertiginoso crecimiento en la actualidad.
determinado. 

\section{Objetivos del sistema}

Los principales objetivos del sistema a desarrollar son:

\begin{itemize}
    \item Permitir que los usuarios del sistema puedan realizar el estudio de campo de una manera rápida y eficiente valiéndose de la tecnología de un dispositivo móvil.
    \item El sistema cuenta de dos partes, una aplicación móvil para generar registros de las muestras del estudio de campo, y una aplicación web para gestionar y administrar dichos registros.
\end{itemize}

El sistema debe diseñarse para:

\begin{itemize}
    \item Asegurar la escalabilidad de los requisitos.
    \item Mantener de forma sencilla la plataforma.
    \item Promover la seguridad de la información en todas sus capas.
    \item Ser fácil de usar.
\end{itemize}

\section{Desarrollo de Software}



